% -*- Mode:TeX -*-

%% IMPORTANT: The official thesis specifications are available at:
%%            http://libraries.mit.edu/archives/thesis-specs/
%%
%%            Please verify your thesis' formatting and copyright
%%            assignment before submission. If you notice any
%%            discrepancies between these templates and the 
%%            MIT Libraries' specs, please let us know
%%            by e-mailing thesis@mit.edu

%% The documentclass options along with the pagestyle can be used to generate
%% a technical report, a draft copy, or a regular thesis. You may need to
%% re-specify the pagestyle after you \include cover.tex. For more
%% information, see the first few lines of mitthesis.cls. 

%\documentclass[12pt,vi,twoside]{mitthesis}
%%
%%  If you want your thesis copyright to you instead of MIT, use the
%%  ``vi'' option, as above.
%%
%\documentclass[12pt,twoside,leftblank]{mitthesis}
%%
%% If you want blank pages before new chapters to be labelled ``This
%% Page Intentionally Left Blank'', use the ``leftblank'' option, as
%% above. 

\documentclass[12pt,twoside, singlespace, vi]{mitthesis}
\usepackage{lgrind}
%% These have been added at the request of the MIT Libraries, because
%% some PDF conversions mess up the ligatures.  -LB, 1/22/2014
\usepackage{cmap}
\usepackage[T1]{fontenc}
\pagestyle{plain}

\usepackage{geometry}
\usepackage{pdflscape}

\usepackage{float}
\usepackage{graphicx}
\usepackage{acronym}
\usepackage{array}
\usepackage{rotating}
\usepackage[threshold=1]{csquotes}
\usepackage{setspace}
%\usepackage[table]{xcolor}
\usepackage{soul}
\usepackage{color}
\usepackage[usenames,dvipsnames,table]{xcolor}

\usepackage[inkscapeformat=eps]{svg}

\usepackage[sort&compress]{natbib}

\pdfpagewidth 8.5in
\pdfpageheight 11in

\usepackage{csvsimple}
\usepackage{booktabs}

\newcommand{\tabitem}{~~\llap{\textbullet}~~}

%\usepackage{silence}

\usepackage{longtable}
\setlength{\LTcapwidth}{\textwidth}
\def\@cline#1-#2\@nil{%
  \omit
  \@multicnt#1%
  \advance\@multispan\m@ne
  \ifnum\@multicnt=\@ne\@firstofone{&\omit}\fi
  \@multicnt#2%
  \advance\@multicnt-#1%
  \advance\@multispan\@ne
  \leaders\hrule\@height\arrayrulewidth\hfill
  \cr
  \noalign{\nobreak\vskip-\arrayrulewidth}}

\usepackage{siunitx}

\usepackage{hyperref}
\hypersetup{
    colorlinks=true,
    linkcolor=black,
    filecolor=magenta,      
    urlcolor=black,
    citecolor=black
    }

\newcommand{\hlc}[2][yellow]{ {\sethlcolor{#1} \hl{#2}} }

\newenvironment*{singlespcquote}
    {\begin{spacing}{1}\quote}
    {\endquote\end{spacing}}

\SetBlockEnvironment{singlespcquote}

\usepackage{enumitem}
\newenvironment{packed_enum}{
\begin{enumerate}
  \setlength{\itemsep}{-1pt}
  \setlength{\parskip}{0pt}
  \setlength{\parsep}{0pt}
}{\end{enumerate}}
\usepackage{multicol}

\usepackage{multirow}
\newcolumntype{L}[1]{>{\raggedright\let\newline\\\arraybackslash\hspace{0pt}}p{#1}}
\newcolumntype{C}[1]{>{\centering\let\newline\\\arraybackslash\hspace{0pt}}m{#1}}
\newcolumntype{R}[1]{>{\raggedleft\let\newline\\\arraybackslash\hspace{0pt}}m{#1}}
\makeatletter
\def\hlinewd#1{%
\noalign{\ifnum0=`}\fi\hrule \@height #1 %
\futurelet\reserved@a\@xhline} 
\makeatother
%% This bit allows you to either specify only the files which you wish to
%% process, or `all' to process all files which you \include.
%% Krishna Sethuraman (1990).

%\typein [\files]{Enter file names to process, (chap1,chap2 ...), or `all' to process all files:}
\def\all{all}
\ifx\files\all \typeout{Including all files.} \else %\typeout{Including only \files.} \includeonly{\files} \fi
%\singlespacing


\setcounter{secnumdepth}{4}



\begin{document}




% -*-latex-*-
% 
% For questions, comments, concerns or complaints:
% thesis@mit.edu
% 
%
% $Log: cover.tex,v $
% Revision 1.9  2019/08/06 14:18:15  cmalin
% Replaced sample content with non-specific text.
%
% Revision 1.8  2008/05/13 15:02:15  jdreed
% Degree month is June, not May.  Added note about prevdegrees.
% Arthur Smith's title updated
%
% Revision 1.7  2001/02/08 18:53:16  boojum
% changed some \newpages to \cleardoublepages
%
% Revision 1.6  1999/10/21 14:49:31  boojum
% changed comment referring to documentstyle
%
% Revision 1.5  1999/10/21 14:39:04  boojum
% *** empty log message ***
%
% Revision 1.4  1997/04/18  17:54:10  othomas
% added page numbers on abstract and cover, and made 1 abstract
% page the default rather than 2.  (anne hunter tells me this
% is the new institute standard.)
%
% Revision 1.4  1997/04/18  17:54:10  othomas
% added page numbers on abstract and cover, and made 1 abstract
% page the default rather than 2.  (anne hunter tells me this
% is the new institute standard.)
%
% Revision 1.3  93/05/17  17:06:29  starflt
% Added acknowledgements section (suggested by tompalka)
% 
% Revision 1.2  92/04/22  13:13:13  epeisach
% Fixes for 1991 course 6 requirements
% Phrase "and to grant others the right to do so" has been added to 
% permission clause
% Second copy of abstract is not counted as separate pages so numbering works
% out
% 
% Revision 1.1  92/04/22  13:08:20  epeisach

% NOTE:
% These templates make an effort to conform to the MIT Thesis specifications,
% however the specifications can change. We recommend that you verify the
% layout of your title page with your thesis advisor and/or the MIT 
% Libraries before printing your final copy.
\title{Using Integrated Earth Observation-Informed Modeling to Inform Sustainable Development Decision-Making}

\author{Jack Reid}
% If you wish to list your previous degrees on the cover page, use the 
% previous degrees command:
%       \prevdegrees{A.A., Harvard University (1985)}
% You can use the \\ command to list multiple previous degrees
%       \prevdegrees{B.S., University of California (1978) \\
%                    S.M., Massachusetts Institute of Technology (1981)}
\department{Program in Media Arts and Sciences}

% If the thesis is for two degrees simultaneously, list them both
% separated by \and like this:
% \degree{Doctor of Philosophy \and Master of Science}
\degree{Doctor of Philosophy in Media Arts and Sciences}

% As of the 2007-08 academic year, valid degree months are September, 
% February, or June.  The default is June.
\degreemonth{December}
\degreeyear{2022}
\thesisdate{Dec 21, 2022}

%% By default, the thesis will be copyrighted to MIT.  If you need to copyright
%% the thesis to yourself, just specify the `vi' documentclass option.  If for
%% some reason you want to exactly specify the copyright notice text, you can
%% use the \copyrightnoticetext command.  
%\copyrightnoticetext{\copyright IBM, 1990.  Do not open till Xmas.}

% If there is more than one supervisor, use the \supervisor command
% once for each.
\supervisor{Danielle R. Wood}{ Assistant Professor of Media Arts and Sciences}

% This is the department committee chairman, not the thesis committee
% chairman.  You should replace this with your Department's Committee
% Chairman.
\chairman{Tod Machover}{Chairman, Department Committee on Graduate Theses}

% Make the titlepage based on the above information.  If you need
% something special and can't use the standard form, you can specify
% the exact text of the titlepage yourself.  Put it in a titlepage
% environment and leave blank lines where you want vertical space.
% The spaces will be adjusted to fill the entire page.  The dotted
% lines for the signatures are made with the \signature command.
\maketitle

% The abstractpage environment sets up everything on the page except
% the text itself.  The title and other header material are put at the
% top of the page, and the supervisors are listed at the bottom.  A
% new page is begun both before and after.  Of course, an abstract may
% be more than one page itself.  If you need more control over the
% format of the page, you can use the abstract environment, which puts
% the word "Abstract" at the beginning and single spaces its text.

%% You can either \input (*not* \include) your abstract file, or you can put
%% the text of the abstract directly between the \begin{abstractpage} and
%% \end{abstractpage} commands.

% First copy: start a new page, and save the page number.
\cleardoublepage
% Uncomment the next line if you do NOT want a page number on your
% abstract and acknowledgments pages.
% \pagestyle{empty}
\setcounter{savepage}{\thepage}
\begin{abstractpage}
% $Log: abstract.tex,v $
% Revision 1.1  93/05/14  14:56:25  starflt
% Initial revision
% 
% Revision 1.1  90/05/04  10:41:01  lwvanels
% Initial revision
% 
%
%% The text of your abstract and nothing else (other than comments) goes here.
%% It will be single-spaced and the rest of the text that is supposed to go on
%% the abstract page will be generated by the abstractpage environment.  This
%% file should be \input (not \include 'd) from cover.tex.
This work aims to demonstrate the viability of a methodology for supporting local, sustainable development decision-making through the development of clearer linkages between environmental modeling and societal impact, with a particular emphasis on the use of earth observation data. To accomplish this, it explores the efficacy and difficulties of \textbf{\textit{collaboratively developing}} a \textbf{\textit{systems-architecture-informed}}, multidisciplinary \textbf{\textit{\acs{gis} decision support system}} for \textbf{\textit{sustainable development}} applications that makes significant use of \textbf{\textit{earth observation data}}. 

This is done through the development and evaluation of \acp{dss} for two applications: (1) mangrove forest management and conservation in the state of Rio de Janeiro, Brazil; and (2) coronavirus response in six regions around the world. In both cases, the methodology involves the application of the System Architecture Framework, which includes analyzing the stakeholders to inform the design of the \ac{dss} in question. Other components of the methodology are developing the \ac{dss} through a collaborative process with stakeholders; pursuing targeted analyses; and evaluating the usefulness of both the \ac{dss} and the development process through interviews, workshops, and other feedback mechanisms.

All of this takes place under the umbrella of the \ac{evdt} Modeling Framework for combining remote observation and other types of data to inform decision-making in complex socio-environmental systems, particularly those pertaining to sustainable development. As the name suggests, \ac{evdt} integrates four models into one tool: the Environment; Human Vulnerability and Societal Impact; Human Behavior and Decision-Making; and Technology Design for earth observation systems including satellites, airborne platforms and in-situ sensors. The data from each of these domains is used by established models in each domain, which are adapted to work in concert to address the needs identified during the stakeholder analysis. The capabilities provided by this framework will improve the management of earth observation and socioeconomic data in a format usable by non-experts, while harnessing cloud computing, machine learning, economic analysis, complex systems modeling, and model-based systems engineering.


\end{abstractpage}

% Additional copy: start a new page, and reset the page number.  This way,
% the second copy of the abstract is not counted as separate pages.
% Uncomment the next 6 lines if you need two copies of the abstract
% page.
% \setcounter{page}{\thesavepage}
% \begin{abstractpage}
% % $Log: abstract.tex,v $
% Revision 1.1  93/05/14  14:56:25  starflt
% Initial revision
% 
% Revision 1.1  90/05/04  10:41:01  lwvanels
% Initial revision
% 
%
%% The text of your abstract and nothing else (other than comments) goes here.
%% It will be single-spaced and the rest of the text that is supposed to go on
%% the abstract page will be generated by the abstractpage environment.  This
%% file should be \input (not \include 'd) from cover.tex.
This work aims to demonstrate the viability of a methodology for supporting local, sustainable development decision-making through the development of clearer linkages between environmental modeling and societal impact, with a particular emphasis on the use of earth observation data. To accomplish this, it explores the efficacy and difficulties of \textbf{\textit{collaboratively developing}} a \textbf{\textit{systems-architecture-informed}}, multidisciplinary \textbf{\textit{\acs{gis} decision support system}} for \textbf{\textit{sustainable development}} applications that makes significant use of \textbf{\textit{earth observation data}}. 

This is done through the development and evaluation of \acp{dss} for two applications: (1) mangrove forest management and conservation in the state of Rio de Janeiro, Brazil; and (2) coronavirus response in six regions around the world. In both cases, the methodology involves the application of the System Architecture Framework, which includes analyzing the stakeholders to inform the design of the \ac{dss} in question. Other components of the methodology are developing the \ac{dss} through a collaborative process with stakeholders; pursuing targeted analyses; and evaluating the usefulness of both the \ac{dss} and the development process through interviews, workshops, and other feedback mechanisms.

All of this takes place under the umbrella of the \ac{evdt} Modeling Framework for combining remote observation and other types of data to inform decision-making in complex socio-environmental systems, particularly those pertaining to sustainable development. As the name suggests, \ac{evdt} integrates four models into one tool: the Environment; Human Vulnerability and Societal Impact; Human Behavior and Decision-Making; and Technology Design for earth observation systems including satellites, airborne platforms and in-situ sensors. The data from each of these domains is used by established models in each domain, which are adapted to work in concert to address the needs identified during the stakeholder analysis. The capabilities provided by this framework will improve the management of earth observation and socioeconomic data in a format usable by non-experts, while harnessing cloud computing, machine learning, economic analysis, complex systems modeling, and model-based systems engineering.


% \end{abstractpage}

\clearpage
\hspace{0pt}
\vfill
\blockquote{\footnotesize
 God, grant me the insight to find and use models to understand the world around me, \newline The wisdom to acknowledge that they will someday fail, \newline And the strength to rid myself of them when it is apparent they no longer work.}

\hspace{50mm}-inspired by Ze Frank \& the Serenity Prayer
\normalsize
\vfill
\hspace{0pt}
\cleardoublepage

\section*{Acknowledgments}

Before proceeding onto this work, I would like to thank several individuals and communities. First and foremost is my wife, Rebecca, who has been unfailingly supportive of me throughout the endeavor that has been multiple MIT graduate programs. First in a long-distance relationship and then in person, she has consistently buoyed my hopes and sense of self-worth when I needed it most.

Next I would like to thank the continuing expertiment in cooperative living that is pika. It is there that I truly learned what a home is. Thanks for (almost) always having dinner ready at the end of the day and for filling the house with laughter. Similarly, I must thank the forbidden zone (tfz). This rotating cast of genuine characters kept me sane and happy throughout the pandemic even as we moved through three different houses. I do not know what else to say except that I genuinelly miss those of who you have left already and will geninuelly miss the rest of y'all whenever we spend more than a week apart.

I would of course be highly remiss without thanking my advisor, Prof. Danielle Wood. Thank you for building such a wonderful research community and allowing me to take part in it. You have provided a space for morally important work to be done, for those who are interested in space for more than military or scientific purposes to find a research home, and for neglected perspectives to find voice. I wish you the best of fortune in your career to come.

Similarly, I need to thank my other committee members, Prof. Sarah Williams and Prof. David Lagomasino, for mentoring me in the various fields that I needed to complete this work (and become a better person). Be it teaching me how to use Google Earth Engine one summer in Goddard or piling me why with critical GIS books, I appreciate the time and attention given to my education.

I also need to thank Dr. Donna Rhodes, my masters thesis advisor. She taught me a great deal about how to think critically about my research work and how to present it to an audience. Her level of engagement and enthusiasm for my work has been much appreciated. These lessons and encouragment have stuck with me as I have pursued other fields of study.

In a very direct and literal sense, this thesis could not have been completed without many, many hours of involvement and support from my various international collaborators. this includes (among others): Prof Joga Setiawan (Diponegoro University) and Dr. Hanifa Denny (Diponegoro University) lead coordination for the Indonesia Vida work; Prof Joaquin Salas (Centro de Investigación en Ciencia Aplicada y Tecnología Avanzada, Universidad Querétaro) and Mr. Alejandro Monsivais (Mexican Space Agency) led coordination for the Mexico Vida work; Jose Guiridi (Ministerio de Ciencia, Tecnología, Conocimiento e Innovación) led coordination for the Chile Vida effort; and Zolana Joao, Gilson Santos, Eduina Teodoro, and Joana Caetano (Management Office of the National Space Program) led coordination for the Angola Vida work. In particular, however, I must extend my gratitude and affection towards Mr. Felipe Mandarino of \ac{ipp} in Rio de Janeiro, Brazil. Felipe has remained actively engaged with my work for several years now, amid multiple changes in reesarch direction, a pandemic, changes in governments (both here in the US and there in Brazil). He gave me a place to work in Rio de Janeiro, showed me around, and introduced me to the various folks that I need to speak to for this research work. I doubt that I will ever be able to repay him sufficiently for this.

Finally, I wish to state the following:

\blockquote{MIT and this author acknowledge Indigenous Peoples as the traditional stewards of the land, and the enduring relationship that exists between them and their traditional territories. The land on which this work was performed and these words were written is the traditional unceded territory of the Wampanoag Nation, Massachusett, and Nipmuc peoples. We acknowledge the painful history of genocide and forced occupation of their territory, and we honor and respect the many diverse indigenous people connected to this land on which we gather from time immemorial.}

The above statement is adapted from the \ac{mit} \ac{ipac} statement in partnership with \ac{mit}'s \ac{aises}, the Native American Students Association (NASA)\footnote{To avoid confusion with the \acf{nasa}, the Native American Students Association will always be written out fully in this work.}, and other Native American MIT students. This statement is particularly important for my work and for Space Enabled, because, while we pursue various forms of equity, justice, and sustainable development, including with other indigenous groups, such work does not allow us to absolve ourselves of our sins and responsiblities, both past and ongoing. As of this writing, we have not worked directly with the Wampanoag or Nipmuc peoples (nor do I know if they wish to work with us). Until genuine actions and not mere words are taken to address these atrocities, more work is required.


%%%%%%%%%%%%%%%%%%%%%%%%%%%%%%%%%%%%%%%%%%%%%%%%%%%%%%%%%%%%%%%%%%%%%%
% -*-latex-*-

% Some departments (e.g. 5) require an additional signature page.  See
% signature.tex for more information and uncomment the following line if
% applicable.
% % -*- Mode:TeX -*-
%
% Some departments (e.g. Chemistry) require an additional cover page
% with signatures of the thesis committee.  Please check with your
% thesis advisor or other appropriate person to determine if such a 
% page is required for your thesis.  
%
% If you choose not to use the "titlepage" environment, a \newpage
% commands, and several \vspace{\fill} commands may be necessary to
% achieve the required spacing.  The \signature command is defined in
% the "mitthesis" class
%
% The following sample appears courtesy of Ben Kaduk <kaduk@mit.edu> and
% was used in his June 2012 doctoral thesis in Chemistry. 

\begin{titlepage}

\makesigtitle

\begin{large}
This dissertation/thesis has been reviewed and approved by the following committee members


\signature{Professor Danielle Wood}{Thesis Supervisor \\
   Benesse Corp. Career Development Asst Prof of Research in Education \\
   Program in Media Arts and Sciences \\
   Department of Aeronautics and Astronautics \\ 
   Massachusetts Institute of Technology}

\signature{Professor Sarah Williams}{Member, Thesis Committee \\
	Associate Professor of Technology and Urban Planning \\  
	Director of the Norman B. Leventhal Center for Advanced Urbanism \\
	Urban Science and Computer Science Program \\ 
	Department of Urban Studies and Planning \\
	Massachusetts Institute of Technology}

\signature{Professor David Lagomasino}{Member, Thesis Committee \\
   Assistant Professor of Coastal Studies \\ 
   Department of Coastal Studies \\
   East Carolina University}
   
\end{large}
\end{titlepage}


\pagestyle{plain}


  % -*- Mode:TeX -*-
%% This file simply contains the commands that actually generate the table of
%% contents and lists of figures and tables.  You can omit any or all of
%% these files by simply taking out the appropriate command.  For more
%% information on these files, see appendix C.3.3 of the LaTeX manual. 
\tableofcontents
\newpage
\listoffigures
\newpage
\listoftables
\newpage
\chapter*{List of Acronyms}

\begin{acronym}[HyperLEAVES] \itemsep0pt \setlength{\parskip}{0pt}
\acro{aiaa}[AIAA]{American Institute of Aeronautics and Astronautics}
\acro{cbers}[CBERS]{China-Brazil Earth Resources Satellite Program}
\acro{ceos}[CEOS]{Committee on Earth Observation Satellites}
\acro{cpr}[CPR]{common-pool resources}
\acro{dem}[DEM]{Digital Elevation Model}
\acro{dsm}[DSM]{Digital Surface Model}
\acro{dss}[DSS]{Decision Support System}
\acro{dtm}[DTM]{Digital Terrain Model}
\acro{eo}[EO]{earth observation}
\acro{eoc}[EOC]{Earth Observation Center}
\acro{eos}[EOS]{earth observation system}
\acro{epa}[EPA]{Environmental Protection Agency}
\acro{esa}[ESA]{European Space Agency}
\acro{evdt}[EVDT]{Environment, Vulerability, Decision-Making, Technology}
\acro{fema}[FEMA]{Federal Emergency Management Agency}
\acro{geo}[GEO]{Group of Earth Observations}
\acro{fews}[FEWS NET]{Famine Early Warning Systems Network}
\acro{gis}[GIS]{geographic information system}
\acro{gisc}[GISc]{geographic information science}
\acro{gpm}[GPM]{Global Precipitation Measurement}
\acro{grace}[GRACE]{Gravity Recovery and Climate Experiment}
\acro{icesat2}[ICESat-2]{Ice, Cloud, and land Elevation Satellite 2}
\acro{jaxa}[JAXA]{Japan Aerospace Exploration Agency}
\acro{leed}[LEED]{Leadership in Energy and Environmental Design}
\acro{lidar}[LIDAR]{light detection and ranging}
\acro{lunr}[LUNR]{Land Use and Natural Resources Inventory}
\acro{mdg}[MDG]{Millenium Development Goal}
\acro{modis}[MODIS]{Moderate Resolution Imaging Spectroradiometer}
\acro{nasa}[NASA]{National Aeronautics and Space Administration}
\acro{noaa}[NOAA]{National Oceanic and Atmospheric Administration}
\acro{ngo}[NGO]{non-governmental organization}
\acro{ostp}[OSTP]{Office of Science and Technology Policy}
\acro{ota}[OTA]{Office of Technology Assessment}
\acro{pgis}[PGIS]{participatory geographic information system}
\acro{ppgis}[PPGIS]{public participation geographic information system}
\acro{ppbs}[PPBS]{Planning-Programming-Budgeting System}
\acro{sdg}[SDG]{Sustainable Development Goal}
\acro{servir}[SERVIR]{Sistema Regional De Visualizaci\'{o}n Y Monitoreo De Mesoam\'{e}rica}
\acro{swot}[SWOT]{Surface Water Ocean Topography}
\acro{un}[UN]{United Nations}
\acro{usaid}[USAID]{United States Agency for International Development}
\acro{usgs}[USGS]{United States Geological Survey}
\acro{viirs}[VIIRS]{Visible Infrared Imaging Radiometer Suite}

%\acro{atlas}[ATLAS]{Advanced Topographic Laser Altimeter System}
%\acro{dsm}[DSM]{Digital Surface Model}
%\acro{dtm}[DTM]{Digital Terrain Model}
%\acro{cots}[COTS]{commercial off-the-shelf}
%\acro{eo}[EO]{Earth Observation}
%\acro{eo1}[EO-1]{Earth Observing-1}
%\acro{fov}[FOV]{field-of-view}
%\acro{glas}[GLAS]{Geoscience Laser Altimeter System}
%\acro{hyperleaves}[HyperLEAVES]{Hyperspectral-Lidar Elevation and Vegetation Satellites}
%\acro{hyspiri}[HsypIRI]{Hyperspectral Infrared Imager}
%\acro{icesat2}[ICESat-2]{Ice, Cloud, and land Elevation Satellite 2}
%\acro{leo}[LEO]{Low Earth Orbit}
%\acro{lidar}[lidar]{light detection and ranging}
%\acro{mwir}[MWIR]{mid-wave infrared}
%\acro{nasa}[NASA]{National Aeronautics and Space Administration}
%\acro{nasem}[NASEM]{National Academies of Science, Engineering, and Medicine}
%\acro{ndvi}[NDVI]{Normalized Difference Vegetation Index}
%\acro{nir}[NIR]{near-infrared}
%\acro{swapc}[SWaP-C]{Size, Weight, Power, and Cost}
%\acro{sar}[SAR]{sythetic aperture radar}
%\acro{srtm}[SRTM]{Shuttle Radar Topography Mission}
%\acro{swir}[SWIR]{short-wave infrared}
%\acro{tir}[TIR]{thermal infrared}
%\acro{un}[UN]{United Nations}
%\acro{unced}[UNCED]{United Nations Conference on Environment and Development}
%\acro{uncsd}[UNCSD]{United Nations Conference on Sustainable Development}
\end{acronym}


%% This is an example first chapter.  You should put chapter/appendix that you
%% write into a separate file, and add a line \include{yourfilename} to
%% main.tex, where `yourfilename.tex' is the name of the chapter/appendix file.
%% You can process specific files by typing their names in at the 
%% \files=
%% prompt when you run the file main.tex through LaTeX.
\chapter{Introduction}

Initial introduction

\section{Research Questions}

\section{Framing}

"Relying on metaphors as the foundation for policy advice can lead to results substantially different from those presumed to be likely... One can get trapped in one's own intellectual web. When years have been spent in the development of a theory with considerable power and elegance, analysts obviously will want to apply this tool to as many situations as possible... Confusing a model with the theory of which it is one representation can limit applicability still further." \cite{ostromGoverningCommonsEvolution2015}

instead use "theoretical pluralism" \cite{turkleEmpathyDiariesMemoir2021}

An anecdote from Jane Jacob's \textit{The Death and Life of Great American Cities} \cite{jacobsDeathLifeGreat2016}:

\blockquote{Consider, for example the orthodox planning reaction to a district called the North End in Boston. This is an old, low-rent area merging into the heavy indsutry of the waterfront, and it is officially considered Boston's wost slum and civic shame... When I saw the Norht End again in 1959, I was amazed at the change. Dozens and dozens of buildings had been rehabilitated... The general street atmosphere of buoyancy, friendliness, and good health was so infectious that I began asking directions of people just for hte fun of getting in on some talk. I had seen a lot of Boston in the past couple of days, most of it sorely distressing, and this struck me, with relief, as the healthiest place in the city... I called a Boston planner I know.

"Why in the world are you down in the North End?" he said, "That's a slum!... It has among the lowest delingquency, disease, and infant mortality rates in the city. It hwas has the lowest ratio of rent to income in the city... the child population is just above average for the city, on the nose. The death rate is low, 8.8 per thousand, against the average city rate of 11.2. The TB death rate is very low, less than 1 per ten thousand, can't understand it, it's lower even than Brookline's. In the old days the North End used to be the city's worst spot for tuberculosis, but all that has changed. Well, they must be strong people. Of course it's a terrible slum"

"You should have more slums like this," I said.} 



\section{Methodology}

\section{Structure of Thesis}






%\section{Section sample 1}


%\begin{enumerate}
%  \item Item 1.
%  \item Item 2.
%  \item Item 3.
%\end{enumerate}
%
%
%
%\begin{eqnarray*}
%a_i & = & a_j + a_k \\
%a_i & = & 2a_j + a_k \\
%a_i & = & 4a_j + a_k \\
%a_i & = & 8a_j + a_k \\
%a_i & = & a_j - a_k \\
%a_i & = & a_j \ll m \mbox{shift}
%\end{eqnarray*}
%instead of the multiplication.  For example, you could use:
%\begin{eqnarray*}
%r & = & 4s + s\\
%r & = & r + r
%\end{eqnarray*}
%Or by xx:
%\begin{eqnarray*}
%t & = & 2s + s \\
%r & = & 2t + s \\
%r & = & 8r + t
%\end{eqnarray*}

%% This is an example first chapter.  You should put chapter/appendix that you
%% write into a separate file, and add a line \include{yourfilename} to
%% main.tex, where `yourfilename.tex' is the name of the chapter/appendix file.
%% You can process specific files by typing their names in at the 
%% \files=
%% prompt when you run the file main.tex through LaTeX.
\chapter{Motivation, Theory, and Frameworks}

The question of motivation includes several elements. Why sustainable development? Why remote observation data? Why systems architecture and engineering? Why these particular case studies? Why me? This chapter will address these questions as well as respond to several critiques of the chosen approach.

\section{Motivation}

\subsection{Personal Motiviation}

My background may make my interest in this work, collaborative modeling for sustainable development, seem a bit odd. Almost all of my prior work was eithre funded by the military or done directly for the military, from improving weapons testing procedures at Sandia National Labs to defense acquisition policy analysis for my masters at MIT to summers spent at the RAND Corporation helping the US military to plan aircraft and air defense acquisitions, to name just a few. My one purely private sector job, as an engineering intern at a fossil fuel refinery on the coast of Texas, was hardly emblematic of a great commitment to sustainability.

In another way, however, I am merely following in a well trod, if problematic, pathway. Like Jennifer Light \cite{lightWarfareWelfareDefense2005}, I was exposed to scenario planning and other forms of decision support tools during summers working at the RAND Corporation. And like numerous MIT scholars (Jay Forrester, Norbert Wiener, Joseph Weizenbaum, the list goes on) I have pivoted from, or perhaps built upon, my experience with military engineering to instead tackle societal development problems. The convergence of this two institutions is not something to be passed over. "Support for applying cybernetic principles to research on nonliving systems emerged from organizations... studying management, engineering and control. RAND and MIT stood at the forefront of this trend. With their heritage of mathematical innovation and ties to the armed forces... these and cognate institutions offered ideal laboratories to transform cybernetic principles into management practices." \cite{lightWarfareWelfareDefense2005} 

There is a key difference between me and my predecessors (or so I would like to believe). While some of these  (Weizenbaum in particular \cite{lightWarfareWelfareDefense2005}) came to have doubts about the consequences of applying military-originated technical methods to civilian applications, most of them did not. They resolutely swept aside complications, objections, and planning professionals to solve the problems that they identifed in their own way. They built names and careers in this way, but also caused significant harms in their hubris, as I will discuss more later in this chapter.

My background and perspective is somewhat different from them in certain ways, however. My undergraduate mechanical engineering degree was obtained alongside a philosophy degree. My masters aerospace engineering degree was obtained alongside a technology policy degree. And now, over the course of my doctorate, I have invested time in taking development and planning classes, reading foundational texts, and engaging with my antiracist and anticolonialist peers in Space Enabled. My education in matters of urban development and ethics is thus more significant than the one-month seminars that MIT and the University of California at Berkeley provided to aerospace workers in 1971 to prepare them for local government positions \cite{lightWarfareWelfareDefense2005}. 

Finally, I have the history, both positive and negative, of my MIT predecessors to inform my actions, in a way that they did not. For these reasons, I often find myself more sympathetic to the contemporary critics of some of these MIT scholars, such as Ida Hoos \cite{hoosSystemsAnalysisPublic1983}. This, of course, raises the question of why I am proceeding with this work anyways then.	

The answer to that is multifaceted. For one, I believe that the relevant fields have advanced signficantly and, to some extent at least, have learned from their prior missteps. This is elaborated on in my detail throughout this chapter. Another aspect is that I (and my advisor evidently) believe that my knowledge and systems engineering in general does still have something to offer humanity beyond building rockets. Additionally, I and my peers, with our particular commitment to the principles outlined earlier, may have an important role to play on influencing the aerospace/systems engineering communities, urging them to curb their worst impulses and learn from their own history. Finally, it is because I want to be of service to humanity. As my aerospace education and career progressed, I found myself increasingly faced with only two options: "pure" scientific work or defense work. Reluctant to choose either, I was being quickly sucked into the gravity of the default: the aerospace defense sector. The Space Enabled Research Group, and the work detailed in this thesis in particular, offered me a third option, to apply my skills and interests to directly help humans on Earth. Now all that is left to is to do it.

\subsection{Why Sustainable Development?}

Before exploring the various methodologies and theoretical frameworks used in this work, it is worth exploring exactly what it is we are hoping to accomplish and why it is important. We need to talk about sustainable development.
 
\subsubsection{What is Sustainable Development?}

The term \textit{sustainable development} is simultaneously one that invites immediate, intuitive understanding, and one that reminds frustratingly vague. \textit{Sustainable} here means something somewhat more specific than its general definitions of "able to be maintained or kept going" or "capable of being supported or upheld." Instead, it refers to something more specific, commonly associated with the natural environment: "pertaining to a system that maintains its own viability by using techniques that allow for continual reuse" \cite{DefinitionSustainableDictionary}. As to what "system" we are talking about here, the "development" half of sustainable development, we mean generally, human society and wellbeing. This is of course still much too vague, so let us turn to the first official use of the term, which was in the 1987 report by the \ac{un} World Commission on Environment and Development, commonly known as the Brundtland Report, after the name of the chair of the commission. This report defined sustainable development as "the development that meets the needs of the present without compromising the ability of future generations to meet their own needs" \cite{worldcommissiononenvironmentanddevelopmentOurCommonFuture}. We have now helpfully clarified the time scale under which this system needs to "maintain its own viability" but still have done little to clarify what aspects of human society are included within "development." 

In 1992, the \ac{un} provided more detail in the Rio Declaration on Environment and Development. In this report, they said that "human beings are at the centre of concerns for sustainable development. They are entitled to healthy and productive life in harmony with nature." Furthermore, they state that eradicating poverty is "an indispensable requirement for sustainable development" and environmental protection constitutes "an integral part of the development process" \cite{unitednationsconferenceonenvironmentanddevelopmentRioDeclarationEnvironment1992}. So we know have several key components, including human health and productivity, the protection of the natural environment, and the elimination of poverty. It is still unclear whether this is a complete list, however, and, if so, what are the connections between these components.

Official clarification would come in 2002, at the UN World Summit on Sustainable Development in Johannesburg. There we get the following \cite{worldsummitonsustainabledevelopmentPlanImplementationWorld2002}:

\blockquote{These efforts will also promote the integration of the three components of sustainable development — economic development, social development and environmental protection — as interdependent and mutually reinforcing pillars. Poverty eradication, changing unsustainable patterns of production and
consumption, and protecting and managing the natural resource base of economic and social development are overarching objectives of and essential requirements for sustainable development.}

We now have three linked components along with a set of potential actions for implementation. This is the definition that would stick and become commonplace. From here has built intellectual fields and massive multi-governmental interventions. Jeffery Sachs describes this further, "As an intellectual pursuit, sustainable development tries to make sense of the interactions of three complex systems: the world economy, the global society, and the Earth's physical environment... Sustainable development is also a normative outlook of the world, meaning that it recommends a set of goals to which the world should aspire... SDGs call for socially inclusive and environmentally sustainable growth." \cite{sachsAgeSustainableDevelopment2015}

Questions remain, however, why all this effort? And what are these SDGs?

\subsubsection{Why is Sustainable Development Important?}

As former \ac{un} Secretary-General Ban Ki-moon put it:"Sustainable development is the central challenge of our times." \cite{sachsAgeSustainableDevelopment2015}

\begin{figure}[h]
	\centering
	\includegraphics[scale=0.4]{Figures/chap2/Planetary_Boundaries.png}
	\caption[Planetary Boundaries]{Planetary Boundaries. From \cite{rockstromSafeOperatingSpace2009}}
	\label{fig:boundaries}
\end{figure}

Gini coefficient

Risks of business as usual



\begin{table}[h]
\begin{minipage}{\textwidth}

\caption[Estimated impacts of "business-as-usual" by domain and region.]{Estimated impacts of "business-as-usual" by domain and region. Adapated from \cite{rockstromSustainableDevelopmentPlanetary2013} and \cite{sachsAgeSustainableDevelopment2015} \protect\footnotemark[1]}
\label{table:bau}
\begin{center}
\tiny
\begin{tabular}{ | L{2cm} | C{1cm} | C{1cm} | C{1cm} | C{1cm} | C{1cm} | C{1cm} | C{1cm} | C{1cm} | } \hline
& North America & Latin America \& Caribbean & Europe & Middle East \& North Africa & Sub-Saharan Africa & South \& Central Asia & Southeast Asia \& Pacific & East Asia \\ \hline

Food Insecurity \& Malnutrition & & & & \cellcolor{red!25} H & \cellcolor{red!25} H & \cellcolor{red!25} H & \cellcolor{yellow!25} M  & \cellcolor{yellow!25} M \\ \hline

Poverty & & & & \cellcolor{yellow!25} M & \cellcolor{red!25} H & \cellcolor{red!25} H & \cellcolor{yellow!25} M  & \cellcolor{yellow!25} M  \\ \hline

Land Use Change & & \cellcolor{red!25} H & & & \cellcolor{red!25} H & \cellcolor{yellow!25} M  & \cellcolor{yellow!25} M  & \cellcolor{yellow!25} M  \\ \hline

Soil Degradation & & & & \cellcolor{yellow!25} M  & \cellcolor{red!25} H & \cellcolor{red!25} H   & \cellcolor{yellow!25} M  & \cellcolor{red!25} H   \\ \hline

Water Shortage & \cellcolor{yellow!25} M & & & \cellcolor{red!25} H & \cellcolor{red!25} H & \cellcolor{red!25} H & \cellcolor{yellow!25} M  & \cellcolor{yellow!25} M \\ \hline

Water \& Air Pollution & \cellcolor{yellow!25} M & & \cellcolor{yellow!25} M & \cellcolor{yellow!25} M  & &\cellcolor{red!25} H  & \cellcolor{red!25} H  & \cellcolor{red!25} H \\ \hline

Biodiversity Loss & \cellcolor{yellow!25}  & \cellcolor{red!25} H  & \cellcolor{yellow!25} M & \cellcolor{yellow!25} M  & \cellcolor{yellow!25} M & \cellcolor{yellow!25} M  & \cellcolor{red!25} H  & \cellcolor{red!25} H \\ \hline

Sea Level Rise & \cellcolor{yellow!25} M & \cellcolor{yellow!25} M & \cellcolor{red!25} H & \cellcolor{yellow!25} M & \cellcolor{red!25} H & \cellcolor{red!25} H &\cellcolor{red!25} H  & \cellcolor{red!25} H \\ \hline

Ocean Acidification &  \cellcolor{yellow!25} M & \cellcolor{red!25} H & \cellcolor{red!25} H & \cellcolor{yellow!25} M & \cellcolor{yellow!25} M  & \cellcolor{yellow!25} M & \cellcolor{red!25} H & \cellcolor{yellow!25} M \\ \hline
\end{tabular}
\end{center}
\end{minipage}
\end{table}

\footnotetext[1]{It should be noted that, despite the latter of these two sources citing the former, the two sources differ in noticeable ways, with no explanation provided in either document. Where they are in conflict, I have chosen to use the latter source. In the former source, there is also a error: Ocean Acidification in the Middle East / North Africa is listed as "H" but the cell is in yellow. The correct entry is not known, so I have gone with "M" in yellow here in order to avoid overstatement.}

\begin{figure}[h]
	\centering
	\includegraphics[scale=0.35]{Figures/chap2/services_wellbeing.png}
	\caption[Linkages between categories of ecosystem services and compotents of human wellbeing]{Linkages between categories of ecosystem services and compotents of human wellbeing. Adapted from \cite{reidEcosystemsHumanWellbeing2005}}
	\label{fig:services_wellbeing}
\end{figure}

Historically, surveys and quantifications of the natural environment focused primarily, or even entirely, on resources that could be extracted and exploited for economic benefit. In early forest surveys, for instance, "Missing... were all those parts of trees, even revenue-bearing trees, which might have been useful to the population but whose value could not be converted into fiscal receipts." \cite{scottSeeingStateHow2020}

\subsubsection{What about the Sustainable Development Goals?}


Jeffrey Sachs, "Our new era will son be described by new global goals, the \acp{sdg}. \cite{sachsAgeSustainableDevelopment2015}




Talk about \acp{mdg} and \acp{sdg}, pull from my previously written articles, also \cite{unitednationsWhoWillBe2013}




\subsection{Why Earth Observation Imagery}

This possibility was known even from the earliest days of remote observation. As Jennifer Light recorded, "one proponent [from the last 1940s] explained, photointerpretation data did not directly provide 'social data,' yet they were 'pertinent to social research needs in so far as such 'physical data' have meaningful sociological correlates." \cite{lightWarfareWelfareDefense2005} In the succeeding decades, the degree to which humans have altered the surface of our planet has only increased and, as a result, we can now also infer a great deal more about humans from images of that surface.

This was not always the case. For most of the early history of satellite observation, imagery was kept highly classified and zealously guarded, to the extent that Congressman George Brown Jr., who was integral in the establishment of the US \ac{ostp}, the \ac{epa}, and the \ac{ota}, resigned from his post in the House Intelligence Panel in protest over the enforced secrecy in even discussing the topic \cite{healyRepBrownQuits1987, barry1992mappings}

In the 1970s and early 1980s, Landsat data was a government-managed operation that provided products at a low-cost, based primarily on the cost of reproduction. In the 1980's, however, the program was transfered to a private entity and prices were increased by more than an order of magnitude and significant copyright restrictions were put in place. \cite{mchaffieManufacturingMetaphors1994}


By the early 1970's five rationales for using satellite imagery in city planing had become widespread \cite{lightWarfareWelfareDefense2005}:
\begin{enumerate}[itemsep=0pt,parsep=0pt]
	\item{It offers a synoptic, total view of the complex system in a given area.}
	\item{Satellites provide repetitive, longitudinal coverage.}
	\item{Satellite inventories were more efficient and up-to-date than ground surveys.}
	\item{Remote sensing was objective.}
	\item{Satellites produced digital imagery that could be easiliy combined with ground-based data in novel \acp{gis}.}
\end{enumerate}

Despite these rationales, cities and metropolitan areas largely elected not to use satellite imagery for several decades, choosing instead to rely on aerial imagery and ground-based surveys \cite{lightWarfareWelfareDefense2005}. The reasons for this are many, but probably include that many of these rationeles were overstated for their day. Insufficient resolution and inconsistent coverage limited intra-urban use. While satellite imagery provides a wonderful decades-long longitudinal dataset now, it did not at the time. Satellite imagery was still heavily dependent on human photointerpretation, undermining the argument that the data was "objective" in any meaningful sense. Finally the cost and specialization required to effectively use the data limited its ability to be combined with other datasets. Black-and-white aerial imagery provided sufficient resolution, oblique angles, and immediate interpretability to even the untrained eye. Plus cities were compact enough that the advantages of scale offered by satellites largely did not come into play. Ultimately, while \ac{gis} technology was readily adopted by cities, satellite imagery was not. \cite{lightWarfareWelfareDefense2005}.

NASA "did not go a long way toward incorporating remote sensing into day-to-day practices in city planning agencies. This was compounded by the fact that far more academics than local government officials participated in these experiments, providng applications of satellite data that were almost always a step removed from urban mangers' needs." \cite{lightWarfareWelfareDefense2005} [Talk about EVDT's efforts to fix this, plus that SERVIR suggests that maybe NASA has changed too.]

One of the first use of non-visual imagery for such applications was in 1970, when the city of Los Angeles used aerial infrared imagery to identify unsound housing, and, by 1972, had integrated this imagery with other datasets into a digital decision support system for assessing urban blight \cite{lightWarfareWelfareDefense2005}.

"The geography agenda is distorted by being data-led... The first law of geographical information: the poorer the country, the less and the worse the data." (\cite{overton1991further} as paraphrased by \cite{taylorGeographicInformationSystems1994}) 

\subsection{Why GIS and Decision Support?}

\ac{gis} originated in the 1960's and 70's with the Canda Geographic Information System and the US Bureau of the Census. \cite{goodchildGeographicInformationSystems1994}

"GIS allows geographers to integrate diverse types of data over different spatial scales from the regional to the global, while the advanced capabilities of GIS for organizing and displaying these data transofrm the geographer's view of the world." (\cite{tomlinsonPRESIDENTIALADDRESSGEOGRAPHIC1989} as paraphrased in \cite{vereginComputerInnovationAdoption1994})

One key moment in the development of \ac{gis} as we know it, was ESRI's creation of the shapefile format (which links geometries with data in a standardized, if somewhat limited, fashion) in the late 1980's, and, more importantly, their open publishing of the format, allowing others to create and manipulate such files \cite{goodchildModelingEarth2011}. 

As Maguire et al. wrote back in 2001, "It is not fanciful to suggest that by the end of the century \ac{gis} will be used every day by everyone in the developed world for routine operations." \cite{maguireGeographicalInformationSystems1991}

\ac{pgis}: Refer to macro-micro framework from Table 2.1, pg.17. What parts this thesis covers and what parts we envision EVDT covering in the long term. Also refer to EAST2 model on pg. 21 \cite{jankowskiGISGroupDecision2001}

\begin{table}[h]
\caption[Generic macro-micro, participtory decision strategy]{Generic macro-micro, participtory decision strategy. Adapted from \cite{jankowskiGISGroupDecision2001}}
\label{table:macro-micro}
\begin{center}
\begin{tabular}{ L{3cm} L{4cm}  L{4cm} L{4cm}} \hline
& \multicolumn{3}{c}{\textit{Macro-phases in a decision strategy}}  \\ \cline{2-4}

\textit{Micro-activities in a decision strategy} & \textbf{\textit{1. Intelligence}} about values, objectives, and criteria & \textbf{\textit{2. Design}} of a set of feasible options &  \textbf{\textit{3. Choice}} about recommendations \\ \hline

\textbf{A. Gather...} & issues to develop and refine \textbf{value trees} as a basis for objectives & \textbf{primary criteria} as a basis for option generation & \textbf{values, criteria, and option list scenarios} for an evaluation \\ \hline

\textbf{B. Organize...} & \textbf{objectives} as a basis for criteria and constraints & and apply approaches(es) for \textbf{option generation} & approaches to \textbf{priority and sensitivity analyses} \\ \hline

\textbf{C. Select...} & \textbf{criteria} to be used in analysis as a basis for generating options & the \textbf{feasible option list} & \textbf{recommendation} as a prioritized list of options \\ \hline

\textbf{D. Review...} & criteria, \textbf{resources, constraints,} and \textbf{standards} & \textbf{option set(s)} in line with resources, constraints, and standards & \textbf{recommendation(s)} in line with original \textbf{value(s), goal(s),} and \textbf{objectives} \\ \hline

\end{tabular}
\end{center}
\end{table}

\begin{figure}[h]
	\centering
	\includegraphics[scale=0.4]{Figures/chap2/east2.jpg}
	\caption[Enhanced Adaptive Structuration Theory 2 (EAST2)]{Enhanced Adaptive Structuration Theory 2 (EAST2). Adapted from \cite{jankowskiGISGroupDecision2001}}
	\label{fig:east2}
\end{figure}


Jankowski and Nyerges lay out seven common design requirements for spatial decision support tools \cite{jankowskiGISGroupDecision2001}:

\begin{enumerate}
    \setlength{\itemsep}{0pt}%
    \setlength{\parskip}{0pt}%
	\item{A spatial decision support system for collaborative work should offer decisional guidance to users in the form of an agenda.}
	\item{A system should not be restrictive, allowing the users to select tools and procedures in any order.}
	\item{A system should be comprehensive within the realm of spatial decision problems, and thus offer a number of decision space exploration tools and evaluation techniques.}
	\item{The user interface should be both process-oriented and data-oriented to allow an equally easy access to task-solving techniques, as well as maps and data visualization tools.}
	\item{A system should be capable of supporting facilitated meetings and hence, allow for the information exchange to proceed among group members, and between group members and the facilitator. It should also allow space- and time-distributed collaborative work by facilitating information exchange, electronic submission of solution options, and voting through the internet.}
	\item{A system functionality should include extensive multiple criteria evaluation capabilities, sensitivity analysis, specialized maps to support the enumeration of preferences and comparison of alternative performance, voting, and consensus building tools.}
	\item{A system should provide ncessary functionality to support needs of an advanced user without overwhelming a novice who needs a user-guiding interface.}
\end{enumerate}



\begin{figure}[h]
	\centering
	\includegraphics[scale=0.4]{Figures/chap2/GIScience.png}
	\caption[Overview of Geographical Information Science]{Overview of Geographical Information Science. From \cite{fotheringhamGeographicInformationScience2007}}
	\label{fig:giscience}
\end{figure}

\begin{figure}[h]
	\centering
	\includegraphics[scale=0.4]{Figures/chap2/GIScience.png}
	\caption[The marketplace for geographic data]{The marketplace for geographic data. From \cite{cowenAvailabilityGeographicData2007}}
	\label{fig:marketplace}
\end{figure}


the meeting arrangment that EVDT supports, \ref{table:meeting_arrangements}

\begin{table}[h]
\caption[Different types of meeting arrangements]{Different types of meeting arrangements. Adapted from \cite{jankowskiGISGroupDecision2001}}
\label{table:meeting_arrangements}
\begin{center}
\begin{tabular}{ L{3cm} L{5.5cm}  L{5.5cm}}  \hline
 & \textit{Same time} &\textit{Different time}  \\ \cline{2-3}
\textbf{\textit{Same place}} & \textbf{Conventional Meeting} \qquad \textit{Advantage:} 
\vspace{-5mm}
\begin{itemize}
    \setlength{\itemsep}{0pt}%
    \setlength{\parskip}{0pt}%
	\item{face-to-face expressions}
	\item{immediate response}
\end{itemize} &
\textbf{Storyboard meeting} \qquad \textit{Advantage:} 
\vspace{-5mm}
\begin{itemize}
    \setlength{\itemsep}{0pt}%
    \setlength{\parskip}{0pt}%
	\item{scheduling is easy}
	\item{respond anytime}
	\item{leave-behind note}
\end{itemize} 
\\
& \textit{Disadvantage:} 
\vspace{-5mm}
\begin{itemize}
    \setlength{\itemsep}{0pt}%
    \setlength{\parskip}{0pt}%
	\item{scheduling is difficult}
\end{itemize} &
\textit{Disadvantage:} 
\vspace{-5mm}
\begin{itemize}
    \setlength{\itemsep}{0pt}%
    \setlength{\parskip}{0pt}%
	\item{meeting takes longer}
	\item{difficult to maintain in the long run}
\end{itemize} 
\\ \hline

\textbf{\textit{Different place}} & \textbf{Conference call meeting} \qquad \textit{Advantage:} 
\vspace{-5mm}
\begin{itemize}
    \setlength{\itemsep}{0pt}%
    \setlength{\parskip}{0pt}%
	\item{no need to travel}
	\item{immediate response}
\end{itemize} &
\textbf{Distributed meeting} \qquad \textit{Advantage:} 
\vspace{-5mm}
\begin{itemize}
    \setlength{\itemsep}{0pt}%
    \setlength{\parskip}{0pt}%
	\item{scheduling is convenient}
	\item{no need to travel}
	\item{submit response anytime}
\end{itemize} 
\\
& \textit{Disadvantage:} 
\vspace{-5mm}
\begin{itemize}
    \setlength{\itemsep}{0pt}%
    \setlength{\parskip}{0pt}%
	\item{limited personal perspective from participants}
	\item{meeting protocols are difficult to interpret}
	\item{difficult to maintain meeting dynamics}
\end{itemize} &
\textit{Disadvantage:} 
\vspace{-5mm}
\begin{itemize}
    \setlength{\itemsep}{0pt}%
    \setlength{\parskip}{0pt}%
	\item{meeting takes longer}
	\item{meeting dynamics are different from normal meeting ("netiquette" instead of face-to-face etiquette)}
\end{itemize} 
\\ \hline
\end{tabular}
\end{center}
\end{table}

Levels of decision support \cite{jankowskiGISGroupDecision2001}:

\begin{enumerate}[itemsep=0pt,parsep=0pt]
	\item{\textit{Basic information handling support}}
%	\vspace{-5mm}
		\begin{enumerate}[itemsep=0pt,parsep=0pt,topsep=0pt, partopsep=0pt]
			\item{Information management}
			\item{Visual aids}
			\item{Group collaboration support}
		\end{enumerate}
%	\vspace{-5mm}
	\item{\textit{Decision Analysis Support}}
%	\vspace{-5mm}
		\begin{enumerate}[itemsep=0pt,parsep=0pt,topsep=0pt, partopsep=0pt]
			\item{Option modeling}
			\item{Choice models}
			\item{Structured group process techniques}
		\end{enumerate}
%	\vspace{-5mm}
	\item{\textit{Group reasoning support}}
%	\vspace{-5mm}
		\begin{enumerate}[itemsep=0pt,parsep=0pt,topsep=0pt, partopsep=0pt]
			\item{Judgement refinement/amplification techniques}
			\item{Analytical reasoning methods}
		\end{enumerate}
\end{enumerate}

EVDT is partially based on Tomlin's cartographic modeling concept from 1990. "The cartographic modleing approach attemps to generalize and standardize the analytic and synthetic capabilities of geopgrahical information systems. It does so by decomposing data, data-processing tasks, and data-rpocessing control notation into elementary components that can be recomposed with relative ease and great flexibility." \cite{tomlinGISCartographicModeling2012}

"Scenario planning is a method of long-term strategic planning that creates representations of multiple, plausible futures of the system of interest." It arose in military and corporate strategies \cite{goodspeedScenarioPlanningCities2020} 

[Refer to my old RAND citations about scenario planning]

"A scenario-based strategic plan is... appropriate for vision, framework, comprehensive, system, and redevelopment plans and for those with long time horizons and low or moderate detail." \cite{goodspeedScenarioPlanningCities2020} 

\textit{Oregon Scenario Planning Guidelines} proposes a six-step process for scenario planning ( \cite{oregonsustainabletransportationinitiativeScenarioPlanningGuidelines2017} as paraphrased by \cite{goodspeedScenarioPlanningCities2020}:

\begin{enumerate}[itemsep=0pt,parsep=0pt]
	\item{Create a framework for the scenario planning process.}
	\item{Select evaluation criteria.}
	\item{Set up for scenario planning: evaluation tools ,data, and building blocks.}
	\item{Develop and evaluate base-year conditions and reference case.}
	\item{Develop and evaluate alternative scenarios}
	\item{Select the preferred scenario}
\end{enumerate}

Goodman describes four primary kinds of planning support models \cite{goodspeedScenarioPlanningCities2020}: 

\begin{enumerate}[itemsep=0pt,parsep=0pt]
	\item{Generic Systems Models: Developing a typically non-spatial abstract representation of a system and analyzing how it functions. System dynamics is a classic example.}
	\item{Economic and Demographic Models: The set of techniques that focus on changes in employment of particular sectors and in populuation of different characteristics. Klosterman is the classic text on these methods. \cite{klostermanCommunityAnalysisPlanning1990}}
	\item{Place-Type Development and Analysis: Tools used to simulate future outcomes based on land use, zoning, population density, etc. CommunityViz is an example of this \cite{walkerPlannersGuideCommunityViz2017}.}
	\item{Urban Systems Models: Essentially a combination of generic systems modeling and place-type development and analysis models to accurately represent spatial phenomena over time, such as transportation networks and organic growth. Examples include cellular automata and spatial interaction models.}
\end{enumerate}

The city of Chicago has developed \ac{gis} apps to visualize food deserts and develop maps of where new supermarkets are both needed and economically vialbe. \cite{goldsmithResponsiveCityEngaging2014} 

"Non-human animals are rarely considered within the realms of social theory, and yet... animals can be regarded as a 'marginal social group' that is 'subjected to all manner of socio-spatial inclusions and exclusions.'" (\cite{philolAnimalsGeographyCity1995,westcoatBringingAnimalsBack1995,wolchAnimalGeographiesPlace1998}as paraphrased in \cite{harrisRethinkingMapsMorethanhuman2011}) I would argue that plants, particularly non-domesticated plants, could similarly be regarded as a marginal social group that is often neglected by decision support tools.

The motivation for combining so many variables from different disciplines stems from both push and pull factors. The push factors are the simple increase in availability of data, as has already been described, along with the increase in the interoperability of the variables (which the work described in this thesis is trying to help contribute to). The primary pull factor is our increased understanding of - and appreciation for - the complex relationships between these domains, relationships that were previously ignored in analyses \cite{gaheganMultivariateGeovisualization2007}. 

\subsection{Why Collaborative and Open Source?}

"It is the geomorphologist who is best able to choose the dta model for representation of terrain in a GIS, not the computer scientist or the statistician, and it is the urban geographer who is best able to advicse on how to represent the many facets of hte urban environment in a GIS designed for urban planning." \cite{goodchildGeographicInformationSystems1994}

"Knowing \textit{how} refers to the ability to do something...Knowing \textit{that} refers to knowledge about how something works." \cite{curryGeographicalInformationSystems1994} Vida seeks to help people move from how to that through collaboration and open source. 

In 1994, McHaffie wrote that "for billiions the possibility of accessing the best technolog yand information made available through digital communications network will always be a luxury. Cartographic information, digital or otherwise, becomes a commodity in its mass production and marketing." \cite{mchaffieManufacturingMetaphors1994}

"It is impossible to have sustainable and equitable development without free access to reliable and accurate information." \cite{benmouffokInformationDecisionMaking1993}

"Without equitable access to GIS data and technology, small users, local governments, nonprofit community agencies, and nonmainsream groups are significantly disadvantaged in their capacity to engage in the decision-making process." (\cite{edney1991strategies} as paraphrased in \cite{harrisPursuingSocialGoals1994})

Participatory \ac{gis} has been successfully used in South Africa to try and overcome such issues of inequal access and use of \ac{gis} technology. They sought to achieve five specific objectives \cite{harrisPursuingSocialGoals1994}: 

\begin{enumerate}[itemsep=0pt,parsep=0pt]
	\item{Enhanced community/development planner interaction in a research and policy agenda setting}
	\item{The integration of local knowledge with exogenous technical expertise.}
	\item{The spatial representation of relevant aspects of local knoweledge.}
	\item{Genuine community access to, and use of, advanced technology for rural land reform.}
	\item{The education of "expert" rural land use planners about the importane of popular participation in policy formulation and implementation.}
\end{enumerate}

Expansion of choice is valuable for both intrinsic (for its own sake) and instrumental (to attain preferred positions) reasons \cite{senFreedomChoiceConcept1988}. \ac{evdt} primarily addresses the latter reason, by allowing for more selecting a preferred choice that might otherwise have been overlooked.

Giles Deleuze on Michel Foucault: "You have taught us something absolutely fundamental: The indignity of speaking on someone else's behalf" \cite{sheridanMichelFoucaultWill2003}


While collaborations certainly can introduce additional difficulties, including cultural conflicts and issues of interpersonal trust, they are also immensely rewarding \cite{tullochInstitutionalGeographicInformation2007}.

While \ac{ppgis} refers to involving the public in both the creation and the use of data, in practice, it has tended to focus primarily on the former, thus situating it as a form of \ac{gisc} rather than application \cite{weinerParticipatoryGeographicInformation2007}. For example, in Washington state in 2002 , several American Indian tribes were using \ac{gis} technology to "inventory, analyze, map, and make descisions regarding tribal resources... includ[ing] timber production, grazing and farm land, water rights, wildlife, native plants, cultural sites, environmental data and hazardous site monitoring, historical preservation, health and human resources." \cite{bondCherokeeNationTribal2002}

Digital tools that provide open data (1) frees data from bureaucractic constraints, allowing real time combination of data from different souces; (2) construct a loop between government and the community in which cooperation builds respect continuously; (3) Enable two-way communication, promotiving collaboration \cite{goldsmithResponsiveCityEngaging2014}.

"The open source movement at its core stands for the development of source code... in a completely open and free way. Pragmatically, this manifests itself as a methodology of making code freely available to anyone who may wish to access it for any purpose, unconditionally. Concurrently, open source is for many a philosophical approach to software development, and is see as the only truly sustainable approach to software development... In both its execution as a model for making possible new forms of collaborative work, and its philosophical underpinnings of sustainability and openness, it is an essential component in and fluence upon a computer-based mapping solution." \cite{williamsonTheirworkDevelopmentSustainable2011}

"Map studies needs to open the 'black boxes' of mapping software, to start to interrogate algorithms and databases, and in particular to investigate the production of ready-made maps that appear almost magically on the interfaces of gadgets and devices we carry and use everyday, often without much overt thought about how they work and whose map they project onto their interface." \cite{dodgeMappingModesMethods2011}


\subsection{Why Systems Engineering?}

Refer to Crawley, deWeck, de Neufville, Rhodes, my earlier papers, etc.

"Sustainable development is also a science of complex systems" \cite{sachsAgeSustainableDevelopment2015}

"Sustainable Development involves not just one but four complex interacting systems. It deals with a global economy...; it focuses on social interactions...; it analyzes the changes in complex Earth systems...; and it sutdies the problems of governance." \cite{sachsAgeSustainableDevelopment2015} [Compare this to the EVDT framework]

The growth and development of cities is a complex system. Much work has been done using cellular automata and fractals to model them \cite{battyCitiesComplexity2005}

Urban planners have been seeking to develop useful indices and indicators akin to those used in engineering and remote observation contexts for decades (Section 1, Cahpter 3 of \cite{boyceFrameworkDefiningApplying1972}


"Futures planning as described and prescribed by futurists is different from planning \textit{for} te future; it is an attempt to manipulate or plan \textit{the} future. A basic characteristic of this orientation is the use of such terms as "designing," "inventing," or even "making" the future. When the future is being planned for, rather than designed, the implication is that the planner is trying to make specific and limited accomodations to the broad and overall characteristics of the future he considers either immutable or too formidable to be fundamentally rearranged or restructured." \cite{robinsonDecisionmakingUrbanPlanning1972}

"If alternatives are not carefully related to goals and objectives there is the real danger that they will either fail to reflect certain important issues which the planning process to being used to study, or worse still, be almost irrelevant... Alternatives must reflect the goals sought; the means must reflect the ends." \cite{mcloughlinChartingPossibleCourses1972}



Position \ac{evdt} using the different dimensions of models proposed in \cite{harrisQuantitativeModelsUrban1972}:


Sachs argues that two specific tools are important for implementing the \acp{sdg}: backcasting and technology road-mapping \cite{sachsAgeSustainableDevelopment2015}. Systems engineering is well equiped to address both of these.

\begin{enumerate}[itemsep=0pt,parsep=0pt]
	\item{descriptive vs. analytic}
	\item{holistic vs. partial}
	\item{macro vs. micro}
	\item{static vs. dynamic}
	\item{deterministic vs. probabilistic}
	\item{simultaneous vs. sequential (directly calculate the output or go through intermediate phases)}
\end{enumerate}


In order for cost-benefit analysis to maximize economic welfare, the following conditions must be met \cite{krutillaWelfareAspectsBenefitCost1961}:

\begin{enumerate}[itemsep=0pt,parsep=0pt]
	\item{Opportunity costs are borne by beneficiaries in  such wise as to retain the initial income distribution}
	\item{The initial income distribution is in some sense "best}
	\item{The marginal social rates of transformation between any two commodities are everywhere equal to their corresponding rates of substitution except for the area(s) justifying the intervention in question}
\end{enumerate}

More details modeling, as well as breaking down specific costs and benefits (as opposed to converting them to monetary terms and summing them) and attributing them to specific goals, can circumvent these constraints, though at the cost of increased complexity \cite{hillGoalsAchievementMatrixEvaluating1972}.

This work does not directly incorporate mechanisms for multi-stakeholder negotiation or tradespace exploration, but it is amenable to extension with such mechanisms (refer to SEAri research)

The Law of requisite variety from the field of cybernetics says that the variety (the number of elements or states) of the control device must be at least equal to that of the disturbances \cite{ashbyRequisiteVarietyIts1991}. Any development plan is going to fall far short of the variety expressed by human society and the natural environment. Planning efforts must then make reliance on the natural homeostasis behavior of such systems and of more flexible, ad hoc measures not specified in the plan in order to make up the difference in variety. \cite{mcloughlinSystemGuidanceControl1972}

Refer to epoch-era analysis and tie that in to scenario planning history.

Goodchild defines six different \ac{gis} data field model types and states that "no current \ac{gis} gives its users full access to all six":

\begin{enumerate}
    \setlength{\itemsep}{0pt}%
    \setlength{\parskip}{0pt}%
	\item{Sample randomly located points (e.g. weather stations, \ac{lidar} data)}
	\item{Sample randomly from a grid of regularly space points (e.g. many data validation studies}
	\item{Divide the area into a grid in which each rectangular cell records the average, total, or dominant value; i.e. raster data (e.g. satellite imagery)}
	\item{Divide the area into homogenous regions and record the average, total, or dominant value in each area (e.g. census data, soil maps)}
	\item{Record the locations of lines of fixed values (e.g. contour or isopleth maps}
	\item{Divide the area into irregular shaped triangles and assume the field varies linearly within each (e.g. some \acp{dem})}
\end{enumerate}

In the mid 90's, \ac{GIS} had several limitations \cite{goodchildGeographicInformationSystems1994}:

\begin{itemize}
    \setlength{\itemsep}{0pt}%
    \setlength{\parskip}{0pt}%
	\item{Two-dimensional, with some excursions into three}
	\item{Static, with some limited support for time dependence}
	\item{Limited capabilities for representing forms of interaction between objects}
	\item{A diverse and confusing set of data models}
	\item{Dominated by the map metaphor}
\end{itemize}

\section{EVDT Framework}

Is not itself a means of planning and implementing projects. It is not a full life-cycle tool such as \ac{ppbs} \cite{hatryCriteriaEvaluationPlanning1972}

\section{Intended Use Cases / Applications}

While \ac{evdt} does not include any concrete spatial scale requirements, it is often the most straightforward to apply to it at a a relatively local scale, like much of the early history of \ac{gis} applications \cite{tullochInstitutionalGeographicInformation2007}. Most of the applications to date have been at the area of a metropolitan area or that of a small province.


Commonly has to do with \acp{cpr}. Talk about the three common ways of managing \acp{cpr}: Central management, privatization, self-management. Bring in Table \ref{table:cpr_design}  showing design principles of long-enduring self-management institutions. Refer to successful aspects of the water basin in California (incremental and sequential process to reduce the costs of local institutional supply, shared information at each step, intermediate benefits from initial investments were realized prior to larger investments, transformed structure of incentives within which fuure strategic decisions can be made) (pg. 137. \cite{ostromGoverningCommonsEvolution2015}

\begin{table}[h]
\caption[Design principles illustrated by long-lasting CPR institutions]{Design principles illustrated by long-lasting \ac{cpr} institutions. Adapted from \cite{ostromGoverningCommonsEvolution2015}}
\label{table:cpr_design}
\begin{center}
\begin{tabular}{ L{0.5cm} L{8cm}} \hline
1. & Clearly defined boundaries \\
2. & Congruence between appropriation and provision rules and local conditions \\
3. & Collective-choice arrangements \\
4. & Monitoring \\
5. & Graduated sanctions \\
6. & Conflict-resolution mechanisms \\
7. & Minimal recognition of rights to organize \\
\multicolumn{2}{l}{\textit{For CPRs that are parts of larger systems:}} \\
8. & Nested enterprises \\ \hline
\end{tabular}
\end{center}
\end{table}

We also 

Harris et al. have pointed out that reliance on mapping products to designate certain geographic areas for conservation has several negative consequences, including: 

\begin{itemize}
    \setlength{\itemsep}{0pt}%
    \setlength{\parskip}{0pt}%
	\item{solidifying a notion that humans and non-human others are, and should be, separate.}
	\item{privileging those voices and perspectives that have access and expertise related to Western cartographic approaches and GIScience in conservation debates.}
	\item{favoring those spaces, ecosystems, and natures that may be "more mappable" for protection over other areas.}
	\item{cementing an overly-limited territorial approach to conservation, in ways that potentially sideline non-territorial approaches.}
	\item{consolidating an overly-fixed and static approach to sonservation, rather than enabling approaches that may be more seasonal, fluid, or appropriate for shifting and evolving ecological conditions and needs.}
\end{itemize}

\section{Critiques}

\begin{itemize} \setlength{\itemsep}{0pt} \setlength{\parskip}{0pt} 
	\item{Systems engineering is an inherently elitist methodology whose primary use is to defend the oppressive status quo and eke out greater "efficiencies" with little regard for societal consequences.}
	\item{Western-run technocratic planning and international development perpetuates colonialism, typically fails in its own goals, and merely destroys traditional communities.}
	\item{Sustainable development, as it is commonly used, is essentially meaningless and the \acp{sdg} are likewise such a potpourri of targets and indicators that they have little influence on what would have happened anyways.}
	\item{The effectiveness of scenario planning and most other forms of decision support is ambiguous at best, despite their long history. Another research project in this vein is thus fundamentally flawed and is not real science.}
	\item{Technology itself is at best a major contributor, if not the source of most of the problems you seek to address If you truly want to save the Earth and stop oppression, you should abandon technology rather than doubling down on it.}
\end{itemize}



\subsection{Technology is Bad}

"Keynes's point is that technology is crucial for the long haul of economic development." \cite{sachsAgeSustainableDevelopment2015}

"This slow rate of progress, or lack of progress, was due to two reasons - to the remarkable absence of important technical improvements and to the failure of capital to accumulate." [Cite Keynes 1930, page 2, as seen on page 74 of Sachs]

"Choosing the right technologies, we can achieve continued economic growth and also honor the planetary boundaries." \cite{sachsAgeSustainableDevelopment2015}

"Planning theorists have toooften accepted Habermas's view that technology is primarily associated with technical rather than moral rationality, which leads them to overlook technology's potential normative dimension... Even choosing a digital tool requires making value-laden judgements about what issues matter enough to be analyzed. Becaues digital tools typically inherit the worldviews and assumptions of their creators, even well-meaning applications of them can inhibit potentially valuable new ideas or critical perspectives." Goodspeed proposes the term \textit{tool of inquiry} to "describe the ideal in which tools are continually shaped, used, and tested by public users. \cite{goodspeedScenarioPlanningCities2020}

Passage by John Pickles \cite{picklesGroundTruthSocial1994}:

\blockquote{The Western trop of a public space in which people (usually "men") of good faith join in debate about their future, appropriated by industrial and urban forms of modernity as a mythic image of a democratic culture of debate and negotiation predicated on individual autonomy, private property, and state power has more recently been further appropriated by the news and communication media through their claim to be the embodiment of the modern civic arena. This trope of public space is now being reappropriated by the electronic age as its wish image - the promise and possibility of "information." The putative openness of new electronic information media and the rhetoric of "voice," "openness," and "information" - the trope of reasoned, open, uncoerced discourse in a public place - is appropriated to the project of social development and private profit.

But, like all highways, the information highway requires points of access, capital investment, navigation skills, and spatial and cultural proximity for effective use. Like the automobile highway, the information highway fosters new rounds of creative destruction and differentiates among users and between users and nonusers. It brings regions of diference under a common logic and technology, and throuhg differential access and use exacerbates old and crates new patterns of social and economic differentiation. While for some, information means the provision of alternatives and the satisfication of choice (even if a "choice" signifies a socially constructed yet now naturalized whim of the wealthy consumer), for others this postindustrialism (and its attendant postmodern cultural forms) must still be seen in the context of a political economy of graft, monopolism, and uneven dvelopment.

Such processes of territorial coonizations, globalization, and production of new scales of actoin contrast sharply with a technocultural ideology of enhanced autonomy and self-actualization, and severly complicates the assessment of hte rleationship between technological innovation and social change.}

Passage by Penley and Ross \cite{penleyTechnoculture1991}:

\blockquote{Wary, on the one hand, of the disempowering habit of demonizing technology as a satanic mill of domination, and weary, on the other, of the postmodernist celebrations of the technological sublime, we selected contributors whose critical knowledge might help provide a realistic assessment of the politics - the dangers \textit{and} the possibilities - that are currently at stake in those cultural practices touched by advanced technology.}

"GIS and informatics do open virtual space of 'real' social interaction, new communities of dialogue, and new interactive settings... Systems of informatics provide a potential source of counterhegemonic social action, and GIS... offers a diverse array of practical possibilities... Informatics are seen as a potential liberator of socially and politically marginalized groups, and thus a source of democratizing power for these newly networked groups." \cite{picklesRepresentationsElectronicAge1994}

I align myself with those who feel that "Even though the funding or research and development... of GIS and other imaging systems has come primarily from business, state, and military sources, advocates of the progressive potential information and imaging technologies argue that access is hard to deny, networks are difficult to control, information is readily accessible and used by individuals and groups with limited budgets and expertise, and the ability to use the technology in depth permits groups like environmental organizations to counter claims by polluters about their environmental impacts, by developers about likely local effects of runoff and ground water, and so on... GIS enables communities to make better decisions by providing access to more and better information. It offers more powerful tools for local planning agenceisl if offers exciting possibilities for data coordination, access, and exchange; and it permits more efficient allocation of resources, and a more open rational decision-making process." \cite{picklesRepresentationsElectronicAge1994} [maybe add a caveat talking about "rational"]

As pointed out by Pickles, historically within the GIS research community and its predecessors, there has been a certain "technocratic myopia" and unwillingness to consider novel, insurgent uses of GIS that has led critics to label it as an "inherently conservative form of analysis." \cite{picklesRepresentationsElectronicAge1994}

It was only in the late 1980's did scholars, informed primarily by Michel Foucault and Karl Marx, start challenging the idea that "cartography produces maps of truth in an objective, neutral, scientific fashion." \cite{kitchinThinkingMaps2011}

Cite Langdon Winner somewhere in here

"Mappings do not represent geographics of ideas; rather they effect actualization... Maps remake 'territory over and over again, each time with new and diverse consequences.'" (\cite{cornerAgencyMappingSpeculation1999} as paraphrased by \cite{kitchinThinkingMaps2011}

In some ways, we want to avoid making a seemless tool, as "the most significant impacts of technology tend to occur when the technology becomes indistringuishable from the fabric of every day life" (\cite{weinerComputer21stCentury1991} as paraphrased in \cite{vereginComputerInnovationAdoption1994}). This is not, unfortunately, not sufficient. "We all tend to defer to machines, which can seem more neutral, more objective" even when they are actively warning us of their limitations and fallibility \cite{eubanksAutomatingInequalityHow2018}.

Even the most ardent supporters of digital tools and data collection often fall into a sort of technological determinism. Stephen Goldsmith and Susan Crawford, who did a great deal to implement such technologies in New York City and Indianapolis, wrote that "the process of collection is not going to stop. We think, it fact, that it would be shortsighted and probably impossible to halt this natural evolution. That is all the more reason, then, to carefully establish policies covering data access, data security, and transparency with respect to its collections." \cite{goldsmithResponsiveCityEngaging2014}

"It is also not clear that geography's diversity is a flaw instead of a great strength. Geography is inherently eclectic because the discipline is defined only by a perspective on the world. However, those who advocate the computer as a means to unify geography have a particular conception of the discipline in mind, an empirical and pragmatic one that is by no means universally accepted." \cite{vereginComputerInnovationAdoption1994}

National mapping in the US originated in motives that were explicity of means of resource exploitation and control. \cite{mchaffieManufacturingMetaphors1994}

"Perhaps the 'frightened Africans' who once 'threw spears at an Aero Service aircraft' or the 'suspicious moonshiners in Appalachia' who 'took a few rifle shots' at aerial mappers did so not because the intentions of the mappers were 'not always understood,' but because those intenions, and the powerful forces being them, were understood only too well." \cite{mchaffieManufacturingMetaphors1994}

Cite the million dollar blocks \cite{kurganCloseDistanceMapping2013}, redlining visualizations, etc.

Smithsonian curator Lucy Fellowes: "Every map is someone's way of getting you to look at the world his or her way." \cite{henriksonPowerPoliticsMaps1994} We think EVDT is a means of getting people to think about sustainable development about antiracism, etc. 

For a clear example of a technologically produced geographic object having enormous postive impact, we can look no further than NASA's famous Blue Marble image, which, while perhaps more iconic than cartographic, is still undeniably a geospatial object, a map even, that has essentially created both "one-world" discourse and "whole-earth" discourse \cite{propenCartographicRepresentationConstruction2011},  

Similarly, the Sierra Club has made significant use of Google Earth in their efforts to garner support for conservation efforts in the US Arctic National Wildlife Refuge and elsewhere \cite{propenCartographicRepresentationConstruction2011}. 

As Krygier and Wood so playfully illustrated, maps are, fundamentally, propositions about that world that are asserting a fact and promoting an action. Because of this "you must accept responsibility for the realities you create with maps." \cite{krygierCeEstPas2011}

"The very notion that technologies are neutral must be directly challenged as a misnomer." \cite{nobleAlgorithmsOppressionHow2018}

"Design is purposeful in that it forges both pathways and boundaries in its instrumental and cultural use." (\cite{paceyCultureTechnology1983} as paraphrased in \cite{nobleAlgorithmsOppressionHow2018})

"Those who have hte power to design systems - classification or technical - hold the ability to prioritize hierarchical schemes that privilege certain tyeps of information over others." \cite{nobleAlgorithmsOppressionHow2018}

\subsection{Scenario Planning and Decision-Support is unfounded}


Often, however, so-called "strategic planning" is anything but. "A strategic plan might more closely resemble a project plan, with long lists of specific proposals and policies... many have relatively short time frames. Scenario planning may not make sense for these plans." \cite{goodspeedScenarioPlanningCities2020}

Many projections have been bad

\begin{figure}[h]
	\centering
	\includegraphics[scale=0.35]{Figures/chap2/dresden_projections.png}
	\caption[Population changes in Dresden compared to various projections]{Population changes in Dresden compared to various projections. From \cite{wiechmannErrorsExpectedAligning2008}}
	\label{fig:dresden_population}
\end{figure}


While forecasting can be problematic as it constitutes "someone else's understanding and judgement crystallized in a figure that then becomes a substitute for thinking," scenario planning instead allows users to "develop their own feel for the nature of the system, the forces at work within it, the uncertanties that underlies the laternative scenarios, and the concepts useful for interpreting key data." \cite{wackScenariosShootingRapids1985}

The evidence for such practices as scenario planning is decidedly mixed. Goodspeed's review of scenario planning use in urban planning and environmental research resulted in only modest benefits, with use in management being more unambiguously positive \cite{goodspeedScenarioPlanningCities2020}. His own study of impact of a scenario planning project in Lockhart, Texas, which corrected some of the flaws he identified in many previous studies, confirmed that modest, but real positive changes are the result of scenario planning.

I readily acknowledge and embrace the fact that this work is predominantly a piece of design science, which aims to "design propositions, which inform speciic practices, artifacts, or tools", rather than 'conventional' science, which "primarily aims to describe, explain, or predict the world but not to change it." \cite{goodspeedScenarioPlanningCities2020}

In fact, there are significant reasons to avoid practicing "conventional science" in these domains as treating society as a laboratory can lead to significant harms and a "vivisectionaist" mentality \cite{banandynuriModernMedicineIts1990}.

Evidence that collaboration improves \ac{dss} functionality and usability \cite{goodspeedDeathLifeCollaborative2016, vonkSociotechnicalPSSDevelopment2010, brommelstroetPlanningSupportSystems2010a, ulibarriCollaborativeModelDevelopment2018} 

"Dewey famously distinguishes between a \textit{planned} society, which subordinates the present in pursuit of a rigid planned future, and a \textit{planning} society, which is intellectually preoccupied by the future but knows that only the present - and not the future can be controlled." (\cite{deweyHumanNatureConduct2007} as paraphrased by \cite{goodspeedScenarioPlanningCities2020}

Notably, one of the early successes at combining remote observation imagery with socioeconomic data, \ac{lunr}, elected in 1968 to not use the military-developed land use classification schemes, but instead to interview future users about their needs and to use the results from these interviews to develop a classification system tailored to the application. 

\subsection{Systems Engineering}


"Many who seek to harness computaitonal power for social justice tend to find affinity with systems engineering approaches to social problems. These perspectives assume that complex controversies can be solved by getting correct information where it needs to go as efficiently as possible. In this model, political conflict arises primarily rom a lack of information. If we just gather lal the facts, systems engineers assume, the correct answers to intractable policy problems like homelessness will be simple, uncontroversial, and widely shared. But, for better or worse, this is not how politics work." \cite{eubanksAutomatingInequalityHow2018}

US Vice President Herbert Humphrey said in 1968 that "The techniques that are going to put a man on the Moon are going to be exactly the techniques that we are going to need to clean up our cities." \cite{lightWarfareWelfareDefense2005}

"Trying to solve 'earthly problems,' especially urban problems through aerospace innovations had shown that 'transporting the astornauts from terra firma to land on the lunar sphere, travel hither and yon over its surface, and then back home to Houston' was a comparatively simple task." \cite{lightWarfareWelfareDefense2005}

In 1968 the RAND Corporation established a multi-year attempt to bring systems analysis and engineering to urban planning. Around the same time the \ac{aiaa} hosted meetings on urban technologies to bring aerospace expertise to bare on the urban crises of the time \cite{lightWarfareWelfareDefense2005}.

These applications were justified by several different rationale, chief among them were \cite{lightWarfareWelfareDefense2005}: 

\begin{itemize} \setlength{\itemsep}{0pt} \setlength{\parskip}{0pt} 
	\item{Computer simulations and related techniques were simply advances on the statistical models already widely used by the urban planning profession.}
	\item{The rise of cybernetics, with its cross-disciplinary control analogies, promised to unify disparate fields within urban planning and analysis, resulting in a unified understanding of cities.}
	\item{The use of these military innovations would transform urban planning and decision-making into scientific endeavors.}
\end{itemize}

As was discussed in Section \ref{sec:technocracy}, however, this last rationale tended to be more of an aesthetic argument, rather than one grounded in fact. It is a preference for the simple, regular, controlled, and "rational", all of which are, in fact, hideously artificial. Thankfully, among systems engineers this perspective is a view that is largely considered to be outmoded. Instead, in the guise of theories of complex systems and chaos, they have adopted Jane Jacob's view that "intricate minglings of different uses are not a form of chaos. On the contrary they represent a complex and highly developed form of order." \cite{jacobsDeathLifeGreat2016}. Systems engineering has moved away from prescription, embraced multi-stakeholder analysis and negotiation, and uses chaos and complexity theory, uses probabalistic modeling.

There were certain factors that contributed to the lack of success during this era \cite{lightWarfareWelfareDefense2005}:

\begin{itemize} \setlength{\itemsep}{0pt} \setlength{\parskip}{0pt} 
	\item{These new techniques, heavily dependent on quantification, suffered from a lack of relevant data, particularly on wellbeing.}
	\item{There was a lack of prior goal setting by decision makers. Technical experts were driven by a desire to use the tools available to them rather than to actually address the specific needs of the community.}
	\item{Technocratic systems and defense analysts displaced professional urban planners and others with detailed subject knowledge, rather than seeking to cooperate and learn from them. Some urban planners went so far as to complain that "their work had been hijacked by refugees from aerospace."}
\end{itemize}

There are many possible causes of this arrogance. James Scott argues that "the willful disdain for local comptence... was not, I believe, simply a case of prejudice (of the edcuated, urban, and Westernized elite twoards the peasantry) or of aesthetic commitments implicity in high modernism. Rather, official attitudes were also a matter of institutional privilege. To the degree that [local] pratices were presumed reasonable until proven otherwise, to the degree that specialists might learn as from [locals] as vice versa, and to the degree that specialists had to negotiate with [locals] as political equals, would the basic premise behind the officials' institutional power and status be undermined." \cite{scottSeeingStateHow2020}


This era of systems engineering applications on urban planning was also marked by significant secrecy, as was the cultural norm of these military-connected organizations \cite{lightWarfareWelfareDefense2005}.

"From cybernetics to computer simulation to satellite reconnaissance, techniques and technologies originally developed for military users in the 1940s, 1950s, and 1960s thus became the focus of efforts to better plan and manage U.S. cities in the 1960s and 1970s" Ultimately however, they "rarely served as sources of solutions" and instead resulted in the "creation and maintenance of an urban 'power elite' whose influence on the ways Americans conceptualize cities and their problems has persisted to the present day. \cite{lightWarfareWelfareDefense2005}

Figure \ref{fig:friedman_timeline}; Systems engineering is positioned on the far left of the figure, indicating that the field (or at least the authors listed associated with it) "look to the confirmation and reproduction of existing relationships of power in society. Expressing predominantly technical concerns, they proclaim a carefully nurtured stance of political nuetrality. In reality, they address their work to those who are in power and see their primary mission as serving the state." \cite{mazza2017}

"The engineer's sense of certainty (and his ignorance of history) informed some of the most prominent of later planning theorists... all of whom were enthralled by the idea of "designing society" \cite{mazza2017}

"There was a moment in time when aeronautic and space engineers throught that, having reached the moon, they could now turn their energies to solving the problem of growing violence in cities along with other urban "crises." \cite{mazza2017}

\clearpage
\begin{sidewaysfigure}[t]
	\centering
	\includegraphics[scale=0.65]{Figures/chap2/friedman_timeline.png}
	\caption[Timeline of intellectual influences on American planning theory]{Timeline of intellectual influences on American planning theory. From \cite{mazza2017}}
	\label{fig:friedman_timeline}
\end{sidewaysfigure}
\clearpage

Also use diagram/framework from \cite{marcuseThreeHistoricCurrents2016}:

\begin{itemize}
    \setlength{\itemsep}{0pt}%
    \setlength{\parskip}{0pt}%
	\item{\textbf{Technicist:} Planning focused on maximizing efficient of the system being planned. The planner is a technical professional with specialized knowledge. The "technicist is inherently conservative: it is to serve an economic and social and policital order in which its role is to make that order function smoothly."}
	\item{\textbf{Social Reform:} Planning grounded in social ideas and values while viewing the necessary changes as possible within the exisiting framework of social, political, and economic order. The exact values of concern have varied over the years, with environmental sustainability coming to the rise more recently.}
	\item{\textbf{Social Justice:} Planning by grass-roots groups and social movements, putting values ahead of efficiency, and willing to work outside of existing systems to accomplish its goals.}
\end{itemize}


\begin{table}[h]
\caption[Axes of currents of city planning]{Axes of currents of city planning. Based on  \cite{marcuseThreeHistoricCurrents2016}}
\label{table:currents}
\begin{center}
\begin{tabular}{| C{0.1cm} C{0.1cm} | C{2cm} | C{3cm} | C{3cm} |} \cline{4-5}

\multicolumn{1}{c}{} & \multicolumn{1}{c}{} & \multicolumn{1}{c|}{} & \multicolumn{2}{c|}{\textit{Stance towards existing relations of power}}  \\ \cline{4-5}

\multicolumn{1}{c}{} & \multicolumn{1}{c}{} & \multicolumn{1}{c|}{} & \textbf{Critical} & \textbf{Deferential} \\ \hline

\multirow{4}{*}{\parbox{4cm}{\rotatebox[origin=c]{90}{\textit{Primary}}}}
& \multirow{4}{*}{\parbox{4cm}{\rotatebox[origin=c]{90}{\textit{concern}}}} 
& \multirow{2}{*}{\textbf{Social}} 
& \multirow{2}{*}{Social Justice} 
& \multirow{2}{*}{Social Reform} \\ 

& & & & \\ \cline{3-5}

& & \multirow{2}{*}{\textbf{Efficiency}} & & \multirow{2}{*}{Technicist} \\

& & & & \\  \hline

\end{tabular}
\end{center}
\end{table}



"The systems engineers bring some expertise and substantial pretensions to the problems of the city. Their prinicpal system expertise seems to be relative to complex organizations that are mission oriented. There is in any case a good deal of difference between the mission of reaching the moon, and the mission of surival and welfare for soceity and the city. The systems engineer can in general deal best with subsystems and specific tasks, and he therefore suboptimizes. This is a charitable description." \cite{robinsonDecisionmakingUrbanPlanning1972}

Goldman suggests that many of these issues can be avoided by paring complex systems theory with collaborative planning theory. \cite{goodspeedScenarioPlanningCities2020}

Goodman notes that "systems dynamics models are not widely used in urban practice" \cite{goodspeedScenarioPlanningCities2020}

Nostikasari notes how assumptions in a transportation planning model can perpetuate inequality \cite{nostikasariRepresentationsEverydayTravel2015}

"Are decision makers less likely to incorporate subjective values and humanistic concerns in policy decisions becaues of the seeming remoteness of hte world in which these policies have an impact?" \cite{vereginComputerInnovationAdoption1994}

This historic preference of the impersonal and "objective" versus the personal and "subjective" is by no means unique to systems engineering. It can also be found in economics, jurisprudence, education theory, political science, and even moral philosophy \cite{banuriModernatizationItsDiscontents1990}. The development of stakeholder analysis has helped to bridge the gap between these two and thus rectify this traditional deficiency. 

Participatory models were not unheard of even in earlier decades however. For example, the Dayton Neighborhod Achievement Model dates back to the early 1970s \cite{lightWarfareWelfareDefense2005}.

The use of stakeholder analysis in contemporary systems engineering is, in a way, a step away from the world view that sees "human beings as unknowable black boxes and machines as transparent," a viewpoint that "surrenders any attempt at empathy and forecloses the possibility of ethical development" and is a tacit "admission that we have abandoned a social commitment to try and understand each other." \cite{eubanksAutomatingInequalityHow2018}

Sufficient deficiencies certainly still remain within the systems engineering community. "One cannot know about the history of media stereotyping or the nuances of structural oppression in any formal, scholarly way through the traditional engineering curriculum of the large research universities from which technology companies hire across the United States. Ethics course are rare." \cite{nobleAlgorithmsOppressionHow2018}

It should be noted that, while \ac{ppgis} developed as a means to avoid some of these concerns, it also arose out of intra-academia discussion and arguments, not as a result of vociferous demand from communities themselves \cite{weinerParticipatoryGeographicInformation2007}. 

\subsection{Technocratic Planning and International Development} \label{sec:technocracy}

By "technocracy" we mean the basic idea that "the human problem of urban design has a unique solution, which an expert can discover and execute. Deciding such technical matters by politics and bargaining would lead to the wrong solution." \cite{scottSeeingStateHow2020}


Some of the sources here cited in this chapter refer specifically to maps, while others to more general quantified data for states. For our purposes, however, the differences between these two are not particularly important. Most "state simplications... have the character of maps. That is, they are designed to summarize precisely thoes aspects of a complex world that are of immediate interest to the mapmaker and to ignore the rest. To complain that a map lacks nuance and detail makes no sense unless it omits information necessary to its function." \cite{scottSeeingStateHow2020}


Respond to critiques of central planning / technocratic efforts by Easterly \cite{easterly2015}



James Scott argued that "legibility [is] a central problem in statecraft." Through the act of making society and nature itself legible, "officials took exceptionally, complex, illegible, and local social practices... and created a standard grid whereby it could be centrally recorded and monitored. The organization of hte natural world was no exception."  \cite{scottSeeingStateHow2020}


Scott argued that four elements were necessary to precipitate the most tragic of social engineering disasters \cite{scottSeeingStateHow2020}:

\begin{itemize} \setlength{\itemsep}{0pt} \setlength{\parskip}{0pt} 
	\item{The "administrative ordering of nature and society." This includes items like cadastral maps, surnames, census records, and a standardized legal system. As Theodore Porter put it, "Society must be remade before it can be the object of quantification." \cite{porter1992objectivity}}
	\item{A "high-modernist ideology," which Scott defines as a "strong," "muscle-bound" "self-confidence about scientific and technical progress, the expansion of production, the growing satisfaction of uman needs, the master of nature... and the rational design of social order commensurate with the scientific understanding of natural laws."}
	\item{An authoritarian state that is both "willing and able" to wield power to inact the high-modernist ideology.}
	\item{A vulnerable civil society that "lacks the capacity to resist" the plans of that authoritarian state.}
\end{itemize}

In essence what is "truly dangerous to us and our environment... is the \textit{combination} of the universalist pretensions of epistemic knowledge and authoritarian social engineering." \cite{scottSeeingStateHow2020}

With regard to the second element a key aspect is that, as Scott notes, high-modernist ideology is not scientific practice. It is a "faith that borrowed from the legitmacy of sciency and technology." In fact, it was more an aesthetic prediliction than anything scientific. Furthermore, the underlying ideas were in fact quite sympathetic. "Doctors and public-health engineers who did possess new knowledge that could save millions of lives were often thwarted by popular prejudices and entrenched political interests"  \cite{scottSeeingStateHow2020}. The dangers were when an authoritarian state adopted the aesthetic veilings of such ideas to justify actions, in the way that Social Darwinism used evolutionary theory to justify horrid actions. In this way "the classism and racism of elites are mathwashed, neutralized by technological mystification and data-based hocus-pocus." \cite{eubanksAutomatingInequalityHow2018} This ideology could also be considered a "dangerous form of magical thinking [that] often accompanies new technological developments, a curious assurance that a revolution in our tools inevitably wipes the slate of the past clean." \cite{eubanksAutomatingInequalityHow2018}


In the USSR, "a set of informal practices lying outside of the formal command economy - and often outside Soviet law as well - [arose] to circumvent some of the colossal waste and inefficiencies built into the system. Collectivized agriculture, in other words, never quite operated according to the hierarchical grid of production plans and procurements." \cite{scottSeeingStateHow2020} The technocratic leaders were often aware of this but so committed to their ideology that they had no alternative but to maintain a sort of pretense, which anthropologist Alexi Yurchak called 'hypernormalization' \cite{yurchakEverythingWasForever2005}, that served to compound problems until the Soviet Union eventually collapsed. Such a phenomena is particularly visible in strictly planned capital cities that have, "as the inevitable accompaniment of its official structures, given rise to another, far more 'disorderly' and complex city \textit{that makes the official city work} - that is virtually a condition of its existence." \cite{scottSeeingStateHow2020}



Even the successful development projects often came at a high cost and raised the question of "successful for whom?" After all "Haussmann's Paris was, \textit{for those who are not expelled}, a far healthier city." (emphasis mine) \cite{scottSeeingStateHow2020}


Facts generated by states are inherently simplifications. Specifically, they tend to simplify in five specific ways \cite{scottSeeingStateHow2020}:

\begin{itemize} \setlength{\itemsep}{0pt} \setlength{\parskip}{0pt} 
	\item{They are interested and utilitarian, aimed at a particular end.}
	\item{They are nearly always written, as opposed to visual or verbal.}
	\item{They are typically static and thus, perpetually out-of-date to at least some extent. "The cadastral map is very much like a still photograph of the current in a river."}
	\item{They are typically aggregate facts, not individual ones.}
	\item{They are standardized, so as to enable comparison and longitudinal analysis.}
\end{itemize}


Planning has come a long way from focusing on single page map and a timescale of 20-30 years (Section 2 Introduction of \cite{robinsonDecisionmakingUrbanPlanning1972})

By providing tools for more participation, we are not necessarily doing anything radical. "Democracies rarely end up expropriating and redistributing capital" \cite{fainsteinSpatialJusticePlanning2016}. "Participation is not power; its reform is not radical" \cite{marcuseThreeHistoricCurrents2016}. Some argue that neoliberalism in factor prefers to use participation as a means of underming resistance, rather than violence, though this has the risk of providing a structure for coalition building and radicalization \cite{miraftabInsurgentPlanningSituating2016}. In fact, increased community involvement can result in more restrictive, unambitious goals that are not in the interests of certain minorities (Section 1, Chapter 2 of \cite{robinsonDecisionmakingUrbanPlanning1972}).


Sachs argues that prescreptive economics should be modeled on clinical medicine and should not seek to attribute all negative outcomes to the same cause nor to prescribe the same solution to all problems, but instead to "make a differential diagnosis for the economic case at hand." He lays out several different conditions of poverty, for example, and proposes different solutions to each. Foreign aid is effective at treating the "poverty trap" condition (wherein "the country is too poor to make the basic investments it needs ot escape from extreme material deprivation and get on the ladder of economic growth"), but less so for other conditions.  \cite{sachsAgeSustainableDevelopment2015} We work with local collaborators to provide them with tools to develop local solutions....


Discuss and critique of informality as a concept \cite{royUrbanInformalityProduction2016}

De Soto argues that the poor already have assets, just needs to be formalized. \cite{sotoMysteryCapitalWhy2003} though others argue that this is just results in a cycle of appeasment / welfare \cite{hollandForbearanceRedistributionPolitics2017}

Power can be wielded in planning discussions in 
three primary ways: "by promoting formal decisions, setting the agenda, and influencing the broader ideogical context of the debate." (\cite{foresterPlanningFacePower2001} as paraphrased by \cite{goodspeedScenarioPlanningCities2020})

Goodspeed, however, argues that scenario-based planning can help address injustices such as racism in urban development and cites several examples of this having occured \cite{goodspeedScenarioPlanningCities2020}.

"Many studies involve ranking places on one or more criteria, and allocating policy benefits accordingly. At its crudest this applied geography merely provides a list of winner ans losers with no understanding of why the differences occur." \cite{taylorGeographicInformationSystems1994}

Part of this was sheer arrogance. Even one of the proponents recognized that "rational, hierarchical, closed-door decision strategies" had negative consequences and that "more democratic process might produces worse rsults, but it would respond to the increasing sense of alienation among the nation's urban population." \cite{lightWarfareWelfareDefense2005}

Virginia Eubanks proposees two gut check questions \cite{eubanksAutomatingInequalityHow2018}:

\begin{enumerate} \setlength{\itemsep}{0pt} \setlength{\parskip}{0pt} 
	\item{Does the tool increase the self-determinationa and agency of the poor?}
	\item{Would the tool be tolerated if it was targeted at non-poor people?}
\end{enumerate}

Jonathan Furner, meanwhile proposes three strategies (\cite{furner2007dewey} as paraphrased by \cite{nobleAlgorithmsOppressionHow2018}):

\begin{enumerate} \setlength{\itemsep}{0pt} \setlength{\parskip}{0pt} 
	\item{Admission on the part of designers that bias in classification schemes exists, and indeed is an inevitable result of the ways in which they are currently structured.}
	\item{Recognition that adherence to a policy of neutrality will contribute little to eradiction of htat bias and indeed can only extend its life.}
	\item{Construction, collection, and analysis of narrative expressions of the feelings, thoughts, and beliefs of classification-scheme users who identify with particularly racially-defined populations.}
\end{enumerate}



\begin{figure}[h]
	\centering
	\includegraphics[scale=0.4]{Figures/chap2/gis_equity.png}
	\caption[Development of GIS development and associated outcomes]{Development of GIS development and associated outcomes. From \cite{tullochTheoreticalModelMultipurpose1999} as reprinted in \cite{tullochInstitutionalGeographicInformation2007}}
	\label{fig:gis_equity}
\end{figure}



\subsection{Sustainable Development and the SDGs}

"Substantive goals, the acievement of which are hard to measure, may be supplanted by thin, notional statistics - the number of villages formed, the number of acres plowed." \cite{scottSeeingStateHow2020}


"The pessimistic thought is that sustainable development has been stripped of its transformative power and reduced to its lowest common demoninator. After all, if both the World Bank and radical ecologists now believe in sustainability, the concept can have no teeth: it is so malleable as to mean many things to manny people without requiring commitment to any specific policies." \cite{campbellGreenCitiesGrowing2016}

"Yet there is also an optimistic interpretation of the broad empbrace given sustainability: the idea has become hegemonic, an accepted meta-narrative, a given. It has shifted from being a variable to being the parameter of hte debate, almost certain to be integrated into any future scenario of development." \cite{campbellGreenCitiesGrowing2016}

"To... critics, the prospect of integrating economic, environmental and equity interests will seem forced and artificial. States will require communities to prepare "Sustinable Development Master Plans," which will prove to be glib wish lists of goals and suspiciously vague implementation steps. To achieve consensus for the plan, language wil lbe reduced to the lowest common demoninator, and the pleasing plans will gather dust." (written in 1996, pre MDGs and SDGs) \cite{campbellGreenCitiesGrowing2016}

"The danger of translation is that one language will dominate the debate and thus define the terms of the solution. It is essential to exert equal effort to translate in each direction, to prevent one linguistic culture from dominating the other... Another lesson from the neocolonial inguistic experience is that it is crucial for each social group to express itself in its own language before any translation. The challenge for planners is to write the best translations among the languages of the economic, the ecological, and the social views, and to avoid a quasi-colonial dominance by the economic \textit{ingua franca}, by creating equal two-way translations... Translation can thus be a powerful planner's skill, and interdisciplinary planning education already provides some multiculturalism. Moreover, the idea of sustainability lends itself nicely to the meeting on common ground of competing value systems." \cite{campbellGreenCitiesGrowing2016}

Williamson and Connolly point out that "the term sustainability exists and operates within a number of governmental hegemonic discourses, i.e. the term itself is continually produced within legislative power structures," and argue that we should not "centre mapmaking praxis on generic or legislative definitions of sustainability, but rather encourages dialogue that supports the re-formation of self, community, and place." Importantly, they do not "seek to overturn generic understandings of sustainability, but rather seek a more complex understanding and proliferation of the term via local 'grounded' definitions. \cite{williamsonTheirworkDevelopmentSustainable2011} 


"That view is much too pessimistic... Investing in faireness may also be investing in efficiency, and... attention to sustainability can be more fair and more efficient at the same time." \cite{sachsAgeSustainableDevelopment2015}

"\ac{mdg} goal setting has energized civil society and helped to orient governments that otherwise might have neglected the chalenges of extreme poverty... the \acp{mdg} have been important in encouraging governments, experts, and civil society to undertake the 'differential diagnoses' necessary to overcome remaining obstacles." \cite{sachsAgeSustainableDevelopment2015}

Goals accomplish several things \cite{sachsAgeSustainableDevelopment2015}:

\begin{itemize}
    \setlength{\itemsep}{0pt}%
    \setlength{\parskip}{0pt}%
	\item{Global goals are critical for social mobilization and coordinated orientation.}
	\item{Global goals provide global peer pressure for adoption, monitoring, and action.}
	\item{Global goals mobilizing epistemic communities (experts, researchers, etc. These in turn can help map pathways to acheiving the goals, making them seem more managable and less remote.)}
	\item{Global goals mobilize stakeholder networks and thereby leverage capital and other resources.}
\end{itemize}

Even Sachs, a booster of global goals like the \acp{mdg} and \acp{sdg}, admitted that the impact of the \acp{mdg} was uneven, with public health receiving the most attention, while sanitation and education were largely sidelined. \cite{sachsAgeSustainableDevelopment2015}

\begin{figure}[h]
	\centering
	\includegraphics[scale=0.35]{Figures/chap2/sustainable_triangle.png}
	\caption[The triangle of conflicting goals of sustainable development]{The triangle of conflicting goals of sustainable development. Adapted from \cite{campbellGreenCitiesGrowing2016}}
	\label{fig:sustainable_triangle}
\end{figure}

Respond to critiques of \acp{mdg}/\acp{sdg} \cite{alstonShipsPassingNight2005, reddyGlobalDevelopmentGoals2008}



%\section{Section sample 1}


%\begin{enumerate}
%  \item Item 1.
%  \item Item 2.
%  \item Item 3.
%\end{enumerate}
%
%
%
%\begin{eqnarray*}
%a_i & = & a_j + a_k \\
%a_i & = & 2a_j + a_k \\
%a_i & = & 4a_j + a_k \\
%a_i & = & 8a_j + a_k \\
%a_i & = & a_j - a_k \\
%a_i & = & a_j \ll m \mbox{shift}
%\end{eqnarray*}
%instead of the multiplication.  For example, you could use:
%\begin{eqnarray*}
%r & = & 4s + s\\
%r & = & r + r
%\end{eqnarray*}
%Or by xx:
%\begin{eqnarray*}
%t & = & 2s + s \\
%r & = & 2t + s \\
%r & = & 8r + t
%\end{eqnarray*}

\chapter{\hlc[red]{EVDT Framework}} \label{ch:evdt}

Computational models have been closely linked to the pursuit of sustainable development and with its definition, stemming from the World3 system dynamics model underlying the Club of Rome's \textit{The Limits to Growth} report in 1972 \cite{meadowsLimitsGrowth1972}.

"Sustainable Development involves not just one but four complex interacting systems. It deals with a global economy...; it focuses on social interactions...; it analyzes the changes in complex Earth systems...; and it studies the problems of governance". \cite{sachsAgeSustainableDevelopment2015} [Compare this to the EVDT framework]

We are far from the first to argue that such integration is necessary, nor to recognize that it is easier said than done \cite{shahumyanIntegrationLandUse2017}.

There have been many land use and transportation models. The open source UrbanSim, for example, combines land use, transportation, and certain environmental factors in a dynamic, area-based simulation system that, similar to \ac{evdt}, is a collection multiple models \cite{waddellUrbanSimModelingUrban2002}.

The agent-based \ac{ilute} model simulated the urban spatial form, demographics, travel behavior, and environmental impacts for the Toronto area \cite{millerHistoricalValidationIntegrated2011}.

Existing \ac{pss} have often been criticized for being lacking with regard to "visioning, storytelling sketching, and developing strategies," as well as being "too generic, too complex, inflexible, incompatible..., oriented towards technology rather than problems" \cite{brommelstroetPlanningSupportSystems2010}" This leads to what some have called the "implementation gap" of \acp{pss} \cite{BottlenecksBlockingWidespread}.

For the past couple of decades, there has been a recognition that \acp{dss} and \acp{pss} must include more than purely spatial analysis components \cite{geertmanPlanningSupportSystems2004}.

The closest attempt to what we are proposing is probably that of Shahumyan and Moeckel, though their approach focused on linking together existing models in a loose manner using ArcGIS Model Builder, to avoid having to gain access to proprietary source code. While their example focused on combinging transportation, land use, mobile emissions, building emissions, and land cover, with only limited feedbacks, their approach could be extended to capture the full feedback loops proposed by \ac{evdt}. Their example is also proof that the kind of loose integration of library of models that \ac{evdt} envisions is possible \cite{shahumyanIntegrationLandUse2017}. 

Lauf et al. combined cellular automata with systems dynamics to capture both spatial dynamics and macroscale demand-supply dynamics in order to simulate residential development \cite{laufUncoveringLanduseDynamics2012}

Pert et al. combined environmental and decision-making in a participatory model to improve conservation outcomes \cite{pertParticipatoryDevelopmentNew2013}. 

Miller argues that, despite the historical dificulties that integrated urban models have had, there is reason to be optimistic about the state of the art moving forward, particularly for integrating transportation and land-use models in particular \cite{millerIntegratedUrbanModeling2018}. <---important to discuss at more length

Is not itself a means of planning and implementing projects. It is not a full life-cycle tool such as \ac{ppbs} \cite{hatryCriteriaEvaluationPlanning1972}

Clifton et al. breaks down the various ways of modeling the urban form into five categories (though they do not assert that these are comprehensive or mutually exclusive), as seen in Table \ref{table:urban_form} \cite{cliftonQuantitativeAnalysisUrban2008}. While \ac{evdt} does not focus specifically on urban form, it is interested in these types of models, with the case studies presented in this work focusing on landscape ecology and community design in particular. One downside of examinations of urban form is that they tend to focus on areas and residences, while various forms of social exclusion are better measured by focusing on individuals instead \cite{scottRoleUrbanForm2008}.

\begin{table}[h]
%\begin{center}
\caption[Five categories of urban form models]{Five categories of urban form models. Adapted from \cite{cliftonQuantitativeAnalysisUrban2008}}
\label{table:urban_form}
\makebox[\linewidth]{
\begin{tabular}{ L{3cm} L{3cm}  L{3cm} L{3cm} L{3cm} L{3cm}} \hline
Perspective & Principal concern & Disciplinary Orientation & Scale & Nature of Data & Common Metrics  \\ \hline

Landscape ecology & Environmental protection & Natural scientists & Regional & Land cover & Land cover change; Contagion \\ 

Economic structure & Economic efficiency & Economists & Metropolitan & Employment and population & Density gradient; Land value  \\

Transportation planning & Accessibility & Transportation planners & Submetropolitan & Employment, population and transportation network & Expected travel time; capacity  \\

Community design & Social welfare & Land-use planners & Neighborhood & Local \ac{gis} data & Proximity to needs; Zoning; Accessibility \\

Urban design & Aesthetics and walkability & Urban designers & Block face & Images, surveys, and audits & Lot size; Accessibility \\ \hline

\end{tabular}
}
%\end{center}
\end{table}


The motivation for combining so many variables from different disciplines stems from both push and pull factors. The push factors are the simple increase in availability of data, as has already been described, along with the increase in the interoperability of the variables (which the work described in this thesis is trying to help contribute to). The primary pull factor is our increased understanding of - and appreciation for - the complex relationships between these domains, relationships that were previously ignored in analyses \cite{gaheganMultivariateGeovisualization2007}. 


Position \ac{evdt} using the different dimensions of models proposed in \cite{harrisQuantitativeModelsUrban1972}:

\begin{enumerate}[itemsep=0pt,parsep=0pt]
	\item{descriptive vs. analytic}
	\item{holistic vs. partial}
	\item{macro vs. micro}
	\item{static vs. dynamic}
	\item{deterministic vs. probabilistic}
	\item{simultaneous vs. sequential (directly calculate the output or go through intermediate phases)}
\end{enumerate}

Goodchild defines six different \ac{gis} data field model types and states that "no current \ac{gis} gives its users full access to all six":

\begin{enumerate}
    \setlength{\itemsep}{0pt}%
    \setlength{\parskip}{0pt}%
	\item{Sample randomly located points (e.g. weather stations, \ac{lidar} data)}
	\item{Sample randomly from a grid of regularly space points (e.g. many data validation studies}
	\item{Divide the area into a grid in which each rectangular cell records the average, total, or dominant value; i.e. raster data (e.g. satellite imagery)}
	\item{Divide the area into homogenous regions and record the average, total, or dominant value in each area (e.g. census data, soil maps)}
	\item{Record the locations of lines of fixed values (e.g. contour or isopleth maps}
	\item{Divide the area into irregular shaped triangles and assume the field varies linearly within each (e.g. some \acp{dem})}
\end{enumerate}

In the mid 90s, \ac{gis} had several limitations \cite{goodchildGeographicInformationSystems1994}:

\begin{itemize}
    \setlength{\itemsep}{0pt}%
    \setlength{\parskip}{0pt}%
	\item{Two-dimensional, with some excursions into three}
	\item{Static, with some limited support for time dependence}
	\item{Limited capabilities for representing forms of interaction between objects}
	\item{A diverse and confusing set of data models}
	\item{Dominated by the map metaphor}
\end{itemize}

To some extent, many of these issues, such as the lack of three dimensional systems, persisted well past the 90s \cite{goodchildTwentyYearsProgress2010}.

Yamu et al. argue that urban modeling should treat the urban form as a \ac{cas} and use fractal metrics to develop scenarios for planning purposes \cite{yamuAssumingItAll2016}.

In order for cost-benefit analysis to maximize economic welfare, the following conditions must be met \cite{krutillaWelfareAspectsBenefitCost1961}:

\begin{enumerate}[itemsep=0pt,parsep=0pt]
	\item{Opportunity costs are borne by beneficiaries in  such wise as to retain the initial income distribution}
	\item{The initial income distribution is in some sense "best}
	\item{The marginal social rates of transformation between any two commodities are everywhere equal to their corresponding rates of substitution except for the area(s) justifying the intervention in question}
\end{enumerate}

More details modeling, as well as breaking down specific costs and benefits (as opposed to converting them to monetary terms and summing them) and attributing them to specific goals, can circumvent these constraints, though at the cost of increased complexity \cite{hillGoalsAchievementMatrixEvaluating1972}.

The Law of requisite variety from the field of cybernetics says that the variety (the number of elements or states) of the control device must be at least equal to that of the disturbances \cite{ashbyRequisiteVarietyIts1991}. Any development plan is going to fall far short of the variety expressed by human society and the natural environment. Planning efforts must then make reliance on the natural homeostasis behavior of such systems and of more flexible, ad hoc measures not specified in the plan in order to make up the difference in variety. \cite{mcloughlinSystemGuidanceControl1972}

\section{\hlc[red]{The Framework}} 

\subsection{\hlc[red]{System Architecture Framework}} \label{sec:saf}

\subsection{\hlc[red]{Collaborative Development}} 

\subsection{\hlc[red]{EVDT Questions \& Models}} 

\subsection{\hlc[red]{Interactive Decision Support System}} 

\subsection{\hlc[red]{Re-use \& Community Development}} 



\section{\hlc[red]{Intended Applications \& User Types}}

While \ac{evdt} does not include any concrete spatial scale requirements, it is often the most straightforward to apply to it at a a relatively local scale, like much of the early history of \ac{gis} applications \cite{tullochInstitutionalGeographicInformation2007}. Most of the applications to date have been at the area of a metropolitan area or that of a small province.


Tends to be at intersections of rural and urban areas. Urban areas often depend on an area significantly larger than the built-up area for basic resources and ecosystem services, particularly for water, bulky materials, and waste disposal. I will not attempt to strictly define rural and urban here, as the "distinctions are often arbitrary" \cite{tacoliRuralurbanInteractionsGuide1998}. Instead this work will rely upon local definitions of urban, rural, and peri-urban, similarly to the \ac{un} \cite{sachsAgeSustainableDevelopment2015}. 



Commonly has to do with \acp{cpr}. Talk about the three common ways of managing \acp{cpr}: Central management, privatization, self-management. Bring in Table \ref{table:cpr_design}  showing design principles of long-enduring self-management institutions. Refer to successful aspects of the water basin in California (incremental and sequential process to reduce the costs of local institutional supply, shared information at each step, intermediate benefits from initial investments were realized prior to larger investments, transformed structure of incentives within which fuure strategic decisions can be made) (pg. 137. \cite{ostromGoverningCommonsEvolution2015}

\begin{table}[h]
\caption[Design principles illustrated by long-lasting CPR institutions]{Design principles illustrated by long-lasting \ac{cpr} institutions. Adapted from \cite{ostromGoverningCommonsEvolution2015}}
\label{table:cpr_design}
\begin{center}
\begin{tabular}{ L{0.5cm} L{8cm}} \hline
1. & Clearly defined boundaries \\
2. & Congruence between appropriation and provision rules and local conditions \\
3. & Collective-choice arrangements \\
4. & Monitoring \\
5. & Graduated sanctions \\
6. & Conflict-resolution mechanisms \\
7. & Minimal recognition of rights to organize \\
\multicolumn{2}{l}{\textit{For CPRs that are parts of larger systems:}} \\
8. & Nested enterprises \\ \hline
\end{tabular}
\end{center}
\end{table}

We also 

Harris et al. have pointed out that reliance on mapping products to designate certain geographic areas for conservation has several negative consequences, including: 

\begin{itemize}
    \setlength{\itemsep}{0pt}%
    \setlength{\parskip}{0pt}%
	\item{solidifying a notion that humans and non-human others are, and should be, separate.}
	\item{privileging those voices and perspectives that have access and expertise related to Western cartographic approaches and GIScience in conservation debates.}
	\item{favoring those spaces, ecosystems, and natures that may be "more mappable" for protection over other areas.}
	\item{cementing an overly-limited territorial approach to conservation, in ways that potentially sideline non-territorial approaches.}
	\item{consolidating an overly-fixed and static approach to sonservation, rather than enabling approaches that may be more seasonal, fluid, or appropriate for shifting and evolving ecological conditions and needs.}
\end{itemize}

\section{\hlc[red]{Novelty}}

\section{\hlc[red]{Development \& Evaluation}}


\section{\hlc[red]{Mapping and Visualization}}


"A single map is but one of an indefiniteliy large number of maps that might be produced for the same situation or from the same data." \cite{monmonierHowLieMaps1996}

Data maps have a long history. Tufte dates them to the seventeenth century and cites Edmond Halley's 1686 chart of trade winds as "one of the first data maps" \cite{tufteVisualDisplayQuantitative2001} though arguably Scheiner's 1626 sunspot visualization qualifies as a data map \cite{friendlyBriefHistoryData2008}, as perhaps do Polynesian knot maps, which long predates either [CITE]. Graphing data over time, meanwhile dates by to the 14th century \cite{friendlyBriefHistoryData2008}.

Choropleths are one of the more common types of non-imagery geospatial data that \ac{evdt} uses. These are maps that express "quantity in area" (i.e. some statistic tied to a particular geographic area with color, texture, or shading). It should be noted that choropleths have a few well-known limitations, including the ecological fallacy and the modifiable areal unit problem \cite{cramptonRethinkingMapsIdentity2011, sawickiNeighborhoodIndicatorsReview1996}. It is for these reasons that \ac{evdt} does not rely entirely on choropleths and why we strive to store data with the finest geospatial resolution available.

Historically, \ac{gis} implementations have often struggled to handle temporal data \cite{harrisLocationalModelsGeographic1993}.

Historically social indicators tended to be defined for city, province, or national areas, the \acp{mdg} and \acp{sdg} being the preeminent examples of the latter. Advances in \ac{gis}, however did enable the creation of more neighborhood level indicators starting in the late 1990s \cite{sawickiNeighborhoodIndicatorsReview1996}. 

Sawicki and Flynn argue that one must specify the goals before specifying what indicators to use. From their list of possible aims, the following are the most relevant to \ac{evdt} \cite{sawickiNeighborhoodIndicatorsReview1996}:

\begin{itemize}[itemsep=0pt,parsep=0pt]
	\item{Developing dynamic models of neighborhood change}
	\item{Evaluating the likely impact of existing and/or poposed policies on neighborhoods and/or their residents.}
	\item{Measuring inequality over space and time both within and between regions.}
\end{itemize}



Initial versions of \ac{evdt} and Vida featured quite large graphics. Tufte argues that graphics in general should be significantle shrunk and that "many data graphics can be reduced in area to half their current published size with virtually not loss in legibility and information." \cite{tufteVisualDisplayQuantitative2001} Inaccordance with this Shrink Principle, these graphics were greatly reduced in later versions.

As with most \ac{gis} software \cite{heikkilaGISDeadLong1998}, early verions of \ac{evdt} were structured as entirely object-oriented, and later versions remained primarily object-oriented. This has many advantages but also comes at certain costs, the most important of which include (a) difficulty in recording continuous spatial variables and (b) a requirement to pre-identify the different classes (objects) to sort phenomena and relationships into \cite{goodchildModelingEarth2011}. 

It is recognized that this desktop version comes with numerous downsides. \textit{theirwork}, an early collaborative, open source \ac{gis} platform, specifically "decided at an early stage to make the software Web-based to allow for a process of rapid development and iteration and allow a maximum number of potential participants." \cite{williamsonTheirworkDevelopmentSustainable2011} It should be noted, however, that \textit{theirwork} was a UK-based project (an area with high internet connectivity penetration) and started in the mid 2000's, a period with significantly diversity of internet browsing methods, which simplified the task of ensuring accessibility. Nonetheless, it is impossible to deny the collaboration and software sustainability benefits of an online platform, particularly in an age when many of the early concerns with the internet (low speeds, lack of knowledge about how to use it, etc.) \cite{shifterInteractiveMultimediaPlanning1995} have been largly alleviated.

the meeting arrangment that EVDT supports, Table \ref{table:meeting_arrangements}

\begin{table}[h]
\caption[Different types of meeting arrangements]{Different types of meeting arrangements. Adapted from \cite{jankowskiGISGroupDecision2001}}
\label{table:meeting_arrangements}
\begin{center}
\begin{tabular}{ L{3cm} L{5.5cm}  L{5.5cm}}  \hline
 & \textit{Same time} &\textit{Different time}  \\ \cline{2-3}
\textbf{\textit{Same place}} & \textbf{Conventional Meeting} \qquad \textit{Advantage:} 
\vspace{-5mm}
\begin{itemize}
    \setlength{\itemsep}{0pt}%
    \setlength{\parskip}{0pt}%
	\item{face-to-face expressions}
	\item{immediate response}
\end{itemize} &
\textbf{Storyboard meeting} \qquad \textit{Advantage:} 
\vspace{-5mm}
\begin{itemize}
    \setlength{\itemsep}{0pt}%
    \setlength{\parskip}{0pt}%
	\item{scheduling is easy}
	\item{respond anytime}
	\item{leave-behind note}
\end{itemize} 
\\
& \textit{Disadvantage:} 
\vspace{-5mm}
\begin{itemize}
    \setlength{\itemsep}{0pt}%
    \setlength{\parskip}{0pt}%
	\item{scheduling is difficult}
\end{itemize} &
\textit{Disadvantage:} 
\vspace{-5mm}
\begin{itemize}
    \setlength{\itemsep}{0pt}%
    \setlength{\parskip}{0pt}%
	\item{meeting takes longer}
	\item{difficult to maintain in the long run}
\end{itemize} 
\\ \hline

\textbf{\textit{Different place}} & \textbf{Conference call meeting} \qquad \textit{Advantage:} 
\vspace{-5mm}
\begin{itemize}
    \setlength{\itemsep}{0pt}%
    \setlength{\parskip}{0pt}%
	\item{no need to travel}
	\item{immediate response}
\end{itemize} &
\textbf{Distributed meeting} \qquad \textit{Advantage:} 
\vspace{-5mm}
\begin{itemize}
    \setlength{\itemsep}{0pt}%
    \setlength{\parskip}{0pt}%
	\item{scheduling is convenient}
	\item{no need to travel}
	\item{submit response anytime}
\end{itemize} 
\\
& \textit{Disadvantage:} 
\vspace{-5mm}
\begin{itemize}
    \setlength{\itemsep}{0pt}%
    \setlength{\parskip}{0pt}%
	\item{limited personal perspective from participants}
	\item{meeting protocols are difficult to interpret}
	\item{difficult to maintain meeting dynamics}
\end{itemize} &
\textit{Disadvantage:} 
\vspace{-5mm}
\begin{itemize}
    \setlength{\itemsep}{0pt}%
    \setlength{\parskip}{0pt}%
	\item{meeting takes longer}
	\item{meeting dynamics are different from normal meeting ("netiquette" instead of face-to-face etiquette)}
\end{itemize} 
\\ \hline
\end{tabular}
\end{center}
\end{table}

Does \ac{evdt} aimed at \textit{backward visualization}, which is aimed at assisting experts and professoinals, or \textit{forward visualization}, which is aimed at a less informed audience \cite{battyVisualizingCityCommunication2000}.

While three dimensional data exists for both the urban environment \cite{battyVisualizingCityCommunication2000} and from remote sensing (reference lidar), \ac{evdt} focuses primarily on two dimensional symbolic visualizations.

\ac{evdt} takes a somewhat Harleian approach to visualization, in which "\textit{presentation} is de-emphasized in favor of \textit{exploration} of data" \cite{cramptonMapsSocialConstructions2001}.
\chapter{\hlc[red]{Rio de Janeiro Mangroves}} \label{ch:mangroves}



\section{\hlc[yellow]{Study Area \& Context}}

Guaratiba is a relatively rural district of Rio de Janeiro situated in the southwestern corner of the municipality. It is home to a mix of land uses, including decorative plant farming, multiple fishing communities, a military base and training center, a state-run biological reserve, some informal settlements, and a growing ecotourism industry. The biological reserve exists to protect the largest remaining mangrove forest within the municipality. These mangroves are vulnerable due to landward urbanization, including a recently opened urban transit line, and rising sea levels \cite{goldbergEcoMapDecisionsupportTool2018} They provide a variety of ecosystem services, including serving as a mechanism for highly efficient carbon sequestration, supporting a small-scale industry of fishing and crab catching, preventing coastal erosion, and attracting the aforementioned local ecotourism industry \cite{schwenkResearchEnvironmentalSocioeconomical2008}. Government policies to conserve the mangroves can use integrated modeling tools to consider both the benefits of protecting the forests as well as the economic needs of low-income communities. This, coupled with the Rio de Janeiro municipal government's pre-existing interest in generating useful datasets and making them available online through the Data.Rio platform \cite{matheusOpenGovernmentData2014}, made the Guaratiba mangroves a particularly suitable case study for the \ac{evdt} Modeling Framework.

Our primary Local Context Experts and points of contact are at \ac{ipp}, which is the municipal data agency, and ESPAÇO, a research group at the \ac{ufrj} who study various coastal ecosystems in Brazil and elsewhere \cite{cruzClassificacaoOrientadaObjetos2007, seabraMapeamentoDinamicaCobertura2013} and who are also familiar with examining socioeconomic impacts of environmental phenomena \cite{schwenkResearchEnvironmentalSocioeconomical2008}. The latter can also be considered to be Technical Area Experts. Other Local Context Experts include a member of a local fisher association and government officials at the municipal urban development agency and the municipal environmental agency Additional Technical Area Experts include two ecosystem services economists (one from the University of West Virginia and one from \ac{rff}) and arguably the committee members for this thesis. The primary intended users for this case study are government officials at the \ac{ipp} who have a fair amount of experience with mapping. Future projects in this area would ideally expand that userbase to non-government individuals. 

This project began in 2018 and since that time Jack Reid made two multi-week field visits to Rio de Janeiro and Guaratiba in particular.

\subsection{\hlc[red]{Stakeholders}}

a

\section{\hlc[red]{Systems Architecture Framework}}

a

\subsection{\hlc[red]{Interviews}}

a

\begin{figure}[h]
	\centering
	\includegraphics[scale=0.3]{Figures/chap4/Stakeholder_Map_v2.jpg}
	\caption[Stakeholder Map for the Mangrove Forests of Rio de Janeiro]{Stakeholder Map for the Mangrove Forests of Rio de Janeiro}
	\label{fig:rio_stakemap}
\end{figure}

\subsection{\hlc[red]{Needs, Outcomes, and Objectives}}

a

\subsection{\hlc[red]{System Architecture}}


\section{\hlc[red]{EVDT Application}}

\subsection{\hlc[red]{Environment:}}

\subsection{\hlc[red]{Vulnerability}}

\subsection{\hlc[red]{Decision-making}}

\subsection{\hlc[red]{Technology}}

\section{\hlc[red]{Decision Support System}}

\section{\hlc[red]{Evaluation}}

\section{\hlc[red]{Discussion}}






\chapter{Vida Decision Support System} \label{ch:vida}



\section{\hlc[yellow]{Study Area \& Context}}

As the coronavirus pandemic swept the globe, many of the local points of contact working with Space Enabled on \ac{evdt} and other projects had sudden changes in priorities. Several of them raised the possibility of adapting and expanding the \ac{evdt} Modeling Framework to approach coronavirus-related decision-making and impact analysis. This seemed relevant because, as others have noted, coronavirus impacts and response can be characterized as a complex system warranting a multi-domain, model-based approach \cite{deweckHandlingCOVID192020}. The second case study will focus on this project, which ultimately became known as Vida and came to involve six metropolitan areas across Angola, Brazil, Chile, Indonesia, Mexico, and the United States. In each of these areas, Vida was (and is) developed in collaboration with local government officials, university researchers, and general community members. 

Whereas the first case study focuses on simulating the changes in mangrove forest over decades, the focus of Vida is examining hourly to weekly air and water quality data alongside daily coronavirus epidemiological data and weekly quarantine policies. Government officials need actionable data to both address the ongoing public health crisis and to cope with the resultant socioeconomic and environmental consequences. Community members need to understand why their government is making the decisions that it is and understand the risks associated with their own actions. \color{OliveGreen} The Technical Area Experts on this project include researchers from Harvard Medical School. Meanwhile the Local Area Experts (many of whom are technical experts in their own right) include a mix of government officials and academic researchers, most of whome work in the public health and/or in \ac{gis}. The intended Users are those same individuals as well as the various public health agencies / task forces that they are affiliated with.  In general, \color{black} the concept is for our partner organizations to use \ac{evdt} to develop analyses and presentations that can inform pandemic response. The exact process by which this takes place varies from location to location.

\subsection{\hlc[red]{Stakeholders}}

\section{\hlc[red]{Systems Architecture Framework}}

\subsection{\hlc[red]{Interviews}}

\subsection{\hlc[red]{Needs, Outcomes, and Objectives}}

\subsection{\hlc[red]{System Architecture}}

\begin{figure}[h]
	\centering
	\includegraphics[scale=0.3]{Figures/architecture.png}
	\caption{The high-level functional systems architecture of the Vida \ac{dss}.}
	\label{fig:architecture}
\end{figure}


\section{\hlc[red]{Vida Variant of EVDT}}


\subsection{\hlc[red]{Environment:}}

\subsection{\hlc[red]{Public Health:}}


\subsection{\hlc[red]{Vulnerability}}

\subsection{\hlc[red]{Decision-making}}

\subsection{\hlc[red]{Technology}}

\section{\hlc[red]{Decision Support System}}

\section{\hlc[red]{Evaluation}}

\section{\hlc[red]{Discussion}}


%\chapter{\hlc[red]{Discussion}} \label{ch:discussion}



\section{\hlc[red]{Lessons Learned}}


\section{\hlc[red]{The Future of EVDT}}



\chapter{Summary, Discussion, \& Conclusion} \label{ch:conclusion}

\section{Chapter Purpose \& Structure}

This chapter has two purposes. The first is to review the previous chapters and summarize the contributions presented in each. This is the focus of Section \ref{sec:summary}. The second purpose is to supply an answer to Research Question 3:

\blockquote{What steps are necessary to establish \acf{evdt} as a continually development framework, a community of practice, and a growing code repository?}

This is the focus of Section \ref{sec:future-inquiry}, which provides Research Deliverables 3a and 3b:

	\begin{enumerate}[label=\emph{\alph*},itemsep=0pt,parsep=0pt]
		\item{An assessment of lessons learned from these \acf{dss} development processes} 
		\item{An outline of potential future \ac{evdt} refinement and extension, such as using \ac{evdt} to inform the development of future \acf{eo} systems that are better designed for particular application contexts}
	\end{enumerate}
	
The chapter, and the thesis as whole, then ends with a brief concluding statement.

\section{Summary \& Contributions} \label{sec:summary}

The following subsections briefly summarize the work performed in this dissertation. They thus do not contain anything not already presented in their respective chapters. More novel discussion is provided in Section \ref{sec:future-inquiry}.

\subsection{Theory \& Critical Analysis}

This thesis, due to its multidisciplinary aspect, did not have a singular literature survey, but instead had several distinct sections summarizing and analyzing existing literature. Chapter \ref{ch:theory} was the first and largest of these. It discussed six distinct fields that are foundational to the \ac{evdt} Framework:

\begin{itemize}[itemsep=0pt,parsep=0pt]
	\item{\textbf{Section \ref{sec:sustainable_development} - Sustainable Development}. This term, which has grown enormously in popularity over the past couple of decades, encapsulates the desires to balance the sometimes aligning, sometimes conflicting needs for environmental protection, economic development, and advancement of human society (including health and wellness). The idea of such linkages between different domains, creating a complex system, underlies the \ac{evdt} framework.}
	\item{\textbf{Section \ref{sec:remote} - Remote Observation:} This field provides a rich source of data, both historical and present for applications around the world. Its capabilities have rapidly expanded over the past few decades and promise to continue doing so in the coming years.}
	\item{\textbf{Section \ref{sec:se} - Systems Engineering:} This discipline, which largely arose out of the need for a kind of meta-engineering for large aerospace projects, provides many of the tools and methods used by the \ac{evdt} Framework, including most notably the \acf{saf}.}
	\item{\textbf{Section \ref{sec:gis} - \Acf{gis}:} This field provides the geospatial backbone to the analyses conducted and \acp{dss} created over the course of an \ac{evdt} project.} 
	\item{\textbf{Section \ref{sec:collaborative} - Collaborative Planning:} This field is where the \ac{evdt} Framework draws its means of engaging stakeholders to a greater extent that is common in systems engineering applications.}
	\item{\textbf{Section \ref{sec:dss} - \Acf{dss}:} The goal of the \ac{evdt} Framework is to support sustainable development decision-making, so it is only natural that we draw upon the existing literature on methods of supporting decisions.}
\end{itemize}

Once summarized, these fields were then reconsidered through a critical lens in Section \ref{sec:critiques}. This was done to better understand the causes of various historical failings of each of these fields and how such pitfalls might be avoided when constructing and implementing a new framework. 

The first of these critiques focused on whether it is even possible to advance sustainable development, and its human wellness component in particular, through the use of technical means and expertise. This question was considered both in the general case and as it applied to \ac{gis}, planning and international development, and systems engineering in particular. This analysis demonstrated that any given technology is not ethically neutral, but comes laden with intent, incentives, and constraints on its use; requiring us to be conscientious when designing new technologies. Specific steps to accomplish this include considering and involving a wide set of stakeholders during the design of a new system; focusing on supporting members of a community to make their own decisions, rather than prescribing a single "optimal" decision from the outside; and maintain a certain level of epistemic humility about the capabilities of the technologies that we use.

The second critique considered whether sustainable development, as it is commonly defined and used, is actually an effective means of advancing environmental protection, social advancement, and economic development; or if its just a means of greenwashing the last of these. In particular I considered the utility of the \acfp{sdg}. In this analysis, I noted that there are several, legitimate critiques of the \acp{sdg}, including the perhaps over-focus on specific, quantifiable metrics, a largely top-down national perspective, and a lack of cross-goal connections. That said, I also noted several positive aspects of the \acp{sdg}, including their improvement upon the earlier \acfp{mdg} and their ability to facilitate communication about sustainable development. Ultimately, I conclude that the best approach for our new framework is to focus on more local, targeted, bottoms-up applications that have significant stakeholder involvement.

The third and final critique considered whether there is any evidence that \acp{dss}, and scenario planning in particular, have any demonstrable value. After discussing various cases of muddled evidence or even counterproductive results, I lay out the case for a participative, model-based form decision support that has firmer evidence and is reasonable grounds on which to base a doctoral dissertation.

Through the summaries and critiques, Chapter \ref{ch:theory} provided Research Deliverable 1a, ``A critical analysis of systems engineering, \ac{gis}, and the other fields relied upon in this work.", and one step towards answering Research Question 1:

\blockquote{What aspects of systems architecture (and systems engineering in general) can be used to support sustainability in complex \ac{sets}? In particular, how can they be adapted using techniques from collaborative planning theory and other critical approaches to avoid the technocratic excesses of the past?}

\subsection{The EVDT Framework}

Chapter \ref{ch:evdt} took these various foundational fields and lessons presented and used them to construct a methodology for constructing a \ac{dss} for sustainable development applications. It surveyed a variety of processes and projects that seek to accomplish similar goals. I found that there remained a need for a generalized framework that combined multidisciplinary model integration, stakeholder participation and collaboration on a local scale, and the significant use of remote observation data for sustainable development.

The remainder of the chapter was dedicated to walking through the new proposed process, entitled the \ac{evdt} Framework, shown in Figure \ref{fig:evdt_framework}, and constituted by five basic elements:

\begin{enumerate}[label=\emph{\Alph*)},itemsep=0pt,parsep=0pt]
	\item{The use of the \acf{saf} to understand the system context, identify stakeholder needs, and design the \ac{dss}.}
	\item{A conceptualization of the sustainable development application in terms of its Environment, Human Vulnerability and Societal Impact, Human Behavior and Decision-Making, and Technology Design components.}
	\item{An interactive \ac{dss}.}
	\item{A consideration towards modularity and re-use in future applications.}
	\item{Collaborative development of the \ac{dss} that continues beyond initial stakeholder engagement.}
\end{enumerate}

Through this guide, Chapter \ref{ch:evdt} provided Research Deliverable 1b: "A proposed framework for applying systems engineering for sustainable development in an participatory and social-justice-oriented manner" and, together with the previous chapter, supplied an answer to Research Question 1. This chapter did not, however, provide any concrete demonstration or evaluation of the \ac{evdt} Framework. That was a task left to the following two chapters.

\subsection{Rio de Janeiro Development \& Mangroves}

The first of the two case studies, presented in Chapter \ref{ch:mangroves}, was on the development of a \ac{dss} for the setting of urban planning zones and environmentally protected areas in the western portion of the city of Rio de Janeiro, particularly around the Guaratiba area. It focused on the relationship between coastal mangroves and the local inhabitants, including the ecosystem services provided by the mangroves.

To accomplish this, I conducted a stakeholder analysis process via the \ac{saf}, coming to understanding the history of the Guaratiba area, its possible futures, and the relationships between many of its stakeholders. This process informed the design of a desktop-based \ac{dss} and the targets of a series of analyses that included tracking mangrove extent and health, estimating mangrove biomass and carbon sequestration, the value of various mangrove ecosystem services, and the impact of zoning and conservation policy on all of the above. The \ac{dss} compiled these in an interactive manner for use by stakeholders.

In doing so, I was able to supply the first instances of Research Deliverables 2a and 2b:

\begin{enumerate}[label=\emph{\alph*},itemsep=0pt,parsep=0pt]
	\item{System architecture analyses of each of the case studies.} 
	\item{Development of an \ac{evdt}-based \ac{dss} for each of the case studies.} 
\end{enumerate}

The onset of the \ac{covid} pandemic and the abrupt termination of the project resulted in an uncompleted Research Deliverable 2c, "An interview-based assessment of the development process and usefulness of each \ac{dss}." Chapter \ref{ch:theory} was able to present both what feedback was received and the originally intended plan for collecting such feedback. Section \ref{sec:future-inquiry} below will also contain some additional evaluation of this case study.

\subsection{Vida DSS for COVID-19 Response}

The second of the two case studies, presented in Chapter \ref{ch:vida}, was on the development of two \acp{dss} for supporting \ac{covid} response policymaking in several different regions around the world, including Luanda, Rio de Janeiro, Regíon Metropolitana de Santiago, Java \& Sulawesi, Querétaro de Arteaga, and Boston. 

To pursue an \ac{evdt} project for so many study areas, a large team of Local Context Experts and Technical Area Experts (with some individuals serving in both capacities) was assembled. With their participation, we were able to conduct a wide variety of analyses, including on the impacts of \ac{covid} on air quality, nightlights, and human mobility. We also constructed two \acp{dss}, one desktop-based and one online, the former of which was capable of simulating \ac{covid} cases and other phenomena using a \acf{sir} model. 

This case study, in addition to the within-study-area stakeholder collaboration originally conceived by the \ac{evdt} Framework, had significant cross-study-area collaboration, providing additional benefits beyond the analyses and \acp{dss} that were the focus of this project. 

These actions were allowed to provide additional instances of Research Deliverables 2a, 2b, and 2c. As a result, we can now offer a tentative affirmative to Research Question 2:

\blockquote{Does the \ac{evdt} Framework effectively support decision-making in in complex \ac{sets}?}

\section{Lessons \& Opportunities for Future Inquiry} \label{sec:future-inquiry}

This section is dedicated to noting certain limitations, lessons, and opportunities for future inquiry that have arisen out of the work presented in this thesis. In particular, it discusses each of these as they pertain to the \ac{evdt} Framework and the general methodology. For notes on limitations and opportunities connected to specific elements of analysis (such as air quality monitoring), refer to the Discussion section of the respective case study chapter.

I will first focus on each of the two case studies, recounting some of the lessons and opportunities noted in their respective chapters. Then I will take these and conduct a more wholistic evaluation of this thesis and the \ac{evdt} Framework in general.

Collectively, this section will supply both Research Deliverables 3a and 3b:

\begin{enumerate}[label=\emph{\alph*},itemsep=0pt,parsep=0pt]
	\item{An assessment of lessons learned from these \ac{dss} development processes.} 
	\item{An outline of potential future \ac{evdt} refinement and extension, such as using \ac{evdt} to inform the development of future \ac{eo} systems that are better designed for particular application contexts.} 
\end{enumerate}



%Chapter \ref{ch:conclusion} will review these lessons, including discussing how the case studies would have been performed differently in retrospect.

%The second portion of the chapter, Section \ref{sec:critiques}, turns towards to critiques of the literature and the concept of this thesis. It is an attempt to recognize and preemptively address potential pitfalls of the approach taken in this thesis. These are primarily fundamental or ethical concerns, as opposed to mere questions of implementation, the latter of which are largely held for Chapter \ref{ch:conclusion}.

\subsection{Lessons Regarding the Foundational Fields} \label{sec:lessons-foundational}

As summarized above, Chapter \ref{ch:theory} laid out the six fields that provide the foundation of this thesis: sustainable development, systems engineering, remote observation, \ac{gis}, collaborative planning, and decision support. It provided both histories of these fields that focused on their merits and a series of critical analyses that examined their weaknesses. This resulted in a set of general lessons that were used in the creation of the \ac{evdt} Framework. Now that the framework has been developed and applied, however, we can now draw some additional, more specific notes about these fields.   

\textbf{The use of remote sensing and \ac{gis} for sustainable development is rapidly expanding and we need to ensure that this is done in a stakeholder-informed way.} At the time of writing, the US tech industry is undergoing mass layoffs and contractions. Despite this, remote observation companies are rapidly expanding, including both satellite operators and more downstream players. Hyperspectral constellations are coming online \cite{planetlabspbcPlanetAnnouncesNew2022, rainbowPixxelRaises252022}; multiple greenhouse gas monitoring platforms are being launched \cite{clarkExclusiveSatelliteImages2022, brownSecurityCameraPlanet2023}, existing players are extending their offerings \cite{jewettPlanetDebutsPlanetary2022}; and machine learning is rapidly advancing, with significant implications for the capabilities of geospatial data \cite{joyceCanYouUse2023}. Jobs with "remote sensing" or "\ac{gis}" in the title are in abundance on Climatebase, the largest board for climate-related jobs. \ac{nasa} has recently launched a slate of activities under the title "Equity and Environmental Justice" including community workshops and grant solicitations \cite{bollesEquityEnvironmentalJustice2021}. Floodbase has leveraged satellite data and modeling to provide sufficiently reliable historical and monitoring data to enable new populations to receive flood insurance \cite{tellmanRegionalIndexInsurance2022}. This dissertation and other projects by Space Enabled provide further demonstrations of the power of geospatial data to support sustainable development around the world \cite{lombardoDevelopmentDecisionSupport2021, lombardoEnvironmentVulnerabilityDecisionTechnologyFrameworkDecision2022, ovienmhadaInclusiveDesignEarth2021, ovienmhadaEnvironmentVulnerabilityDecisionTechnologyModelingFramework2021}. 

It remains to be seen, however, the extent that local communities will be involved in the design of these \ac{eo} systems, the analysis of their data, or the use of that analysis. Many of the private corporation actors are focused on carbon measurement (either storage or emissions). While certainly critically important for the global climate, carbon is often not the highest priority ecosystem service to local communities around the world, as I learned in the Rio de Janeiro case study. This is not to say that these interests are misaligned. Fishers in Guaratiba want to preserve the mangroves (which prevents emissions), they just want to do it for other reasons than carbon. 

\ac{nasa}'s activities, meanwhile, have been almost entirely aimed at an academic audience. Their first community workshop "featured representatives from social science research organizations" \cite{bollesEquityEnvironmentalJustice2021}. A ROSES grant application is a major endeavor for an experienced academic research group, much less a community organization unfamiliar with the \ac{nasa} grant system. Floodbase primarily partners with large insurance companies like Munich Re and governmental organizations like \ac{fema} and \ac{wfp} \cite{floodbaseFloodMonitoringSudan2022, floodbaseAnnouncingOurWork2023, rostonClimateStartupAims2023}.

Part of this is because, at a very fundamental level, precious few organizations or communities in the world have an "\ac{eo} problem." They are not clamoring for new satellites or access to data. Instead they have a flooding problem, a rice farming problem, a heat stroke problem, or respiratory health problem. All of these problems are things that remote sensing data can help with (though certainly not solve on its own). It should be the role of the technical expert to help bridge that gap and bring \ac{eo} data to bear on the problems defined by the community. Instead, all too often, those experts want to drag the community into remote sensing, or worse yet, remote sensing with a particular instrument (such as the \ac{nasa} environmental justice workshop focused solely on \ac{nisar}). This is precisely why the \ac{evdt} Framework centers the stakeholder needs and then works backwards from there to identify what instrument (or, more likely, what collection of instruments) and analysis techniques can help address their problem.

\textbf{There exists a very real tension between the utility of technical fields (remote observation, systems engineering, the components of \ac{evdt}, etc.) and their accessibility to a wide set of stakeholders (which is demanded by the \ac{evdt} Framework).} Section \ref{sec:se_critique} had a discussion of this in the general case, that concluded with a quote from Campbell: "the idea of sustainability lends itself nicely to the meeting on common ground of competing value systems." After working on multiple sustainable development projects (including the ones in this thesis), I still believe that this is true but also believe that there is no guarantee that this will remain so in the coming years or in all situations. Experts and those with institutional power have a way of gradually taking over a space and coming to their own consensus about language that can be inaccessible to the broader populace or to those with expertise outside of that circle.

Part of this is addressed by careful attention to presentation and terminology, as discussed in Section \ref{sec:language}. This is only a partial solution however, as there will always be the temptation by technical experts to re-center the situation on their personal expertise or perspective. I can illustrate this with an example.

Much of the earth data that \ac{nasa} produces is organized into the \ac{eosdis}. This system has defined data processing levels based on how much error correcting, processing, and analysis has gone into the creation of a given data product \\cite{earthsciencedatasystemsDataProcessingLevels2016}. These levels are described in Table \ref{tab:data-processing-levels}. Levels 0 through 3, have several, more specific sublevels not shown in this table. Meanwhile, virtually all applications-oriented data, including virtually all of the analysis products presented in this thesis are lumped together in the final level, Level 4, which has no sublevels. I recently attended a \ac{nasa} organized workshop on the potential environmental justice applications of an upcoming satellite, \ac{nisar}. When one of the attendees accessed about the accessibility of \ac{nisar} data to community activities in the Deep South, one of the \ac{nasa} organizers proudly explained that they would be provided global Level 2 data, seemingly failing to understand the significant technical knowledge and expertise required to access Level 2 data and transform it into actionable and broadly intelligible Level 4 data. Getting technical experts and community activists into the same room as one another, have them agree on mutual respect for one another, and they still may have difficulty understanding one another.

\begin{table}[!htb]
\caption[NASA EOSDIS Data Processing Levels]{Descriptions of the data processing levels used by \ac{nasa} \ac{eosdis} data products. Based on \cite{earthsciencedatasystemsDataProcessingLevels2016} with additional notes and examples provided by me.}
\label{tab:data-processing-levels}
\begin{center}
%\scriptsize
\small
\begin{tabular}{ C{3cm}   L{8cm} } \hline
 
\textbf{Data Level} & \textbf{Description} \\ \hline

Level 0 & Instrument data that is largely unprocessed, other than removing communications artifacts. This data is rarely seen or used outside of \ac{nasa}. \\

Level 1 & Instrument data that is time-referenced and annotated with relevant metadata (such as georeferencing parameters and radiometric coefficients). Some Level 1 data is converted into physical units such as radiance. Top-of-atmosphere radiance is in this category. This data is sometimes used by scientists, particularly atmospheric scientists or those studying error correction. \\ 

Level 2 & Derived geophysical variables at the same resolution and location as the L1 source data. Surface reflectance is in this category. L2 represents some of the most commonly used data by scientists and it is commonly available in platforms such as \ac{gee} and Amazon Web Services. Usually the end of where science agencies provide data. \\

Level 3 & Geophysical variables mapped on uniform grids (both in time and space). Monthly or annual mosaics are in this category. Sometimes provided by more operations-oriented agencies such as \ac{usgs}. \\

Level 4 & Model or analysis outputs. A broad category that includes essentially all of the remote observation results in this thesis. Occasionally provided by operations-oriented agencies, such as weather models and forecasts provided by \ac{noaa}. \\ \hline

\end{tabular}
\end{center}
\end{table}

And I certainly ran into some of these barriers to communication during the case studies. Some stakeholders, even those with some local degree of authority such as municipal government officials, would seek to defer to me in conversations rather than engage and collaborate, as I was a graduate student from a prestigious US university. 

All of this is to say that just because the barriers to communication in sustainable development are lower than in many technical fields, it by no means that they are nonexistent. This also underscores the importance of having multiple Local Context Experts who can help to provide more equitable introductions and translations with stakeholders in the community.


\subsection{Lessons from Rio de Janeiro}

\subsubsection{Summary of Lessons from Chapter \ref{ch:mangroves}} 

The following are short summaries of the lessons identified in Chapter \ref{ch:mangroves}. For more detailed versions of these, see Section \ref{sec:rio-lessons-learned}.

\textbf{The need for the two separate iterations of the \ac{saf}.} Earlier versions of the \ac{evdt} Framework where not as strictly linear as Figure \ref{fig:evdt_framework}. Both the \ac{saf} and the \ac{evdt} components were viewed as spanning the entire project and not being particularly distinct, with only one iteration of the \ac{saf} explicitly called for. This led to ambiguity about whether the \ac{evdt} practitioner should be focusing the \ac{saf} process on the existing {sets} that the community is operating in or on the \ac{dss} that was to be developed. 

Through this case study and other projects, we realized that it is important to first use the \ac{saf} to define and provide information on the existing \ac{sets} that stakeholders live in, then frame that \ac{sets} using the \ac{evdt} components, and only then embark on the design of a \ac{dss} using the \ac{saf}. This helps to avoid putting the cart before the horse and forcing a particular pre-conceived solution architecture upon the stakeholders. 

\textbf{The importance of Local Context Experts.} While stakeholder participation and collaboration was key to the \ac{evdt} framework in even its earliest versions, the importance of Local Context Experts as discussed in Section \ref{sec:intended} was not fully appreciated. 

In particular the importance of having firm connections to multiple stakeholders can be critical to properly involving as many stakeholders as possible. Sometimes, however, such alignment with certain stakeholders is desirable even if it runs the risk of alienating others, but such a decision should be consciously and explicitly made. Ovienmhada et al., for example, working on environmental justice in carceral landscapes of the US, intentionally positioned the project as a social justice endeavor that was aligned with prison abolition activists \cite{ovienmhadaEnvironmentVulnerabilityDecisionTechnologyModelingFramework2021}. 

\textbf{The importance of appropriately scoping an \ac{evdt} project.} Involving as many stakeholders as possible and taking the multidisciplinary approach called for by the \ac{evdt} Framework can quickly cause a project to balloon out of the realm of feasibility. Care must be taken to ensure that the project remains within the resources of the direct participants, mediators, and developers.

\textbf{The power of reusing assets and building upon experience.} These enable the more rapid pursuit of more ambitious \ac{evdt} projects and can lower the barrier-of-entry for future projects.

\textbf{The importance of the \ac{saf} for understanding how to balance competing concerns of different stakeholders.} And using that understanding to build a project that supports the community's decision-making rather than trying to present a singular solution to any particular stakeholder.

\subsubsection{Additional Lessons and Future Work}

\textbf{The power of perception and perspectives.} When it comes to sociotechnical systems, Maier and Rechtin proposed what they called "a painful design heuristic," namely that "it's not the facts, it's the perceptions that count" \cite{maierArtSystemsArchitecting2009}. This was all too apparent in the Rio de Janeiro case study. \ac{smu} treated the value of the mangroves as close to zero because they did not have data on hand as to their value. \ac{smac} desired a policy of no human activity within the boundaries of the \ac{rbag}, even though there was abundant on-the-ground evidence that local communities could engage in sustainable use of the area. 

Notably, these examples run somewhat counter to the point that Maier and Rechtin were seeking to make. The examples they provide are all about the public or lay people being unreasonable and they go one to suggest that "perhaps the only antidote" is to be so transparent during the design process that "the skeptical elite is convinced, and through them the general public." Here though, we find the technically trained elite ignoring what was readily apparent to the local community. 

This does underscore the potential dangers of modern systems engineering methodologies like the \ac{saf}, as originally raised in Section \ref{sec:se_critique}. While they call for increased awareness of the relationships between stakeholders and provide tools for mediating multi-stakeholder negotiations, they do not commit the engineer to any particular perspective. In absence of this, the engineer is likely to default to alignment with the more technically competent stakeholders, those able to speak the same language (both literally and figuratively) as the engineer. I definitely experienced this temptation myself and count myself lucky that the of my more technically experienced Local Context Experts, \ac{ipp} was more interested in capacity building that any particular application and ESPAÇO was ideologically aligned with the local community.

The \ac{evdt} Framework must have a certain politic, as was (hopefully) made clear by the critiques in Chapter \ref{ch:theory}, to avoid being co-opted into alignment with elites, as has happened too many times already. Space Enabled has long drawn on Ibram Kendi's \textit{Stamped from the Beginning: The Definitive History of Racist Ideas in America} for its work (including non-\ac{evdt} work). This book, however does not have many specific lessons for \ac{evdt}-esque work (as opposed to more general lessons about anti-racism). The \ac{evdt} Framework could be well served by explicitly drawing on more targeted books in this vein, such as \textit{Data Feminism}, \textit{Design Justice}, and \textit{Data Action} (all referenced in various places earlier in this thesis), as well as the recently released \textit{Decolonizing Design} \cite{tunstallDecolonizingDesignCultural2023}. 

\textbf{A question of positionality and sustainability.} The \ac{evdt} Framework as presented in this thesis is somewhat vague as to the position of the framework mediator / guide in any particular project. Section \ref{sec:perspective} emphasized the importance of considering one's own position in the network of stakeholders. Section \ref{sec:intended} discussed the various ways that individuals could be involved with \ac{evdt} for a particular project or across multiple projects. This case study (Rio de Janeiro) even demonstrated this, including Space Enabled as a stakeholder in Table \ref{tab:rio-dss-needs}, for instance. Each of these, however, stopped short of offering a normative statement on the ideal position of the "\ac{evdt} expert."

In this Rio de Janeiro case study, I was literally positioned a continent away for the vast majority of the time, visiting only twice, unable to speak the local language with any real fluency, and cut off from many of the stakeholders by the onset of the \ac{covid} pandemic. On the other hand, I was able to remain involved with this community for five years (much longer than many academic projects), regularly corresponding and meeting (virtually) with several stakeholders. 

The \ac{evdt} Framework has to straddle a certain line. It explicitly calls for collaboration and capacity building, which involves bringing in various kinds of external Technical Experts. It also emphasizes the agency of the local community and its ability to make its own decisions. If I was based in Rio de Janeiro (or even on the same continent), I likely would have been able to engage much more deeply with a wider variety of stakeholders. Is that what future projects should strive for?

Beyond literal geographic position, must the \ac{evdt} Framework always be applied by an academic? For the foreseeable future this is likely to be the case, but that is a different question than whether it should be the end goal.

I do not have any immediate answer to these questions. It would seem to defeat the point of the framework to restrict future \ac{evdt} projects to those geographically or culturally proximate to that of the implementor, but perhaps the framework should more explicitly address the preferred relationship here. 


\subsection{Lessons from COVID-19}

\subsubsection{Summary of Lessons from Chapter \ref{ch:vida}} 

The following are short summaries of the lessons identified in Chapter \ref{ch:vida}. For more detailed versions of these, see Section \ref{sec:vida-lessons}.

\textbf{The benefits of multi-study-area or multi-project communication and collaboration.} The Vida International Network has facilitated international collaboration, allowing participants to share innovations and insights from their \ac{covid} response efforts. It has also encouraged intra-country collaboration by providing a motivation for outreach between government officials, academic researchers, and community leaders in order to fill data gaps and answer pressing questions. This process has also raised awareness of the utility of space-based \ac{eo} data, potentially preparing participants for future pandemic and non-pandemic applications. 

The success of the multilateral Network meetings prompted us to continue such meetings for a more general EVDT community audience, as well as to consider other potential means of engagement such as webinars or online resources. This likely would not have occurred if we had continued to pursue more individual, siloed \ac{evdt} projects as was the norm.

\textbf{The importance of clearly scoped problem and use.} Overall, the Vida project included many unsuccessful experiments (such as monitoring vehicle traffic) and disconnected components (such as the separate in-situ and \ac{eo}-based air quality analyses). It resulted in some interesting results that did not find their way into the \ac{dss} proper or to supporting decision-making in other ways. This was partially due to the rushed stakeholder analysis, which was in turn due to the urgency of the situation and the complexity of multiple study areas. As such, it was potentially unavoidable in this case study but nonetheless represents a cautionary tale for future \ac{evdt} analyses.

\textbf{The power of reusing assets and building upon experience, particularly when a rapid response effort is required.} Reusing code from the Chapter \ref{ch:mangroves} \ac{dss} and workflows for analyzing \ac{eo} data were invaluable, as was a higher level of familiarity with the \ac{evdt} Framework that I and other participants had as a result of previous projects.

\subsubsection{Additional Lessons and Future Work}

\textbf{The pressure to expand \ac{evdt} beyond graduate student projects and potentially beyond academia.} Building upon the above noted lessons from this case study, it must be acknowledged that the adoption and use of the \ac{evdt} Framework is somewhat limited currently, existing primarily as graduate student research projects. Given the time required to become knowledgeable in the framework and its associated methods, to gain some familiarity with the community centered in a project, and to then carry out a project, a graduate student can only be reasonably expected to carry out one to two such projects prior to completing their program. This makes it difficult to sustain much of the inter-project connections and learning that we identified earlier as being so helpful, much less conduct the kind of capacity building work required to enable others to pursue \ac{evdt} projects independently of Space Enabled.

One option here is to bring on a postdoctorate researcher or research scientist to perform this kind of interstitial work while supporting various \ac{evdt} projects conducted by individual graduate students. This may work in the short term, but even it may run into difficulties in the long run. The US federal research funding agencies do not provide significant funds for the kind of work that the \ac{evdt} Framework calls for. I was fortunate enough to be funded by the undirected Media Lab Consortium, but this arrangement is unusual in US higher education. As mentioned earlier, \ac{nasa} is expanding its environmental justice activities, some of which could overlap with \ac{evdt}, but even this is likely to be limited in terms of both funding and acceptable topics for the foreseeable future. We can expect to occasionally obtain funding from the \ac{nsf} or \ac{nasa} for particular innovative analysis techniques, but not for \ac{evdt} projects in their entirety. 

Another possibility would be to form a non-profit \ac{ngo} or some other organization outside of academia to support and directly pursue \ac{evdt} projects. I confess that I am not particularly familiar with the availability and the nature of funding for such organizations, so cannot fully assess the feasibility of this option. 

Still another possibility, not mutually exclusive with either of the previous two, is to partner with \ac{nasa} DEVELOP. This is a relatively recent program where small teams with a problem in mind can gain access to technical experts at \ac{nasa} and partner organizations to develop a solution. Projects can cover any of \ac{nasa} Applied Sciences' thematic areas, namely: Agriculture, Climate, Disasters, Ecological Conservation, Energy, Health \& Air Quality, Urban Development, Water Resources, and Wildland Fires \cite{nasaappliedsciencesDEVELOP}. While these topic areas certainly would cover the vast majority of \ac{evdt} projects, DEVELOP is not a perfect home for \ac{evdt}. The projects are limited to 10 weeks. They involve a competitive application process and relocating to one of the DEVELOP "nodes." They also require a clear analysis problem going in, which reduces the ability to explore the needs and desires of various stakeholders over the course of the project.

Still other options likely exist and should be pursued by future \ac{evdt} practitioners. The focus on local situations and the pressures to scale will always been in tension. Organizations like \ac{nasa} and major development agencies historically have preferred programs and projects that can scale to entire nations, regions, or even globally. Yet this thesis has strongly argued for the ethical and practical importance of local projects. As a result, it may be inevitable that the \ac{evdt} Framework has difficulties scaling.	

\subsection{The Future of EVDT} \label{sec:future}

One final Research Question remains:

\blockquote{What steps are necessary to establish \ac{evdt} as a continually developing framework, a community of practice, and a growing code repository?}

This has been partially addressed by the lessons drawn from the particular case studies in the previous section. This section completes the answer to the Research Question by providing some additional steps drawn more wholistically from the entire thesis.

\textbf{Fulfill the potential of the Technology feedback loop in the \ac{evdt} formulation.} In Figure \ref{fig:model}, four equal components are shown: Environment, Vulnerability, Decision-making, and Technology. The last of these is part of a feedback loop connecting the models in a cycle. The case studies presented in this thesis did not full instantiate this and there is significant potential for future projects to do precisely that. 

For the Chapter \ref{ch:mangroves} case study, there is the potential to extend the \ac{dss} and analyses to support the selection of \ac{eo} data sources that would better support the needs of the Guaratiba area. This could mean designing a constellation of mangrove-monitoring \ac{eo} satellites or comparing aerial surveys versus tasked satellite imagery. In Chapter \ref{ch:vida} case study, there is the possibility of providing support to decision-makers regarding the \ac{covid} testing regime. Future projects could involve using tradespace exploration or other tools to support the design of a new \ac{eo} satellite (or constellation of satellites) to meet the needs of some set of stakeholders. This could build on previous work done by Siddiqi on valuing \ac{eo} missions for decision-making as part of a trade-space analysis \cite{siddiqiValuingNewEarth2019, siddiqiValuingRadiometricQuality2021} and by Grogan on multi-stakeholder, interactive assessments of satellite constellations \cite{groganMultistakeholderInteractiveSimulation2014, groganInteractiveSimulationGames2015}. It could also be as simple as designing an cheap, in-situ sensor, as was done by Ovienmhada \cite{ovienmhadaEarthObservationTechnology2020}.

One proposed conceptualization of the \ac{eo} application value chain has the eight steps shown in Figure \ref{fig:eochain}  \cite{hakimdavarTransboundaryWaterImproving2018, woodPartnershipsEnableEarth2017}. Space agencies frequently find themselves not only providing steps 1-3 (their specialty) but also some combination of 4-8. The \ac{evdt} projects presented in this thesis primarily focus on steps 4-8. A fully fleshed out Technology component and feedback loop, however, has the potential to "close the loop," connecting step 8 back to step 1. This would result in \ac{eo} systems more closely tied to the application needs of particular stakeholders and thus in better decision support. Such projects are already being pursued by other members of the Space Enabled research group, including on supporting the design of an Italian \ac{eo} system and for space sustainability decision support.

\begin{figure}[ht]
    \centering
    \includegraphics[width=0.7\textwidth]{Figures/chap7/EOChain.jpg}
    \caption{Generic Earth Observation Data Value Chain}
    \label{fig:eochain}
\end{figure}

\textbf{Conduct rapid prototyping \& co-design, without sacrificing stakeholder analysis.} In order to confirm that the proper data and dynamics are being captured, as well as to ensure the utility of the model to decision-makers and designers, the key stakeholders must be involved at all stages of the design process. Additionally, since most individuals have a difficult time providing concrete advice and criticism when discussing the abstract, rapid prototyping and mock-ups are important for stimulating feedback. One key component of this is always making sure that the user interface is available in the native language of the primary users.

Pursuing such rapid prototyping should not be allowed to cause the initial iteration of the \ac{saf} or the stakeholder analysis portion in particular to be sacrificed. The Chapter \ref{ch:vida} did precisely this, resulting in a series of analyses and \ac{dss} components that were not as well connected to each other and to stakeholder needs as should have been the case.

\textbf{Enlist appropriate experts and stakeholders.} The design of any \ac{eo} system is inherently interdisciplinary and this is also true for \ac{eo} applications. For this reason the development necessarily involves a wide range of collaborators who vary both in terms of discipline (systems engineering, urban planning, earth science, economics, etc.) and it terms of institution (academic researchers, government officials, \ac{ngo} and corporate leaders, local activists, etc.). Rather than make assumptions when confronted with an issue outside our expertise, we must our utmost to recruit or consult with a relevant expert. 

A key part of this is recognizing that these experts and stakeholders will have their own perspectives and priorities that will not always align. This is one of the primary purposes of the \ac{saf}: to understand and synthesize these perspectives. But it is also important to recognize that, particularly if the \ac{evdt} moderator is outside of the target community, one must be intentional with how one aligns (or appears to align) with the various stakeholders. Finding a relatively neutral actor to serve as a primary Local Context Expert can be quite useful for enabling productive collaboration with a wide range of stakeholders, as can independently cultivating relationships with different stakeholders. This was done in the Chapter \ref{ch:mangroves} case study, for example. Other projects however call for a more clear declaration of one's stance, even if this risks alienating certain stakeholders.

\textbf{Maintain open access and modularity for adaptation and reuse, without overly sacrificing usability.} The intent is not for the \ac{evdt} Framework is not to develop complete, black-box products, but rather to facilitate the development of widely accessible \acp{dss}. Part of this includes the collaborative aspect of the \ac{evdt} Framework, but another part is making the code itself readily available online and designing the analyses and \ac{dss} to be as reusable as possible. This helps prevent the monopolization of information that characterized some of the technocratic excesses discussed in Chapter \ref{ch:theory}. It also helps to ensure that new \ac{evdt} projects may, for example, be able to reuse previous \ac{eo} data processing techniques, while focusing on the vulnerability or decision-making components.

At the same time, a rigid fixation on maximal open access and reusability should not be allowed to significantly detract from usability. The case studies from Chapters \ref{ch:mangroves} and \ref{ch:vida} perhaps ran amiss here, focusing too heavily on custom, open source, desktop-based \acp{dss} that limited there ability to be easily accessed and used by stakeholders with a wide range of technical competence. This can be contrasted with Jaffe's online \ac{dss}, which saw wider use \cite{jaffeEnvironmentalEconomicSystems2022}. Such cloud-based and internet-hosted tools can eliminate the need to direct possession of high performance computing equipment by either the end users or the developers. This reduces cost-of-entry for potential developers of \ac{evdt} models and ensures that end users can interact with, critique, and apply the models wherever they are, provided they have internet access. In general, implementers of future \ac{evdt} projects should, during the \ac{saf} process, be particularly attentive to existing decision-making processes of different stakeholders and design the \ac{dss} to fit well within them.

\textbf{Beyond technical open access, develop \ac{evdt} terminology, figures, and materials that are more inclusive of a non-technical audience.} One of the lessons of Section \ref{sec:lessons-foundational} was that a tension exists between the utility of technical fields and their accessibility to a wide set of stakeholders. This definitely applies to \ac{evdt} Framework, which has a highly technical-sounding name (and quite a bit of technical terminology within it). As discussed in Section \ref{sec:language}, it is important to adjust language based on the stakeholder to avoid exclusion. I certainly did precisely this during the case studies presented in this work, rarely using the term \ac{evdt} or other technical terminology when speaking to non-academic stakeholders, focusing instead on much more concrete and directly relevant matters instead. This is true of the other Space Enabled \ac{evdt} practitioners as well. 

Nonetheless, all of the publications on \ac{evdt} to date and almost all of the ancillary materials generated (outside of the \acp{dss} themselves) have been aimed at a technical audience. Moving forward, it would behoove us to document how we speak about the framework with non-technical stakeholders and us this to build up a corpus of more accessible materials. Figure \ref{fig:evdt_public} is an initial step in this direction, but a great deal more could be done. Other items could include a simplified how-to manual and case study summaries. These could broaden the appeal of the \ac{evdt} Framework (or whatever more accessible name is chosen for it) in a much more scalable fashion than having active practitioners explain it in understandable language to interested parties (as is currently done).

\textbf{Seek to build relationships and collaboration across \ac{evdt} projects.} One of the key findings of the Vida case study in Chapter \ref{ch:vida} was that there is significant utility in providing venues for communication and collaboration across study areas or even across \ac{evdt} projects. Some of these benefits are to the \ac{evdt} projects themselves, identifying potential new topics of analysis or providing opportunities for the re-use of code and other assets across projects. Some of these benefits are outside of the \ac{evdt} projects themselves but still quite useful in their own right. These can include peer-to-peer learning and instruction on other sustainable development topics.
 

\section{Advancing Stakeholder-Involved Sustainable Development Decision-making}

This dissertation presented a new framework for supporting sustainable development decision-making. The goal of the research was to demonstrate both the need for and the usefulness of this framework, and to thereby advance sustainable development action, particularly on the local scale. To address these goals, I undertook three main research efforts. First, I conducted a critical analysis of a variety of fields relevant to sustainable development (including the concept of sustainable development itself) to identify potential points of synergy and how to avoid historical pitfalls. Second, I built upon this analysis to develop the \acf{evdt} Framework. Third, I demonstrated the framework in two case studies, one a rather straightforward application and one a significant deviation that still contained useful lessons. In pursuing these case studies, I also provided useful analyses and tools to stakeholders.

As might be expected for a project of this scope, I did encounter challenges. Some of these challenges were internal to each case study and involved such things as the availability of data and understanding of the dynamics of the underlying phenomena. Some were part of the implementation, including a failure to conduct detailed usability studies or an urgency-and-complexity-based divergence from the \ac{evdt} Framework. Both provided useful lessons and the opportunity for future work.

It is my hope that the work presented in this thesis, I helped to fulfill a need for a generalized framework that combined multidisciplinary model integration, stakeholder participation and collaboration on a local scale, and the significant use of remote observation data for sustainable development. More broadly, I hope that this project serves to bridge the gap between disciplines (such as systems engineering and urban planning) and between stakeholders (such as academics, government officials, and members of the public) as we jointly try to pursue a more just and sustainable world.








\appendix
\chapter{\hlc[yellow]{Glossary}}

Language in general and technical jargon (of which this glossary qualifies) in particular is intended to communicate. This requires that both the speaker and the listener have some common understanding of the terms used. For this reason, I rarely find it helpful to generate new definitions for commonly used words, except to clarify when there is some significant discrepancies in how the term is commonly used. It is generally preferable to coin a new term if a new meaning is required (see, for instance Myoa Bailey's coining of the term \textit{misogynoir} \cite{baileyMoreOriginMisogynoir} or the significantly less elegant socio-environmental-technical system in this document).  

\textbf{Availability/Accessibility:}

%\textbf{Collaborative Systems:} A system that is not under central control, either in its conception, development, or operation. They tend to be assembled and operated through the voluntary choices of the participants, not through the dictates of an individual client \cite{maierArtSystemsArchitecting2009}.

\textbf{Context:}

\textbf{Critical Remote Sensing:} 

\textbf{\acf{dss}:} A technical system aimed at facilitating and improving decision-making. Functions can include visualization of data, analysis of past data, simulations of future outcomes, and comparisons of options.

\textbf{\acf{eo}:} 

\textbf{Ecosystem Services:}

\textbf{\acf{evdt}:} A four-part modeling framework created by Space Enabled for use in \acp{sets} and sustainable development applications \cite{reidCombiningSocialEnvironmental2019}. For more detail, including diagrams, see Chapter \ref{ch:evdt}.

\textbf{Form:}

\textbf{Function:}

\textbf{\acf{gis}:} Any digital system for storing, visualizing, and analyzing geospatial data, that is data that has some geographic component. The term can also be used to discuss specific systems, a method that uses such systems, a field of studying focusing on or involving such systems, or even the set of institutions and social practices that make use of such a system \cite{sheppardGISSocietyResearch1995}. For more discussion of this definition, see Section \ref{sec:gis}.

\textbf{Multidisciplinary Optimization:} A methodology for the design of systems in which strong interaction between disciplines motivates designers to simultaneously manipulate variables in several disciplines \cite{sobieszczanski-sobieskiMultidisciplinaryAerospaceDesign1997}.

\textbf{Multi-Stakeholder Decision-Making:} Any decision-making process in which more than one stakeholder must collaborate to reach a decision \cite{fitzgeraldRecommendationsFramingMultistakeholder2016}. This can take a variety of forms, including cooperation, negotiation, voting, or consultation \cite{garberMultiStakeholderTradeSpace2015}.

\textbf{Objective:}

\textbf{\acf{osse}:} A method of investigating the potential impacts of prospective observing systems through the generation of simulated observations that are then ingested into a data assimilation system and compared to other real-world data or other simulated data. Most commonly used for remote observation satellite design for purposes of meteorology \cite{masutaniObservingSystemSimulation2010} .

%\textbf{Organizational Policy:} Policy, decision-making, and politics within an organizational stakeholder. This includes decision-making policies, mechanisms of institutional learning and memory, capability development, etc. See \textcolor{black}{the Day 1 Response} for further discussion.

\textbf{\acf{pgis}:} A subset of \ac{gis} that seeks to directly involve the public and other stakeholders, including government officials, \acp{ngo}, private corporations, etc \cite{sieberPublicParticipationGeographic2006}. It should be noted that these means involvement in both the production of data and in its application, not merely one or the other \cite{weinerParticipatoryGeographicInformation2007, talenBottomUpGIS2000}. This is to be contrasted with the older term, \ac{ppgis}, which focuses specifically on the involvement of the public and not that of government agencies or other organizations \cite{sieberPublicParticipationGeographic2006}. For more discussion of this and related terms, see Section \ref{sec:collaborative}.

\textbf{Planning}: ``the premeditation of action, in contrast to management [which is] the direct control of action" \cite{harrisLocationalModelsGeographic1993}. In general, planning tends to concern itself with more long-term affairs that management does, during which it strives for the "avoidance of unintended consequences while pursuing intended goals." Models, and their specific implementations as decision/planning support tools, are one means of achieving this. The term is often prefaced with `urban' or `regional' to indicate the specific spatial scale under consideration.

\textbf{\acf{pss}:} A type of \ac{dss} specifically designed to support urban or regional planning efforts. These often involve longer time scales and more general/strategic decisions than most \acp{dss}. In general, this work will use the more general term, \ac{dss}, and will only use \ac{pss} when referring to the literature.

\textbf{Remote Observation}: Any form of data collection that takes place at some remote distance from the subject matter \cite{jensenRemoteSensingEnvironment2006}. While there is no specific distance determining whether a collector is `remote,' in practice this tends to mean some distance of more than a quarter of a kilometer. Handheld infrared measurement devices are thus usually excluded (and thereby classified as \textit{in-situ} observations. Aerial and satellite imagery are definitively in the remote observation category. Low altitude drone imagery, particularly when the operator is standing in the field of view, is a gray area that is not well categorized at this time.

\textbf{Remote Sensing:} See \textit{remote observation}.

\textbf{Scenario Planning}: A particular form of planning that focuses on long-term strategic decisions through the representation of multiple, plausible futures of a system of interest \cite{goodspeedScenarioPlanningCities2020}. These futures are often generated by models such as \ac{evdt}.

\textbf{Sustainable Development}: The integration of three separate, previously separate fields: economic development, social development and environmental protection \cite{worldsummitonsustainabledevelopmentPlanImplementationWorld2002}.  For a more detailed discussion of the history of this term, see Section \ref{sec:sustain}.

\textbf{Socio-environmental System:} The complex phenomena that occurs due to the interactions of human and natural systems \cite{elsawahEightGrandChallenges2020}.

\textbf{Sociotechnical System:} Technical works involving significant social participation, interests, and concerns \cite{maierArtSystemsArchitecting2009}.

\textbf{Socio-environmental-technical System:} A system in which social, environmental, and technical subsystems are linked together in such a way that none can be neglected without compromising the modeling, planning, or forecasting objectives at hand. This can be seen as the combination of the terms sociotechnical system and socio-environmental system. Note the particular emphasis on the needs of the observer, not the inherent system itself, as virtually all systems on Earth can be viewed as socio-environmental-technical Systems.

\textbf{Stakeholder:} 

\textbf{Stakeholder Analysis:} Identifying, mapping, and analyzing the stakeholders in a system and their connections to one another in order to inform the design of the system. This involves both qualitative and quantitative tools, such as the Stakeholder Requirements Definition Process \cite{incoseINCOSESystemsEngineering2015} and Stakeholder Value Network Analysis \cite{fengDependencyStructureMatrix2010a}. It should be noted that this term is commonly used by systems engineers but is not clearly defined as some specific list of methods. In a Space Enabled context, it commonly refers to the coding of qualitative interviews with stakeholders to elicit such items as needs, desired outcomes, and objectives. These are then often analyzed in some other method, such as Stakeholder Value Network Analysis.

\textbf{Systems Architecture/Architeting:} As defined by Maier, the art and science of creating and building complex systems. That part of systems development most concerned with scoping, structuring, and certification \cite{maierArtSystemsArchitecting2009}. This tends to refer to the high level form and function of a system, rather than detailed design. Other's, such as Crawley prefer to characterize it as the mapping of function to form such that the essential features of the system are represented. The intent of architecture is to reduce ambiguity, employ creativity, and manage complexity \cite{crawleySystemArchitectureStrategy2015}. Arguably this is a more specific formulation of Maier's definition. In general, Space Enabled and I tend to use Crawley's definition, both due to its clarity, and for the various qualitative and quantitative methods that have been developed to work well with this formulation.

\textbf{Systems Engineering:} An interdisciplinary approach and means to enable the realization of successful systems. It focuses on holistically and concurrently understanding stakeholder needs; exploring opportunities; documenting requirements; and synthesizing, verifying, validating, and evolving solutions while considering the complete problem, from system concept exploration through system disposal \cite{systemsengineeringbodyofknowledgeSystemsEngineeringGlossary2021}. 
For a more detailed discussion of this definition, including its flaws, see Section \ref{sec:se}.

\textbf{Tradespace:}  The space spanned by the completely enumerated design variables, i.e. the set of possible design options \cite{rossTradespaceExplorationParadigm2005}.

\textbf{Tradespace Exploration:} A process by which various options with a tradespace may be examined and compared in the absence of a single utility function, such as when multiple stakeholders are involved or multiple contexts with no clear priority exist \cite{rossTradespaceExplorationParadigm2005}.

\clearpage
\newpage

\chapter{\hlc[green]{Rio de Janeiro Stakeholder Interview Questions}} \label{interview-questions}

The following questions were used during the stakeholder interviews and meetings conducted during field visits to Rio de Janeiro in August of 2019 and March of 2020. This list does not include followup questions triggered by stakeholder responses.

\begin{enumerate}\setlength{\itemsep}{0pt}\setlength{\parskip}{0pt}
    \item{To confirm, are you okay being recorded?}
    \item{What is your name?}
    \item{What organization are you associated/work with?}
    \item{What is your role there?}
    \item{What is your organization's primary goal/mission? How does it usually pursue that goal?}
    \item{What are some example projects/activities?}
    \item{How do individual projects in your organization originate?}
    \item{What stakeholders does your organization work with?}
    \item{What do you view as the primary pressures on mangroves in the Rio de Janeiro area?}
    \item{What do you view as the primary pressures on the people living near the mangroves?}
    \item{What, if any, interest does your organization have in the Rio de Janeiro mangroves?}
    \item{Does your organization use any remote sensing data?}
    \item{Has your organization ever participated in the design of an earth observation satellite? Or considered doing so?}
    \item{What other sources of data does your organization rely upon?}
    \item{What are some questions that you would like to be able to answer but can't? What are some challenges that your organization faces?}
    \item{How much are you or your organization able to explore new data sources or data analysis methods, as opposed to continuing to rely upon your current methods?}
    \item{Anything else you want to add?}
\end{enumerate}
\chapter{\hlc[green]{COVID-19 Response Policy Summary}} \label{app:policy-summary}

The below tables summarize what specifically policies were included in each \ac{covid} response phase in each of the study locations for Chapter \ref{ch:vida}, as well as how these policies were quantified in the initial round of normalization. These are based on a combination of input from collaborating stakeholders and reference to news publications, official statutes, and CoronaNet \cite{CoronaNetResearchProject}. This was used to help construct the policy quantification scores shown in Figure \ref{fig:policy-comparison} and would have been used to help to construct the later version detailed in Table \ref{tab:policy-system}.

\begin{landscape}
\scriptsize
\csvreader[
  longtable=|C{1.5cm}|C{2cm}|C{1.5cm}|C{3cm}|C{3cm}|C{3cm}|C{2cm}|,
  table head=\caption{Summary of policy actions taken by each partner region \label{tab:covid-policy-history}}\\
    \toprule\textbf{Location} & \textbf{Policy Name} & \textbf{Mask Mandate?} & \textbf{Border Closures?} & \textbf{Closures of Businesses and Public Spaces}  & \textbf{Curfews (of individuals or for businesses)} & \textbf{Impact on Schools} \\ \midrule\endfirsthead
    \toprule\textbf{Location} & \textbf{Policy Name} & \textbf{Mask Mandate?} & \textbf{Border Closures?} & \textbf{Closures of Businesses and Public Spaces}  & \textbf{Curfews (of individuals or for businesses)} & \textbf{Impact on Schools} \\ \midrule\endhead
    \bottomrule\endfoot,
  late after line=\\ \hline,
]{Figures/chap5/PolicyBreakdownsHistory.csv}{1=\loc,2=\name,3=\mask,4=\border,5=\close,6=\curfew,7=\school}
		{\loc & \name & \mask & \border & \close & \curfew & \school}

\clearpage

\scriptsize
\csvreader[
  longtable=|C{1.5cm}|C{2cm}|C{1.5cm}|C{1.5cm}|C{2cm}|C{2cm}|C{1cm}|C{1.5cm}|C{1cm}|C{1.5cm}|,
  table head=\caption{COVID-19 Response Policy Quantification - Initial Version \label{tab:covid-policy-scores}}\\
    \toprule\textbf{Location} & \textbf{Policy} & \textbf{Mask Mandate} & \textbf{Travel Restrictions} & \textbf{Business and Public Space Closures}  & \textbf{Gathering Restrictions} & \textbf{Curfews} & \textbf{School Closures} & \textbf{Average} & \textbf{Category} \\ \midrule\endfirsthead
    \toprule\textbf{Location} & \textbf{Policy} & \textbf{Mask Mandate} & \textbf{Travel Restrictions} & \textbf{Business and Public Space Closures}  & \textbf{Gathering Restrictions} & \textbf{Curfews} & \textbf{School Closures} & \textbf{Average} & \textbf{Category} \\ \midrule\endhead
    \bottomrule\endfoot,
  late after line=\\ \hline,
]{Figures/chap5/PolicyBreakdownsNormalization.csv}{1=\loc,2=\name,3=\mask,4=\border,5=\close,6=\curfew,7=\school,8=\rec,9=\ave,10=\cat}
		{\loc & \name & \mask & \border & \close & \curfew & \school & \rec & \ave & \cat}


\end{landscape}
\chapter{Code Repositories and Other EVDT Publications} \label{app:code}

One of the goals of the \ac{evdt} Framework is to be provide openly available code and data products, enabling their access by a wide audience and their reuse for future projects. This thesis is one part that, providing a guide to the framework and examples of its prior use. This appendix contains links to code repositories and to other \ac{evdt} publications, some of which are referenced throughout the thesis.

\textbf{Western Rio de Janeiro Development \& Mangroves Code}

\begin{itemize}[itemsep=0pt,parsep=0pt]
	\item{\ac{gee} code used for \ac{evdt} analyses: \url{https://code.earthengine.google.com/?accept_repo=users/jackreid/thesis}}
	\item{\ac{dss} code: \url{https://github.com/mitmedialab/evdt}}
\end{itemize}


\textbf{Vida \ac{dss} for COVID-19 Response Code}

\begin{itemize}[itemsep=0pt,parsep=0pt]
	\item{\ac{gee} code used for \ac{evdt} analyses: \url{https://code.earthengine.google.com/?accept_repo=users/jackreid/vida}}
	\item{\ac{dss} code and Python code used for \ac{evdt} analyses: \url{https://github.com/mitmedialab/Vida_Modeling}}
	\item{Miscellaneous functions useful for working with \ac{gee}: \url{https://github.com/mitmedialab/gee_custom_utilities}}
\end{itemize}

\textbf{Other \ac{evdt} And Related Publications:} \url{https://dspace.mit.edu/handle/1721.1/147051}
%\chapter{Code Repositories and Other EVDT Publications} \label{app:code}

One of the goals of the \ac{evdt} Framework is to be provide openly available code and data products, enabling their access by a wide audience and their reuse for future projects. This thesis is one part that, providing a guide to the framework and examples of its prior use. This appendix contains links to code repositories and to other \ac{evdt} publications, some of which are referenced throughout the thesis.

\textbf{Western Rio de Janeiro Development \& Mangroves Code}

\begin{itemize}[itemsep=0pt,parsep=0pt]
	\item{\ac{gee} code used for \ac{evdt} analyses: \url{https://code.earthengine.google.com/?accept_repo=users/jackreid/thesis}}
	\item{\ac{dss} code: \url{https://github.com/mitmedialab/evdt}}
\end{itemize}


\textbf{Vida \ac{dss} for COVID-19 Response Code}

\begin{itemize}[itemsep=0pt,parsep=0pt]
	\item{\ac{gee} code used for \ac{evdt} analyses: \url{https://code.earthengine.google.com/?accept_repo=users/jackreid/vida}}
	\item{\ac{dss} code and Python code used for \ac{evdt} analyses: \url{https://github.com/mitmedialab/Vida_Modeling}}
	\item{Miscellaneous functions useful for working with \ac{gee}: \url{https://github.com/mitmedialab/gee_custom_utilities}}
\end{itemize}

\textbf{Other \ac{evdt} And Related Publications:} \url{https://dspace.mit.edu/handle/1721.1/147051}
\include{biblio}
\end{document}

