% $Log: abstract.tex,v $
% Revision 1.1  93/05/14  14:56:25  starflt
% Initial revision
% 
% Revision 1.1  90/05/04  10:41:01  lwvanels
% Initial revision
% 
%
%% The text of your abstract and nothing else (other than comments) goes here.
%% It will be single-spaced and the rest of the text that is supposed to go on
%% the abstract page will be generated by the abstractpage environment.  This
%% file should be \input (not \include 'd) from cover.tex.
This work aims to demonstrate the viability of a particular methodology for increasing the accessibility and relevance of earth observation data products to a wider audience of local decision-makers through the development of clearer linkages between environmental modeling and societal impact, while laying the groundwork for a more detailed consideration of (c). To that end, this work centers on exploring the efficacy and difficulties of \textbf{\textit{collaboratively developing}} a \textbf{\textit{systems-architecture-informed}}, multidisciplinary \textbf{\textit{\ac{gis} \ac{dss}}} for \textbf{\textit{sustainable development}} applications that makes significant use of \textbf{\textit{remote observation data}}. 

This is done through the development and evaluation of \acp{dss} for two primary applications: (1) mangrove forest management and conservation in the state of Rio de Janeiro, Brazil; and (2) coronavirus response in six metropolitan areas across Angola, Brazil, Chile, Indonesia, Mexico, and the United States. In both cases, the methodology involves the application of the system architecture framework, an approach that has been previously adapted from the aerospace engineering discipline by Prof. Wood for use in sociotechnical systems. This includes using stakeholder mapping and network analysis to inform the design of the \ac{dss} in question. Other components of the methodology taken in this work are developing the \ac{dss} through an iterative and collaborative process with specific stakeholders; pursuing targeted, related analyses, such as on the value of certain ecosystem services, the value of remote sensing information, and human responses to various policies; and evaluating the usefulness of both the \ac{dss} and the development process through interviews, workshops, and other feedback mechanisms.

All of this takes place under the umbrella of the \ac{evdt} Modeling Framework for combining remote observation and other types of data to inform decision-making in complex socio-environmental systems, particularly those pertaining to sustainable development. As the name suggests, \ac{evdt} integrates four models into one tool: the Environment (data including Landsat, Sentinel, VIIRs, Planet Lab’s PlanetScope, etc.; Human Vulnerability and Societal Impact (data including census and survey-based demographic data, NASA’s Socioeconomic Data and Applications Center, etc.); Human Behavior and Decision-Making (data including policy histories, mobility data, and urban nightlight data); and Technology Design for earth observation systems including satellites, airborne platforms and in-situ sensors (data including design parameter vectors for such systems). The data from each of these domains is used by established models in each domain, which are adapted to work in concert to address the needs identified during the stakeholder analysis. This framework is currently being used by several researchers in the Space Enabled Research Group and elsewhere. The capabilities provided by this framework will improve the management of earth observation and socioeconomic data in a format usable by non-experts, while harnessing cloud computing, machine learning, economic analysis, complex systems modeling, and model-based systems engineering.

