\chapter{Glossary}

Language in general and technical jargon (of which this glossary qualifies) in particular is intended to communicate. This requires that both the speaker and the listener have some common understanding of the terms used. For this reason, I rarely find it helpful to generate new definitions for commonly used words, except to clarify when there is some significant discrepancies in how the term is commonly used. It is generally preferable to coin a new term if a new meaning is required (see, for instance Myoa Bailey's coining of the term \textit{misogynoir} \cite{baileyMoreOriginMisogynoir} or the significantly less elegant socio-environmental-technical system in this document).  

%\textbf{Collaborative Systems:} A system that is not under central control, either in its conception, development, or operation. They tend to be assembled and operated through the voluntary choices of the participants, not through the dictates of an individual client \cite{maierArtSystemsArchitecting2009}.

\textbf{\acf{dss}:} A technical system aimed at facilitating and improving decision-making. Functions can include visualization of data, analysis of past data, simulations of future outcomes, and comparisons of options.

\textbf{\acf{evdt}:} A four-part modeling framework created by Space Enabled for use in \acp{sets} and sustainable development applications \cite{reidCombiningSocialEnvironmental2019}. For more detail, including diagrams, see Chapter \ref{ch:evdt}.

\textbf{\acf{gis}:} Any digital system for storing, visualizing, and analyzing geospatial data, that is data that has some geographic component. The term can also be used to discuss specific systems, a method that uses such systems, a field of studying focusing on or involving such systems, or even the set of institutions and social practices that make use of such a system \cite{sheppardGISSocietyResearch1995}. For more discussion of this definition, see Section \ref{sec:gis}.

\textbf{Multidisciplinary Optimization:} A methodology for the design of systems in which strong interaction between disciplines motivates designers to simultaneously manipulate variables in several disciplines \cite{sobieszczanski-sobieskiMultidisciplinaryAerospaceDesign1997}.

\textbf{Multi-Stakeholder Decision-Making:} Any decision-making process in which more than one stakeholder must collaborate to reach a decision \cite{fitzgeraldRecommendationsFramingMultistakeholder2016}. This can take a variety of forms, including cooperation, negotiation, voting, or consultation \cite{garberMultiStakeholderTradeSpace2015}.

\textbf{\acf{osse}:} A method of investigating the potential impacts of prospective observing systems through the generation of simulated observations that are then ingested into a data assimilation system and compared to other real-world data or other simulated data. Most commonly used for remote observation satellite design for purposes of meteorology \cite{masutaniObservingSystemSimulation2010} .

%\textbf{Organizational Policy:} Policy, decision-making, and politics within an organizational stakeholder. This includes decision-making policies, mechanisms of institutional learning and memory, capability development, etc. See \textcolor{black}{the Day 1 Response} for further discussion.

\textbf{\acf{pgis}:} A subset of \ac{gis} that seeks to directly involve the public and other stakeholders, including government officials, \acp{ngo}, private corporations, etc \cite{sieberPublicParticipationGeographic2006}. It should be noted that these means involvement in both the production of data and in its application, not merely one or the other \cite{weinerParticipatoryGeographicInformation2007, talenBottomUpGIS2000}. This is to be contrasted with the older term, \ac{ppgis}, which focuses specifically on the involvement of the public and not that of government agencies or other organizations \cite{sieberPublicParticipationGeographic2006}. For more discussion of this and related terms, see Section \ref{sec:collaborative}.

\textbf{Planning}: ``the premeditation of action, in contrast to management [which is] the direct control of action" \cite{harrisLocationalModelsGeographic1993}. In general, planning tends to concern itself with more long-term affairs that management does, during which it strives for the "avoidance of unintended consequences while pursuing intended goals." Models, and their specific implementations as decision/planning support tools, are one means of achieving this. The term is often prefaced with `urban' or `regional' to indicate the specific spatial scale under consideration.

\textbf{\acf{pss}:} A type of \ac{dss} specifically designed to support urban or regional planning efforts. These often involve longer time scales and more general/strategic decisions than most \acp{dss}. In general, this work will use the more general term, \ac{dss}, and will only use \ac{pss} when referring to the literature.

\textbf{Remote Observation}: Any form of data collection that takes place at some remote distance from the subject matter \cite{jensenRemoteSensingEnvironment2006}. While there is no specific distance determining whether a collector is `remote,' in practice this tends to mean some distance of more than a quarter of a kilometer. Handheld infrared measurement devices are thus usually excluded (and thereby classified as \textit{in-situ} observations. Aerial and satellite imagery are definitively in the remote observation category. Low altitude drone imagery, particularly when the operator is standing in the field of view, is a gray area that is not well categorized at this time.

\textbf{Scenario Planning}: A particular form of planning that focuses on long-term strategic decisions through the representation of multiple, plausible futures of a system of interest \cite{goodspeedScenarioPlanningCities2020}. These futures are often generated by models such as \ac{evdt}.

\textbf{Sustainable Development}: The integration of three separate, previously separate fields: economic development, social development and environmental protection \cite{worldsummitonsustainabledevelopmentPlanImplementationWorld2002}.  For a more detailed discussion of the history of this term, see Section \ref{sec:sustain}.

\textbf{Socio-environmental System:} The complex phenomena that occurs due to the interactions of human and natural systems \cite{elsawahEightGrandChallenges2020}.

\textbf{Sociotechnical System:} Technical works involving significant social participation, interests, and concerns \cite{maierArtSystemsArchitecting2009}.

\textbf{Socio-environmental-technical System:} A system in which social, environmental, and technical subsystems are linked together in such a way that none can be neglected without compromising the modeling, planning, or forecasting objectives at hand. This can be seen as the combination of the terms sociotechnical system and socio-environmental system. Note the particular emphasis on the needs of the observer, not the inherent system itself, as virtually all systems on Earth can be viewed as socio-environmental-technical Systems.

\textbf{Stakeholder Analysis:} Identifying, mapping, and analyzing the stakeholders in a system and their connections to one another in order to inform the design of the system. This involves both qualitative and quantitative tools, such as the Stakeholder Requirements Definition Process \cite{incoseINCOSESystemsEngineering2015} and Stakeholder Value Network Analysis \cite{fengDependencyStructureMatrix2010a}. It should be noted that this term is commonly used by systems engineers but is not clearly defined as some specific list of methods. In a Space Enabled context, it commonly refers to the coding of qualitative interviews with stakeholders to elicit such items as needs, desired outcomes, and objectives. These are then often analyzed in some other method, such as Stakeholder Value Network Analysis.

\textbf{Systems Architecture/Architeting:} As defined by Maier, the art and science of creating and building complex systems. That part of systems development most concerned with scoping, structuring, and certification \cite{maierArtSystemsArchitecting2009}. This tends to refer to the high level form and function of a system, rather than detailed design. Other's, such as Crawley prefer to characterize it as the mapping of function to form such that the essential features of the system are represented. The intent of architecture is to reduce ambiguity, employ creativity, and manage complexity \cite{crawleySystemArchitectureStrategy2015}. Arguably this is a more specific formulation of Maier's definition. In general, Space Enabled and I tend to use Crawley's definition, both due to its clarity, and for the various qualitative and quantitative methods that have been developed to work well with this formulation.

\textbf{Systems Engineering:} An interdisciplinary approach and means to enable the realization of successful systems. It focuses on holistically and concurrently understanding stakeholder needs; exploring opportunities; documenting requirements; and synthesizing, verifying, validating, and evolving solutions while considering the complete problem, from system concept exploration through system disposal \cite{systemsengineeringbodyofknowledgeSystemsEngineeringGlossary2021}. 
For a more detailed discussion of this definition, including its flaws, see Section \ref{sec:se}.

\textbf{Tradespace:}  The space spanned by the completely enumerated design variables, i.e. the set of possible design options \cite{rossTradespaceExplorationParadigm2005}.

\textbf{Tradespace Exploration:} A process by which various options with a tradespace may be examined and compared in the absence of a single utility function, such as when multiple stakeholders are involved or multiple contexts with no clear priority exist \cite{rossTradespaceExplorationParadigm2005}.

\clearpage
\newpage
