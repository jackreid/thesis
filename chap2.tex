%% This is an example first chapter.  You should put chapter/appendix that you
%% write into a separate file, and add a line \include{yourfilename} to
%% main.tex, where `yourfilename.tex' is the name of the chapter/appendix file.
%% You can process specific files by typing their names in at the 
%% \files=
%% prompt when you run the file main.tex through LaTeX.
\chapter{Literature Survey / Framing / Theory}


\section{Development}

\subsection{History and Theory}

Refer to and respond to argument that computational models and systems engineering act predominantly to uphold exisiting power structure, the history of aerospace engineers who think that they can solve violence in urban areas. Use timeline / categories of theories diagram and position this work on it \cite{mazza2017}

Also use diagram/framework from \cite{marcuseThreeHistoricCurrents2016}

More critical remarks about systems engineering on pg. 12 of \cite{robinsonDecisionmakingUrbanPlanning1972}

Dangers of relying on just one theory, or of using metaphors to enact policies (pg.23-24) \cite{ostromGoverningCommonsEvolution2015} instead use "theoretical pluralism" /cite{turkleEmpathyDiariesMemoir2021}

What to avoid situation of Boston central planners wanting to demolish the North End because they imagine it as a ghetto. \cite{jacobsDeathLifeGreat2016}

\ac{pgis}: Refer to macro-micro framework from Table 2.1, pg.17. What parts this thesis covers and what parts we envision EVDT covering in the long term \cite{jankowskiGISGroupDecision2001}

Respond to critiques of central planning / technocratic efforts by Easterly \cite{easterly2015}

Planning has come a long way from focusing on single page map and a timescale of 20-30 years (Section 2 Introduction of \cite{robinsonDecisionmakingUrbanPlanning1972})

By providing tools for more participation, we are not necessarily doing anything radical. "Democracies rarely end up expropriating and redistributing capital" \cite{fainsteinSpatialJusticePlanning2016}. "Participation is not power; its reform is not radical" \cite{marcuseThreeHistoricCurrents2016}. Some argue that neoliberalism in factor prefers to use participation as a means of underming resistance, rather than violence, though this has the risk of providing a structure for coalition building and radicalization \cite{miraftabInsurgentPlanningSituating2016}. In fact, increased community involvement can result in more restrictive, unambitious goals that are not in the interests of certain minorities (Section 1, Chapter 2 of \cite{robinsonDecisionmakingUrbanPlanning1972}).

\subsection{Sustainable Development}

Use triangle diagram of sustainability, respond to critique that sustainability has been watered down in meaning "If both the world bank and radical ecologists both believe in it", respond to critique of Sustainable Development Master Plans generated by states, need to be able to translate from one domain to another to avoid one dominating the others \cite{campbellGreenCitiesGrowing2016}

Talk about \acp{mdg} and \acp{sdg}, pull from my previously written articles, also \cite{unitednationsWhoWillBe2013}

Respond to critiques of \acp{mdg}/\acp{sdg} \cite{alstonShipsPassingNight2005, reddyGlobalDevelopmentGoals2008}

\subsection{[Tentative] Informality}
Discuss and critique of informality as a concept \cite{royUrbanInformalityProduction2016}

De Soto argues that the poor already have assets, just needs to be formalized. \cite{sotoMysteryCapitalWhy2003} though others argue that this is just results in a cycle of appeasment / welfare \cite{hollandForbearanceRedistributionPolitics2017}


\section{Types of places that EVDT deals with}

Commonly has to do with \acp{cpr}. Talk about the three common ways of managing \acp{cpr}: Central management, privatization, self-management. Bring in Table 3.1 (pg. 90) showing design principles of long-enduring self-management institutions. Refer to successful aspects of the water basin in California (incremental and sequential process to reduce the costs of local institutional supply, shared information at each step, intermediate benefits from initial investments were realized prior to larger investments, transoformed structure of incentives within which fuure strategic decisions can be made) (pg. 137. \cite{ostromGoverningCommonsEvolution2015}

\section{Complex Systems and Modeling}

The growth and development of cities is a complex system. Much work has been done using cellular automata and fractals to model them \cite{battyCitiesComplexity2005}

Urban planners have been striving towards useful indices for decades (Section 1, Cahpter 3 of \cite{robinsonDecisionmakingUrbanPlanning1972})

Difference between planning for the future and planning the future (Section 2, Introduction of \cite{robinsonDecisionmakingUrbanPlanning1972})

"Alternatives [explored in a simulation] must reflect the goals sought." (Section 2, Chapter 5 of \cite{robinsonDecisionmakingUrbanPlanning1972})

Position \ac{evdt} using the different dimensions of models proposed in (Section 2, Chapter 6 of \cite{robinsonDecisionmakingUrbanPlanning1972})

More detailed modeling and including environmental factors gets around some of the inaccurate assumptions and constraints of traditional cost-benefit analysis (Section 3, Chapter 9 of \cite{robinsonDecisionmakingUrbanPlanning1972})


This work does not directly incorporate mechanisms for multi-stakeholder negotiation or tradespace exploration, but it is amenable to extension with such mechanisms (refer to SEAri research)

Law of requisite variety applies to urban planning (Section 5, Chapter 16 of \cite{robinsonDecisionmakingUrbanPlanning1972})


\section{EVDT Framework}

Is not itself a means of planning and implementing projects. It is not Planning-Programming-Budgeting (PPB) (as described in Section 3, Chapter 10 of \cite{robinsonDecisionmakingUrbanPlanning1972})

%\section{Section sample 1}


%\begin{enumerate}
%  \item Item 1.
%  \item Item 2.
%  \item Item 3.
%\end{enumerate}
%
%
%
%\begin{eqnarray*}
%a_i & = & a_j + a_k \\
%a_i & = & 2a_j + a_k \\
%a_i & = & 4a_j + a_k \\
%a_i & = & 8a_j + a_k \\
%a_i & = & a_j - a_k \\
%a_i & = & a_j \ll m \mbox{shift}
%\end{eqnarray*}
%instead of the multiplication.  For example, you could use:
%\begin{eqnarray*}
%r & = & 4s + s\\
%r & = & r + r
%\end{eqnarray*}
%Or by xx:
%\begin{eqnarray*}
%t & = & 2s + s \\
%r & = & 2t + s \\
%r & = & 8r + t
%\end{eqnarray*}
