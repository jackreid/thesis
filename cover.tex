% -*-latex-*-
% 
% For questions, comments, concerns or complaints:
% thesis@mit.edu
% 
%
% $Log: cover.tex,v $
% Revision 1.9  2019/08/06 14:18:15  cmalin
% Replaced sample content with non-specific text.
%
% Revision 1.8  2008/05/13 15:02:15  jdreed
% Degree month is June, not May.  Added note about prevdegrees.
% Arthur Smith's title updated
%
% Revision 1.7  2001/02/08 18:53:16  boojum
% changed some \newpages to \cleardoublepages
%
% Revision 1.6  1999/10/21 14:49:31  boojum
% changed comment referring to documentstyle
%
% Revision 1.5  1999/10/21 14:39:04  boojum
% *** empty log message ***
%
% Revision 1.4  1997/04/18  17:54:10  othomas
% added page numbers on abstract and cover, and made 1 abstract
% page the default rather than 2.  (anne hunter tells me this
% is the new institute standard.)
%
% Revision 1.4  1997/04/18  17:54:10  othomas
% added page numbers on abstract and cover, and made 1 abstract
% page the default rather than 2.  (anne hunter tells me this
% is the new institute standard.)
%
% Revision 1.3  93/05/17  17:06:29  starflt
% Added acknowledgements section (suggested by tompalka)
% 
% Revision 1.2  92/04/22  13:13:13  epeisach
% Fixes for 1991 course 6 requirements
% Phrase "and to grant others the right to do so" has been added to 
% permission clause
% Second copy of abstract is not counted as separate pages so numbering works
% out
% 
% Revision 1.1  92/04/22  13:08:20  epeisach

% NOTE:
% These templates make an effort to conform to the MIT Thesis specifications,
% however the specifications can change. We recommend that you verify the
% layout of your title page with your thesis advisor and/or the MIT 
% Libraries before printing your final copy.
\title{Using Integrated Earth Observation-Informed Modeling to Inform Sustainable Development Decision-Making}

\author{Jack Reid}
% If you wish to list your previous degrees on the cover page, use the 
% previous degrees command:
%       \prevdegrees{A.A., Harvard University (1985)}
% You can use the \\ command to list multiple previous degrees
%       \prevdegrees{B.S., University of California (1978) \\
%                    S.M., Massachusetts Institute of Technology (1981)}
\department{Program in Media Arts and Sciences}

% If the thesis is for two degrees simultaneously, list them both
% separated by \and like this:
% \degree{Doctor of Philosophy \and Master of Science}
\degree{Doctor of Philosophy in Media Arts and Sciences}

% As of the 2007-08 academic year, valid degree months are September, 
% February, or June.  The default is June.
\degreemonth{December}
\degreeyear{2022}
\thesisdate{Dec 21, 2022}

%% By default, the thesis will be copyrighted to MIT.  If you need to copyright
%% the thesis to yourself, just specify the `vi' documentclass option.  If for
%% some reason you want to exactly specify the copyright notice text, you can
%% use the \copyrightnoticetext command.  
%\copyrightnoticetext{\copyright IBM, 1990.  Do not open till Xmas.}

% If there is more than one supervisor, use the \supervisor command
% once for each.
\supervisor{Danielle R. Wood}{ Assistant Professor of Media Arts and Sciences; Assistant Professor of Aeronautics and Astronautics}

% This is the department committee chairman, not the thesis committee
% chairman.  You should replace this with your Department's Committee
% Chairman.
\chairman{Tod Machover}{Chairman, Department Committee on Graduate Theses}

% Make the titlepage based on the above information.  If you need
% something special and can't use the standard form, you can specify
% the exact text of the titlepage yourself.  Put it in a titlepage
% environment and leave blank lines where you want vertical space.
% The spaces will be adjusted to fill the entire page.  The dotted
% lines for the signatures are made with the \signature command.
\maketitle

% The abstractpage environment sets up everything on the page except
% the text itself.  The title and other header material are put at the
% top of the page, and the supervisors are listed at the bottom.  A
% new page is begun both before and after.  Of course, an abstract may
% be more than one page itself.  If you need more control over the
% format of the page, you can use the abstract environment, which puts
% the word "Abstract" at the beginning and single spaces its text.

%% You can either \input (*not* \include) your abstract file, or you can put
%% the text of the abstract directly between the \begin{abstractpage} and
%% \end{abstractpage} commands.

% First copy: start a new page, and save the page number.
\cleardoublepage
% Uncomment the next line if you do NOT want a page number on your
% abstract and acknowledgments pages.
% \pagestyle{empty}
\setcounter{savepage}{\thepage}
\begin{abstractpage}
 \documentclass[notitlepage]{report}
\usepackage[left=1in, right=1in, top=1in, bottom=1in]{geometry}
%% Language and font encodings
\usepackage[english]{babel}
\usepackage{morewrites}

%\selectlanguage{spanish}
\usepackage[utf8x]{inputenc}
\usepackage[nolist]{acronym}
\usepackage{etoolbox}
\usepackage{graphicx}
\usepackage{titling}

%%%%%%%%%%%%%%%%%%%%%%%%%%%%%%%%%%%%%%
%SETTING BIBLIOGRAPHY FORMAT
%%%%%%%%%%%%%%%%%%%%%%%%%%%%%%%%%%%%%%
\usepackage{multibbl}
\addto{\captionsenglish}{\renewcommand{\bibname}{}}
\renewcommand{\bibname}{}
\patchcmd{\thebibliography}{\chapter*}{\subsection*}{}{}
\let\oldbibliography\thebibliography
\renewcommand{\thebibliography}[1]{%
  \oldbibliography{#1}%
  \setlength{\itemsep}{0pt}%
}


%%%%%%%%%%%%%%%%%%%%%%%%%%%%%%%%%%%%%%
%CUSTOM COMMANDS
%%%%%%%%%%%%%%%%%%%%%%%%%%%%%%%%%%%%%%
\newcommand{\ReadingList}[2]{
	\subsection*{#1}
	\addcontentsline{toc}{subsection}{#1}
	\newbibliography{#2}

	\nocite{#2}{*}
	\bibliographystyle{#2}{unsrt}
	\bibliography{#2}{#2}{}}

\newcommand*{\Signature}[1]{
    \par\noindent\makebox[1in]{Signature: } \makebox[5.5in]{\hrulefill} 
    \par\noindent\hfill\makebox[2.5in][r]{#1}}

%%%%%%%%%%%%%%%%%%%%%%%%%%%%%%%%%%%%%%
%EXAMINATION COMMITTEE NAMES AND CREDENTIALS
%%%%%%%%%%%%%%%%%%%%%%%%%%%%%%%%%%%%%%
\newcommand{\compr}{Danielle Wood}
\newcommand{\comte}{David Lagomasino}
\newcommand{\comco}{Sarah Williams}

\newcommand{\comprcredentials}{\newline\indent Assistant Professor of Media Arts and Sciences \newline\indent Program in Media Arts and Sciences \newline\indent Assistant Professor (Joint) of Aeronautics and Astronautics \newline\indent Department of Aeronautics and Astronautics \newline\indent Massachusetts Institute of Technology}

\newcommand{\comtecredentials}{\newline\indent Assistant Professor of Coastal Studies \newline\indent Department of Coastal Studies \newline\indent East Carolina University}

\newcommand{\comcocredentials}{\newline\indent Associate Professor of Technology and Urban Planning \newline\indent  Director of the Norman B. Leventhal Center for Advanced Urbanism \newline\indent Urban Science and Computer Science Program \newline\indent Department of Urban Studies and Planning \newline\indent Massachusetts Institute of Technology}

%%%%%%%%%%%%%%%%%%%%%%%%%%%%%%%%%%%%%%
%AREA NAMES
%%%%%%%%%%%%%%%%%%%%%%%%%%%%%%%%%%%%%%
\newcommand{\Apr}{Socio-environmental-technical Systems Design, Modeling, and Decision-Making}
\newcommand{\Ate}{Remote Observation of Natural and Social Phenomena}
\newcommand{\Aco}{Development, Data, and Justice in Socio-environmental-technical Systems}

\usepackage{titlesec}
\titleformat{\section}{\normalfont\Large\bfseries}{}{0pt}{}

%%%%%%%%%%%%%%%%%%%%%%%%%%%%%%%%%%%%%%
%ACRONYMS
%%%%%%%%%%%%%%%%%%%%%%%%%%%%%%%%%%%%%%
\begin{acronym}[VALUABLES]
	\acro{eo}[EO]{earth observation}
	\acro{nasa}[NASA]{National Aeronautics and Space Administration}
%	\acro{usaid}[USAID]{United States Agency for International Development}
%	\acro{servir}[SERVIR]{Sistema Regional De Visualizaci\'{o}n Y Monitoreo De Mesoam\'{e}rica}
	\acro{eos}[EOS]{Earth Observation System}
	%\acro{geo}[GEO]{Group of Earth Observations}
	\acro{osse}[OSSE]{Observing System Simulation Experiment}
	%\acro{un}[UN]{United Nations}
	%\acro{mdg}[MDG]{Millennium Development Goal}
	\acro{sdg}[SDG]{Sustainable Development Goal}
	\acro{fews}[FEWS NET]{Famine Early Warning Systems Network}
	%\acro{fema}[FEMA]{Federal Emergency Management Agency}
	%\acro{jpl}[JPL]{Jet Propulsion Laboratory}
	%\acro{dod}[DoD]{Department of Defense}
	\acro{osse}[OSSE]{Observing System Simulation Experiment}
	\acro{ngo}[NGO]{non-governmental organization}
	\acro{noaa}[NOAA]{National Oceanic and Atmospheric Administration}
	%\acro{firms}[FIRMS]{Fire Information for Resource Management System}
	\acro{ceos}[CEOS]{Committee on Earth Observation Satellites}
	\acro{gis}[GIS]{geospatial information system}
	%\acro{esa}[ESA]{European Space Agency}
	%\acro{cbers}[CBERS]{China-Brazil Earth Resources Satellite Program}
	%\acro{jaxa}[JAXA]{Japan Aerospace Exploration Agency}
	%\acro{eoc}[EOC]{Earth Observation Center}
	%\acro{usgs}[USGS]{US Geological Survey}
	%\acro{ipcc}[IPCC]{Intergovernmental Panel on Climate Change}
	%\acro{cad}[CAD]{computer-aided design}
	%\acro{arl}[ARL]{Application Readiness Level}
	%\acro{trl}[TRL]{Technology Readiness Level}
	%\acro{mbse}[MBSE]{model-based systems engineering}
	%\acro{siurb}[SIURB]{Sistema Municipal de Informa\c{c}\~{o}es Urbanas}
	\acro{dss}[DSS]{decision support system}
	\acro{evdt}[EVDT]{Environment, Vulnerability, Decision-Making, Technology}
	\acro{mit}[MIT]{Massachusetts Institute of Technology}
\end{acronym}

%%%%%%%%%%%%%%%%%%%%%%%%%%%%%%%%%%%%%%
%TITLE PAGE & TABLE OF CONTENTS
%%%%%%%%%%%%%%%%%%%%%%%%%%%%%%%%%%%%%%
\begin{document}
\pretitle{\begin{center}\Huge\bfseries}
\posttitle{\par\end{center}\vskip 0.5em}
\preauthor{\begin{center}\Large\ttfamily}
\postauthor{\end{center}}
\predate{\par\large\centering}
\postdate{\par}

\title{Using Integrated Earth Observation-Informed Modeling to Inform Sustainable Development Decision-Making \\  
\large Ph.D. Thesis Proposal Abstract and Committee Members} 
\author{%
Jack Reid\\ 
Space Enabled Research Group\\
MIT Media Lab\\
jackreid@mit.edu}

\maketitle

\section*{ }
Dear Professor Machover and the MAS Committee,
\vspace{\baselineskip}

\noindent This is to inform you that Jack Reid intends to develop a thesis on the subject matter described in the below abstract and under the supervision of the listed committee, whose short biographies are available following the abstract.

\vspace{\baselineskip}

\textbf{\compr} \comprcredentials 
 
\vspace{\baselineskip}

\textbf{\comte} \comtecredentials

\vspace{\baselineskip}

\textbf{\comco} \comcocredentials

\thispagestyle{plain}
\pagestyle{plain}

%\tableofcontents

%%%%%%%%%%%%%%%%%%%%%%%%%%%%%%%%%%%%%%
\chapter*{\vspace{-2.0cm}Abstract}
%%%%%%%%%%%%%%%%%%%%%%%%%%%%%%%%%%%%%%

%[A short overview of the key goals, questions, and expected contributions.]
\vspace{-1.0cm}
Over the past two decades satellite-based remote observation has blossomed. We have seen a rapid increase in the number of \acp{eos} in orbit, significant improvements in their capabilities, and much greater availability of the data that they produce. This trend has occurred as part of a greater trend of increasing data availability, computational power, and modeling ability. Unfortunately, up until now, this \ac{eo} data has been largely used only by governments and academics for scientific purposes, typically to understand and predict environmental phenomena. Large corporations and \acp{ngo} have recently been conducting their own analyses, but these have required significant expertise and resources, and the results have sadly been mostly unavailable to the broader public. 

There is a real need for (a) making remote observation data not just available but accessible to a broader audience by developing data products that are relevant to everyday individuals, particularly those involved in local, rather than national or global decision-making; (b) linking the \ac{eo}-supported environmental modeling with the societal impact of a changing environment; and (c) putting policy and sensor design decision-making in the hands of a broader population. 

This work aims to demonstrate the viability of a particular methodology for achieving (a) and (b), while laying the groundwork for a more detailed consideration of (c). To that end, this work centers on exploring the efficacy and difficulties of \textbf{\textit{collaboratively developing}} a \textbf{\textit{systems-architecture-informed}}, multidisciplinary \textbf{\textit{\ac{gis} \ac{dss}}} for \textbf{\textit{sustainable development}} applications that makes significant use of \textbf{\textit{remote observation data}}. 

This is done through the development and evaluation of \acp{dss} for two primary applications: (1) mangrove forest management and conservation in the state of Rio de Janeiro, Brazil; and (2) coronavirus response in six metropolitan areas across Angola, Brazil, Chile, Indonesia, Mexico, and the United States. In both cases, the methodology involves the application the system architecture framework, an approach that has been previously adapted from the aerospace engineering discipline by Prof. Wood for use in sociotechnical systems. This includes using stakeholder mapping and network analysis to inform the design of the \ac{dss} in question. Other components of the methodology taken in this work are developing the \ac{dss} through an iterative and collaborative process with specific stakeholders; pursuing targeted, related analyses, such as on the value of certain ecosystem services, the value of remote sensing information, and human responses to various policies; and evaluating the usefulness of both the \ac{dss} and the development process through interviews, workshops, and other feedback mechanisms.

All of this takes place under the umbrella of the \ac{evdt} Modeling Framework for combining \ac{eo} and other types of data to inform decision-making in complex socio-environmental systems, particularly those pertaining to sustainable development. As the name suggests, \ac{evdt} integrates four models into one tool: the Environment (data including Landsat, Sentinel, VIIRs, Planet Lab’s PlanetScope, etc.; Human Vulnerability and Societal Impact (data including census and survey-based demographic data, NASA’s Socioeconomic Data and Applications Center, etc.); Human Behavior and Decision-Making (data including policy histories, mobility data, and urban nightlight data); and Technology Design for earth observation systems including satellites, airborne platforms and in-situ sensors (data including design parameter vectors for such systems). The data from each of these domains is used by established models in each domain, which are adapted to work in concert to address the needs identified during the stakeholder analysis. This framework is currently being used by several researchers in the Space Enabled Research Group and elsewhere. The capabilities provided by this framework will improve the management of earth observation and socioeconomic data in a format usable by non-experts, while harnessing cloud computing, machine learning, economic analysis, complex systems modeling, and model-based systems engineering.




%%%%%%%%%%%%%%%%%%%%%%%%%%%%%%%%%%%%%%
\chapter*{\vspace{-2.0cm}Thesis Committee Biographies}
%%%%%%%%%%%%%%%%%%%%%%%%%%%%%%%%%%%%%%

\subsection*{Prof. Danielle Wood}

Danielle Wood is an Assistant Professor in the Program in Media Arts \& Sciences and holds a joint appointment in the Department of Aeronautics \& Astronautics at \ac{mit}. Within the Media Lab, Prof. Wood leads the Space Enabled Research Group which seeks to advance justice in Earth's complex systems using designs enabled by space. Prof. Wood is a scholar of societal development with a background that includes satellite design, earth science applications, systems engineering, and technology policy. In her research, Prof. Wood applies these skills to design innovative systems that harness space technology to address development challenges around the world. Prior to serving as faculty at \ac{mit}, Professor Wood held positions at \ac{nasa} Headquarters, \ac{nasa} Goddard Space Flight Center, Aerospace Corporation, Johns Hopkins University, and the United Nations Office of Outer Space Affairs. Prof. Wood studied at \ac{mit}, where she earned a Ph.D. in engineering systems, S.M. in aeronautics and astronautics, S.M. in technology policy, and S.B. in aerospace engineering.

\subsection*{Prof. David Lagomasino}

David Lagomasino is an Assistant Professor in the Department of Coastal Studies at East Carolina University. He previously studied at Florida International University, where he received a B.S. and a Ph.D. in Geological Sciences, in between which he received a M.S. in Geology at East Carolina University. Lagomasino uses satellite, airborne, drone, and ground measurements to identify areas of coastal resilience and vulnerability. His research links remotely sensed spatial data directly with stakeholders in order to address exposure and sensitivity issues for coastal/wetland management and ecosystem valuation. He has been involved in a number of  coastal blue carbon projects with funding from NASA’s Carbon Monitoring Systems Program,  NASA’s Biodiversity and Forecasting Program, USDA’s National Forest Inventory Assessment Program,  NASA’s New Investigator Program, and the Center for International Forestry. His goal is to provide meaningful information that will better inform coastal management practices while also inspiring students and the community to become environmental stewards in order to help sustain our coastal resources. Prior to his current post, he conducted research at NASA’s Goddard Space Flight Center just outside Washington, D.C., in partnership with the University of Maryland, to develop models that measure the where when, and why shorelines are the world are changing.

\subsection*{Prof. Sarah Williams}

Sarah Williams is an Associate Professor of Technology and Urban Planning at \ac{mit} where she is also Director of the Civic Data Design Lab and the Leventhal Center for Advanced Urbanism. Williams’ combines her training in computation and design to create communication strategies that expose urban policy issues to broad audiences and create civic change. She calls the process Data Action, which is also the name of her recent book published by \ac{mit} Press. Williams is co-founder and developer of Envelope.city, a web-based software product that visualizes and allows users to modify zoning in New York City.  Before coming to \ac{mit}, Williams was Co-Director of the Spatial Information Design Lab at Columbia University’s Graduate School of Architecture Planning and Preservation (GSAPP). Her design work has been widely exhibited including work in the Guggenheim, the Museum of Modern Art (MoMA), Venice Biennale, and the Cooper Hewitt Museum. Williams has won numerous awards including being named one of the top 25 technology planners and Game Changer by Metropolis Magazine. 


\end{document}

\end{abstractpage}

% Additional copy: start a new page, and reset the page number.  This way,
% the second copy of the abstract is not counted as separate pages.
% Uncomment the next 6 lines if you need two copies of the abstract
% page.
% \setcounter{page}{\thesavepage}
% \begin{abstractpage}
%  \documentclass[notitlepage]{report}
\usepackage[left=1in, right=1in, top=1in, bottom=1in]{geometry}
%% Language and font encodings
\usepackage[english]{babel}
\usepackage{morewrites}

%\selectlanguage{spanish}
\usepackage[utf8x]{inputenc}
\usepackage[nolist]{acronym}
\usepackage{etoolbox}
\usepackage{graphicx}
\usepackage{titling}

%%%%%%%%%%%%%%%%%%%%%%%%%%%%%%%%%%%%%%
%SETTING BIBLIOGRAPHY FORMAT
%%%%%%%%%%%%%%%%%%%%%%%%%%%%%%%%%%%%%%
\usepackage{multibbl}
\addto{\captionsenglish}{\renewcommand{\bibname}{}}
\renewcommand{\bibname}{}
\patchcmd{\thebibliography}{\chapter*}{\subsection*}{}{}
\let\oldbibliography\thebibliography
\renewcommand{\thebibliography}[1]{%
  \oldbibliography{#1}%
  \setlength{\itemsep}{0pt}%
}


%%%%%%%%%%%%%%%%%%%%%%%%%%%%%%%%%%%%%%
%CUSTOM COMMANDS
%%%%%%%%%%%%%%%%%%%%%%%%%%%%%%%%%%%%%%
\newcommand{\ReadingList}[2]{
	\subsection*{#1}
	\addcontentsline{toc}{subsection}{#1}
	\newbibliography{#2}

	\nocite{#2}{*}
	\bibliographystyle{#2}{unsrt}
	\bibliography{#2}{#2}{}}

\newcommand*{\Signature}[1]{
    \par\noindent\makebox[1in]{Signature: } \makebox[5.5in]{\hrulefill} 
    \par\noindent\hfill\makebox[2.5in][r]{#1}}

%%%%%%%%%%%%%%%%%%%%%%%%%%%%%%%%%%%%%%
%EXAMINATION COMMITTEE NAMES AND CREDENTIALS
%%%%%%%%%%%%%%%%%%%%%%%%%%%%%%%%%%%%%%
\newcommand{\compr}{Danielle Wood}
\newcommand{\comte}{David Lagomasino}
\newcommand{\comco}{Sarah Williams}

\newcommand{\comprcredentials}{\newline\indent Assistant Professor of Media Arts and Sciences \newline\indent Program in Media Arts and Sciences \newline\indent Assistant Professor (Joint) of Aeronautics and Astronautics \newline\indent Department of Aeronautics and Astronautics \newline\indent Massachusetts Institute of Technology}

\newcommand{\comtecredentials}{\newline\indent Assistant Professor of Coastal Studies \newline\indent Department of Coastal Studies \newline\indent East Carolina University}

\newcommand{\comcocredentials}{\newline\indent Associate Professor of Technology and Urban Planning \newline\indent  Director of the Norman B. Leventhal Center for Advanced Urbanism \newline\indent Urban Science and Computer Science Program \newline\indent Department of Urban Studies and Planning \newline\indent Massachusetts Institute of Technology}

%%%%%%%%%%%%%%%%%%%%%%%%%%%%%%%%%%%%%%
%AREA NAMES
%%%%%%%%%%%%%%%%%%%%%%%%%%%%%%%%%%%%%%
\newcommand{\Apr}{Socio-environmental-technical Systems Design, Modeling, and Decision-Making}
\newcommand{\Ate}{Remote Observation of Natural and Social Phenomena}
\newcommand{\Aco}{Development, Data, and Justice in Socio-environmental-technical Systems}

\usepackage{titlesec}
\titleformat{\section}{\normalfont\Large\bfseries}{}{0pt}{}

%%%%%%%%%%%%%%%%%%%%%%%%%%%%%%%%%%%%%%
%ACRONYMS
%%%%%%%%%%%%%%%%%%%%%%%%%%%%%%%%%%%%%%
\begin{acronym}[VALUABLES]
	\acro{eo}[EO]{earth observation}
	\acro{nasa}[NASA]{National Aeronautics and Space Administration}
%	\acro{usaid}[USAID]{United States Agency for International Development}
%	\acro{servir}[SERVIR]{Sistema Regional De Visualizaci\'{o}n Y Monitoreo De Mesoam\'{e}rica}
	\acro{eos}[EOS]{Earth Observation System}
	%\acro{geo}[GEO]{Group of Earth Observations}
	\acro{osse}[OSSE]{Observing System Simulation Experiment}
	%\acro{un}[UN]{United Nations}
	%\acro{mdg}[MDG]{Millennium Development Goal}
	\acro{sdg}[SDG]{Sustainable Development Goal}
	\acro{fews}[FEWS NET]{Famine Early Warning Systems Network}
	%\acro{fema}[FEMA]{Federal Emergency Management Agency}
	%\acro{jpl}[JPL]{Jet Propulsion Laboratory}
	%\acro{dod}[DoD]{Department of Defense}
	\acro{osse}[OSSE]{Observing System Simulation Experiment}
	\acro{ngo}[NGO]{non-governmental organization}
	\acro{noaa}[NOAA]{National Oceanic and Atmospheric Administration}
	%\acro{firms}[FIRMS]{Fire Information for Resource Management System}
	\acro{ceos}[CEOS]{Committee on Earth Observation Satellites}
	\acro{gis}[GIS]{geospatial information system}
	%\acro{esa}[ESA]{European Space Agency}
	%\acro{cbers}[CBERS]{China-Brazil Earth Resources Satellite Program}
	%\acro{jaxa}[JAXA]{Japan Aerospace Exploration Agency}
	%\acro{eoc}[EOC]{Earth Observation Center}
	%\acro{usgs}[USGS]{US Geological Survey}
	%\acro{ipcc}[IPCC]{Intergovernmental Panel on Climate Change}
	%\acro{cad}[CAD]{computer-aided design}
	%\acro{arl}[ARL]{Application Readiness Level}
	%\acro{trl}[TRL]{Technology Readiness Level}
	%\acro{mbse}[MBSE]{model-based systems engineering}
	%\acro{siurb}[SIURB]{Sistema Municipal de Informa\c{c}\~{o}es Urbanas}
	\acro{dss}[DSS]{decision support system}
	\acro{evdt}[EVDT]{Environment, Vulnerability, Decision-Making, Technology}
	\acro{mit}[MIT]{Massachusetts Institute of Technology}
\end{acronym}

%%%%%%%%%%%%%%%%%%%%%%%%%%%%%%%%%%%%%%
%TITLE PAGE & TABLE OF CONTENTS
%%%%%%%%%%%%%%%%%%%%%%%%%%%%%%%%%%%%%%
\begin{document}
\pretitle{\begin{center}\Huge\bfseries}
\posttitle{\par\end{center}\vskip 0.5em}
\preauthor{\begin{center}\Large\ttfamily}
\postauthor{\end{center}}
\predate{\par\large\centering}
\postdate{\par}

\title{Using Integrated Earth Observation-Informed Modeling to Inform Sustainable Development Decision-Making \\  
\large Ph.D. Thesis Proposal Abstract and Committee Members} 
\author{%
Jack Reid\\ 
Space Enabled Research Group\\
MIT Media Lab\\
jackreid@mit.edu}

\maketitle

\section*{ }
Dear Professor Machover and the MAS Committee,
\vspace{\baselineskip}

\noindent This is to inform you that Jack Reid intends to develop a thesis on the subject matter described in the below abstract and under the supervision of the listed committee, whose short biographies are available following the abstract.

\vspace{\baselineskip}

\textbf{\compr} \comprcredentials 
 
\vspace{\baselineskip}

\textbf{\comte} \comtecredentials

\vspace{\baselineskip}

\textbf{\comco} \comcocredentials

\thispagestyle{plain}
\pagestyle{plain}

%\tableofcontents

%%%%%%%%%%%%%%%%%%%%%%%%%%%%%%%%%%%%%%
\chapter*{\vspace{-2.0cm}Abstract}
%%%%%%%%%%%%%%%%%%%%%%%%%%%%%%%%%%%%%%

%[A short overview of the key goals, questions, and expected contributions.]
\vspace{-1.0cm}
Over the past two decades satellite-based remote observation has blossomed. We have seen a rapid increase in the number of \acp{eos} in orbit, significant improvements in their capabilities, and much greater availability of the data that they produce. This trend has occurred as part of a greater trend of increasing data availability, computational power, and modeling ability. Unfortunately, up until now, this \ac{eo} data has been largely used only by governments and academics for scientific purposes, typically to understand and predict environmental phenomena. Large corporations and \acp{ngo} have recently been conducting their own analyses, but these have required significant expertise and resources, and the results have sadly been mostly unavailable to the broader public. 

There is a real need for (a) making remote observation data not just available but accessible to a broader audience by developing data products that are relevant to everyday individuals, particularly those involved in local, rather than national or global decision-making; (b) linking the \ac{eo}-supported environmental modeling with the societal impact of a changing environment; and (c) putting policy and sensor design decision-making in the hands of a broader population. 

This work aims to demonstrate the viability of a particular methodology for achieving (a) and (b), while laying the groundwork for a more detailed consideration of (c). To that end, this work centers on exploring the efficacy and difficulties of \textbf{\textit{collaboratively developing}} a \textbf{\textit{systems-architecture-informed}}, multidisciplinary \textbf{\textit{\ac{gis} \ac{dss}}} for \textbf{\textit{sustainable development}} applications that makes significant use of \textbf{\textit{remote observation data}}. 

This is done through the development and evaluation of \acp{dss} for two primary applications: (1) mangrove forest management and conservation in the state of Rio de Janeiro, Brazil; and (2) coronavirus response in six metropolitan areas across Angola, Brazil, Chile, Indonesia, Mexico, and the United States. In both cases, the methodology involves the application the system architecture framework, an approach that has been previously adapted from the aerospace engineering discipline by Prof. Wood for use in sociotechnical systems. This includes using stakeholder mapping and network analysis to inform the design of the \ac{dss} in question. Other components of the methodology taken in this work are developing the \ac{dss} through an iterative and collaborative process with specific stakeholders; pursuing targeted, related analyses, such as on the value of certain ecosystem services, the value of remote sensing information, and human responses to various policies; and evaluating the usefulness of both the \ac{dss} and the development process through interviews, workshops, and other feedback mechanisms.

All of this takes place under the umbrella of the \ac{evdt} Modeling Framework for combining \ac{eo} and other types of data to inform decision-making in complex socio-environmental systems, particularly those pertaining to sustainable development. As the name suggests, \ac{evdt} integrates four models into one tool: the Environment (data including Landsat, Sentinel, VIIRs, Planet Lab’s PlanetScope, etc.; Human Vulnerability and Societal Impact (data including census and survey-based demographic data, NASA’s Socioeconomic Data and Applications Center, etc.); Human Behavior and Decision-Making (data including policy histories, mobility data, and urban nightlight data); and Technology Design for earth observation systems including satellites, airborne platforms and in-situ sensors (data including design parameter vectors for such systems). The data from each of these domains is used by established models in each domain, which are adapted to work in concert to address the needs identified during the stakeholder analysis. This framework is currently being used by several researchers in the Space Enabled Research Group and elsewhere. The capabilities provided by this framework will improve the management of earth observation and socioeconomic data in a format usable by non-experts, while harnessing cloud computing, machine learning, economic analysis, complex systems modeling, and model-based systems engineering.




%%%%%%%%%%%%%%%%%%%%%%%%%%%%%%%%%%%%%%
\chapter*{\vspace{-2.0cm}Thesis Committee Biographies}
%%%%%%%%%%%%%%%%%%%%%%%%%%%%%%%%%%%%%%

\subsection*{Prof. Danielle Wood}

Danielle Wood is an Assistant Professor in the Program in Media Arts \& Sciences and holds a joint appointment in the Department of Aeronautics \& Astronautics at \ac{mit}. Within the Media Lab, Prof. Wood leads the Space Enabled Research Group which seeks to advance justice in Earth's complex systems using designs enabled by space. Prof. Wood is a scholar of societal development with a background that includes satellite design, earth science applications, systems engineering, and technology policy. In her research, Prof. Wood applies these skills to design innovative systems that harness space technology to address development challenges around the world. Prior to serving as faculty at \ac{mit}, Professor Wood held positions at \ac{nasa} Headquarters, \ac{nasa} Goddard Space Flight Center, Aerospace Corporation, Johns Hopkins University, and the United Nations Office of Outer Space Affairs. Prof. Wood studied at \ac{mit}, where she earned a Ph.D. in engineering systems, S.M. in aeronautics and astronautics, S.M. in technology policy, and S.B. in aerospace engineering.

\subsection*{Prof. David Lagomasino}

David Lagomasino is an Assistant Professor in the Department of Coastal Studies at East Carolina University. He previously studied at Florida International University, where he received a B.S. and a Ph.D. in Geological Sciences, in between which he received a M.S. in Geology at East Carolina University. Lagomasino uses satellite, airborne, drone, and ground measurements to identify areas of coastal resilience and vulnerability. His research links remotely sensed spatial data directly with stakeholders in order to address exposure and sensitivity issues for coastal/wetland management and ecosystem valuation. He has been involved in a number of  coastal blue carbon projects with funding from NASA’s Carbon Monitoring Systems Program,  NASA’s Biodiversity and Forecasting Program, USDA’s National Forest Inventory Assessment Program,  NASA’s New Investigator Program, and the Center for International Forestry. His goal is to provide meaningful information that will better inform coastal management practices while also inspiring students and the community to become environmental stewards in order to help sustain our coastal resources. Prior to his current post, he conducted research at NASA’s Goddard Space Flight Center just outside Washington, D.C., in partnership with the University of Maryland, to develop models that measure the where when, and why shorelines are the world are changing.

\subsection*{Prof. Sarah Williams}

Sarah Williams is an Associate Professor of Technology and Urban Planning at \ac{mit} where she is also Director of the Civic Data Design Lab and the Leventhal Center for Advanced Urbanism. Williams’ combines her training in computation and design to create communication strategies that expose urban policy issues to broad audiences and create civic change. She calls the process Data Action, which is also the name of her recent book published by \ac{mit} Press. Williams is co-founder and developer of Envelope.city, a web-based software product that visualizes and allows users to modify zoning in New York City.  Before coming to \ac{mit}, Williams was Co-Director of the Spatial Information Design Lab at Columbia University’s Graduate School of Architecture Planning and Preservation (GSAPP). Her design work has been widely exhibited including work in the Guggenheim, the Museum of Modern Art (MoMA), Venice Biennale, and the Cooper Hewitt Museum. Williams has won numerous awards including being named one of the top 25 technology planners and Game Changer by Metropolis Magazine. 


\end{document}

% \end{abstractpage}

\clearpage
\hspace{0pt}
\vfill
\blockquote{\footnotesize
 God, grant me the insight to find and use models to understand the world around me, \newline The wisdom to acknowledge that they will someday fail, \newline And the strength to rid myself of them when it is apparent they no longer work.}

\hspace{40mm}-inspired by Ze Frank \& the Serenity Prayer

\vfill

\begin{singlespace}
%\begin{center}
\footnotesize   
\hspace*{52mm} To order, to govern, \newline
\hspace*{59mm} is to begin naming; \newline
\hspace*{55mm} when names proliferate \newline
\hspace*{60mm} it's time to stop. \newline
\hspace*{54mm} If you know when to stop \newline
\hspace*{59mm} you're in no danger. \newline
%\end{center}
\end{singlespace}


\hspace{40mm}-\textit{Tao Te Ching} by Laozi, adapted by Ursula K. Le Guin


\normalsize
\vfill
\hspace{0pt}

\cleardoublepage

\section*{Acknowledgments}

Before proceeding onto this work, I would like to thank several individuals and communities. First and foremost is my wife, Rebecca, who has been unfailingly supportive of me throughout the endeavor that has been multiple MIT graduate programs. First in a long-distance relationship and then in person, she has consistently buoyed my hopes and sense of self-worth when I needed it most.

Next I would like to thank the continuing experiment in cooperative living that is pika. It is there that I truly learned what a home is. Thanks for (almost) always having dinner ready at the end of the day and for filling the house with laughter. Similarly, I must thank the forbidden zone (tfz). This rotating cast of genuine characters kept me sane and happy throughout the pandemic even as we moved through three different houses. I do not know what else to say except that I genuinely miss those of who you have left already and will genuinely miss the rest of y'all whenever we spend more than a week apart.

%I would of course be highly remiss without thanking my advisor, Prof. Danielle Wood. Thank you for building such a wonderful research community and allowing me to take part in it. You have provided a space for morally important work to be done, for those who are interested in space for more than military or scientific purposes to find a research home, and for neglected perspectives to find voice. I wish you the best of fortune in your career to come.

This work was only possible through peer mentoring, support, and discovery. Ufuoma Ovienmhada, Seamus Lombardo, and Caroline Jaffe in particular were critical peers in developing and experimenting with EVDT. David Colby Reed also provided key discussion on ethics, governance, and framing that indelibly influenced not just this thesis, but my also my life.

I need to thank my committee members, Prof. Sarah Williams and Prof. David Lagomasino, for mentoring me in the various fields that I needed to complete this work (and become a better person). Be it teaching me how to use Google Earth Engine one summer in Goddard or piling me why with critical GIS books, I appreciate the time and attention given to my education.

I also need to thank Dr. Donna Rhodes, my masters thesis advisor. She taught me a great deal about how to think critically about my research work and how to present it to an audience. Her level of engagement and enthusiasm for my work has been much appreciated. These lessons and encouragement have stuck with me as I have pursued other fields of study.

In a very direct and literal sense, this thesis could not have been completed without many, many hours of involvement and support from my various international collaborators. this includes (among others): Prof Joga Setiawan (Diponegoro University) and Dr. Hanifa Denny (Diponegoro University) lead coordination for the Indonesia Vida work; Prof Joaquin Salas (Centro de Investigación en Ciencia Aplicada y Tecnología Avanzada, Universidad Querétaro) and Mr. Alejandro Monsivais (Mexican Space Agency) led coordination for the Mexico Vida work; Jose Guiridi (Ministerio de Ciencia, Tecnología, Conocimiento e Innovación) led coordination for the Chile Vida effort; and Zolana Joao, Gilson Santos, Eduina Teodoro, and Joana Caetano (Management Office of the National Space Program) led coordination for the Angola Vida work. In particular, however, I must extend my gratitude and affection towards Mr. Felipe Mandarino of \ac{ipp} in Rio de Janeiro, Brazil. Felipe has remained actively engaged with my work for several years now, amid multiple shifts in research direction, a pandemic, changes in governments (both here in the US and there in Brazil). He gave me a place to work in Rio de Janeiro, showed me around, and introduced me to the various folks that I need to speak to for this research work. I doubt that I will ever be able to repay him sufficiently for this.

Finally, I wish to state the following:

\blockquote{MIT and this author acknowledge Indigenous Peoples as the traditional stewards of the land, and the enduring relationship that exists between them and their traditional territories. The land on which this work was performed and these words were written is the traditional unceded territory of the Wampanoag Nation, Massachusett, and Nipmuc peoples. We acknowledge the painful history of genocide and forced occupation of their territory, and we honor and respect the many diverse indigenous people connected to this land on which we gather from time immemorial.}

The above statement is adapted from the \ac{mit} \ac{ipac} statement in partnership with \ac{mit}'s \ac{aises}, the Native American Students Association (NASA)\footnote{To avoid confusion with the \acf{nasa}, the Native American Students Association will always be written out fully in this work.}, and other Native American MIT students. This statement is particularly important for my work and for Space Enabled, because, while we pursue various forms of equity, justice, and sustainable development, including with other indigenous groups, such work does not allow us to absolve ourselves of our sins and responsibilities, both past and ongoing. As of this writing, we have not worked directly with the Wampanoag or Nipmuc peoples (nor do I know if they wish to work with us). Until genuine actions and not mere words are taken to address these atrocities, more work is required.


%%%%%%%%%%%%%%%%%%%%%%%%%%%%%%%%%%%%%%%%%%%%%%%%%%%%%%%%%%%%%%%%%%%%%%
% -*-latex-*-
