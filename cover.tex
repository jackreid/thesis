% -*-latex-*-
% 
% For questions, comments, concerns or complaints:
% thesis@mit.edu
% 
%
% $Log: cover.tex,v $
% Revision 1.9  2019/08/06 14:18:15  cmalin
% Replaced sample content with non-specific text.
%
% Revision 1.8  2008/05/13 15:02:15  jdreed
% Degree month is June, not May.  Added note about prevdegrees.
% Arthur Smith's title updated
%
% Revision 1.7  2001/02/08 18:53:16  boojum
% changed some \newpages to \cleardoublepages
%
% Revision 1.6  1999/10/21 14:49:31  boojum
% changed comment referring to documentstyle
%
% Revision 1.5  1999/10/21 14:39:04  boojum
% *** empty log message ***
%
% Revision 1.4  1997/04/18  17:54:10  othomas
% added page numbers on abstract and cover, and made 1 abstract
% page the default rather than 2.  (anne hunter tells me this
% is the new institute standard.)
%
% Revision 1.4  1997/04/18  17:54:10  othomas
% added page numbers on abstract and cover, and made 1 abstract
% page the default rather than 2.  (anne hunter tells me this
% is the new institute standard.)
%
% Revision 1.3  93/05/17  17:06:29  starflt
% Added acknowledgements section (suggested by tompalka)
% 
% Revision 1.2  92/04/22  13:13:13  epeisach
% Fixes for 1991 course 6 requirements
% Phrase "and to grant others the right to do so" has been added to 
% permission clause
% Second copy of abstract is not counted as separate pages so numbering works
% out
% 
% Revision 1.1  92/04/22  13:08:20  epeisach

% NOTE:
% These templates make an effort to conform to the MIT Thesis specifications,
% however the specifications can change. We recommend that you verify the
% layout of your title page with your thesis advisor and/or the MIT 
% Libraries before printing your final copy.
\title{Using Integrated Earth Observation-Informed Modeling to Inform Sustainable Development Decision-Making}

\author{Jack Reid}
% If you wish to list your previous degrees on the cover page, use the 
% previous degrees command:
%       \prevdegrees{A.A., Harvard University (1985)}
% You can use the \\ command to list multiple previous degrees
%       \prevdegrees{B.S., University of California (1978) \\
%                    S.M., Massachusetts Institute of Technology (1981)}
\department{Program in Media Arts and Sciences}

% If the thesis is for two degrees simultaneously, list them both
% separated by \and like this:
% \degree{Doctor of Philosophy \and Master of Science}
\degree{Doctor of Philosophy in Media Arts and Sciences}

% As of the 2007-08 academic year, valid degree months are September, 
% February, or June.  The default is June.
\degreemonth{December}
\degreeyear{2022}
\thesisdate{Dec 21, 2022}

%% By default, the thesis will be copyrighted to MIT.  If you need to copyright
%% the thesis to yourself, just specify the `vi' documentclass option.  If for
%% some reason you want to exactly specify the copyright notice text, you can
%% use the \copyrightnoticetext command.  
%\copyrightnoticetext{\copyright IBM, 1990.  Do not open till Xmas.}

% If there is more than one supervisor, use the \supervisor command
% once for each.
\supervisor{Danielle R. Wood}{ Assistant Professor of Media Arts and Sciences; Assistant Professor of Aeronautics and Astronautics}

% This is the department committee chairman, not the thesis committee
% chairman.  You should replace this with your Department's Committee
% Chairman.
\chairman{Tod Machover}{Chairman, Department Committee on Graduate Theses}

% Make the titlepage based on the above information.  If you need
% something special and can't use the standard form, you can specify
% the exact text of the titlepage yourself.  Put it in a titlepage
% environment and leave blank lines where you want vertical space.
% The spaces will be adjusted to fill the entire page.  The dotted
% lines for the signatures are made with the \signature command.
\maketitle

% The abstractpage environment sets up everything on the page except
% the text itself.  The title and other header material are put at the
% top of the page, and the supervisors are listed at the bottom.  A
% new page is begun both before and after.  Of course, an abstract may
% be more than one page itself.  If you need more control over the
% format of the page, you can use the abstract environment, which puts
% the word "Abstract" at the beginning and single spaces its text.

%% You can either \input (*not* \include) your abstract file, or you can put
%% the text of the abstract directly between the \begin{abstractpage} and
%% \end{abstractpage} commands.

% First copy: start a new page, and save the page number.
\cleardoublepage
% Uncomment the next line if you do NOT want a page number on your
% abstract and acknowledgments pages.
% \pagestyle{empty}
\setcounter{savepage}{\thepage}
\begin{abstractpage}
% $Log: abstract.tex,v $
% Revision 1.1  93/05/14  14:56:25  starflt
% Initial revision
% 
% Revision 1.1  90/05/04  10:41:01  lwvanels
% Initial revision
% 
%
%% The text of your abstract and nothing else (other than comments) goes here.
%% It will be single-spaced and the rest of the text that is supposed to go on
%% the abstract page will be generated by the abstractpage environment.  This
%% file should be \input (not \include 'd) from cover.tex.
This work aims to demonstrate the viability of a methodology for supporting local, sustainable development decision-making through the development of clearer linkages between environmental modeling and societal impact, with a particular emphasis on the use of earth observation data. To accomplish this, it explores the efficacy and difficulties of \textbf{\textit{collaboratively developing}} a \textbf{\textit{systems-architecture-informed}}, multidisciplinary \textbf{\textit{\acs{gis} decision support system}} for \textbf{\textit{sustainable development}} applications that makes significant use of \textbf{\textit{earth observation data}}. 

This is done through the development and evaluation of \acp{dss} for two applications: (1) mangrove forest management and conservation in the state of Rio de Janeiro, Brazil; and (2) coronavirus response in six regions around the world. In both cases, the methodology involves the application of the System Architecture Framework, which includes analyzing the stakeholders to inform the design of the \ac{dss} in question. Other components of the methodology are developing the \ac{dss} through a collaborative process with stakeholders; pursuing targeted analyses; and evaluating the usefulness of both the \ac{dss} and the development process through interviews, workshops, and other feedback mechanisms.

All of this takes place under the umbrella of the \ac{evdt} Modeling Framework for combining remote observation and other types of data to inform decision-making in complex socio-environmental systems, particularly those pertaining to sustainable development. As the name suggests, \ac{evdt} integrates four models into one tool: the Environment; Human Vulnerability and Societal Impact; Human Behavior and Decision-Making; and Technology Design for earth observation systems including satellites, airborne platforms and in-situ sensors. The data from each of these domains is used by established models in each domain, which are adapted to work in concert to address the needs identified during the stakeholder analysis. The capabilities provided by this framework will improve the management of earth observation and socioeconomic data in a format usable by non-experts, while harnessing cloud computing, machine learning, economic analysis, complex systems modeling, and model-based systems engineering.


\end{abstractpage}

% Additional copy: start a new page, and reset the page number.  This way,
% the second copy of the abstract is not counted as separate pages.
% Uncomment the next 6 lines if you need two copies of the abstract
% page.
% \setcounter{page}{\thesavepage}
% \begin{abstractpage}
% % $Log: abstract.tex,v $
% Revision 1.1  93/05/14  14:56:25  starflt
% Initial revision
% 
% Revision 1.1  90/05/04  10:41:01  lwvanels
% Initial revision
% 
%
%% The text of your abstract and nothing else (other than comments) goes here.
%% It will be single-spaced and the rest of the text that is supposed to go on
%% the abstract page will be generated by the abstractpage environment.  This
%% file should be \input (not \include 'd) from cover.tex.
This work aims to demonstrate the viability of a methodology for supporting local, sustainable development decision-making through the development of clearer linkages between environmental modeling and societal impact, with a particular emphasis on the use of earth observation data. To accomplish this, it explores the efficacy and difficulties of \textbf{\textit{collaboratively developing}} a \textbf{\textit{systems-architecture-informed}}, multidisciplinary \textbf{\textit{\acs{gis} decision support system}} for \textbf{\textit{sustainable development}} applications that makes significant use of \textbf{\textit{earth observation data}}. 

This is done through the development and evaluation of \acp{dss} for two applications: (1) mangrove forest management and conservation in the state of Rio de Janeiro, Brazil; and (2) coronavirus response in six regions around the world. In both cases, the methodology involves the application of the System Architecture Framework, which includes analyzing the stakeholders to inform the design of the \ac{dss} in question. Other components of the methodology are developing the \ac{dss} through a collaborative process with stakeholders; pursuing targeted analyses; and evaluating the usefulness of both the \ac{dss} and the development process through interviews, workshops, and other feedback mechanisms.

All of this takes place under the umbrella of the \ac{evdt} Modeling Framework for combining remote observation and other types of data to inform decision-making in complex socio-environmental systems, particularly those pertaining to sustainable development. As the name suggests, \ac{evdt} integrates four models into one tool: the Environment; Human Vulnerability and Societal Impact; Human Behavior and Decision-Making; and Technology Design for earth observation systems including satellites, airborne platforms and in-situ sensors. The data from each of these domains is used by established models in each domain, which are adapted to work in concert to address the needs identified during the stakeholder analysis. The capabilities provided by this framework will improve the management of earth observation and socioeconomic data in a format usable by non-experts, while harnessing cloud computing, machine learning, economic analysis, complex systems modeling, and model-based systems engineering.


% \end{abstractpage}

\clearpage
\hspace{0pt}
\vfill
\blockquote{\footnotesize
 God, grant me the insight to find and use models to understand the world around me, \newline The wisdom to acknowledge that they will someday fail, \newline And the strength to rid myself of them when it is apparent they no longer work.}

\hspace{40mm}-inspired by Ze Frank \& the Serenity Prayer

\vfill

\begin{singlespace}
%\begin{center}
\footnotesize   
\hspace*{52mm} To order, to govern, \newline
\hspace*{59mm} is to begin naming; \newline
\hspace*{55mm} when names proliferate \newline
\hspace*{60mm} it's time to stop. \newline
\hspace*{54mm} If you know when to stop \newline
\hspace*{59mm} you're in no danger. \newline
%\end{center}
\end{singlespace}


\hspace{40mm}-\textit{Tao Te Ching} by Laozi, adapted by Ursula K. Le Guin


\normalsize
\vfill
\hspace{0pt}

\cleardoublepage

\section*{Acknowledgments}

%Before proceeding onto this work, I would like to thank several individuals and communities. First and foremost is my wife, Rebecca, who has been unfailingly supportive of me throughout the endeavor that has been multiple MIT graduate programs. First in a long-distance relationship and then in person, she has consistently buoyed my hopes and sense of self-worth when I needed it most.
%
%Next I would like to thank the continuing experiment in cooperative living that is pika. It is there that I truly learned what a home is. Thanks for (almost) always having dinner ready at the end of the day and for filling the house with laughter. Similarly, I must thank the forbidden zone (tfz). This rotating cast of genuine characters kept me sane and happy throughout the pandemic even as we moved through three different houses. I do not know what else to say except that I genuinely miss those of who you have left already and will genuinely miss the rest of y'all whenever we spend more than a week apart.
%
%%I would of course be highly remiss without thanking my advisor, Prof. Danielle Wood. Thank you for building such a wonderful research community and allowing me to take part in it. You have provided a space for morally important work to be done, for those who are interested in space for more than military or scientific purposes to find a research home, and for neglected perspectives to find voice. I wish you the best of fortune in your career to come.
%
%This work was only possible through peer mentoring, support, and discovery. Ufuoma Ovienmhada, Seamus Lombardo, and Caroline Jaffe in particular were critical peers in developing and experimenting with EVDT. David Colby Reed also provided key discussion on ethics, governance, and framing that indelibly influenced not just this thesis, but my also my life.
%
%I need to thank my committee members, Prof. Sarah Williams and Prof. David Lagomasino, for mentoring me in the various fields that I needed to complete this work (and become a better person). Be it teaching me how to use Google Earth Engine one summer in Goddard or piling me why with critical GIS books, I appreciate the time and attention given to my education.
%
%I also need to thank Dr. Donna Rhodes, my masters thesis advisor. She taught me a great deal about how to think critically about my research work and how to present it to an audience. Her level of engagement and enthusiasm for my work has been much appreciated. These lessons and encouragement have stuck with me as I have pursued other fields of study.
%
%In a very direct and literal sense, this thesis could not have been completed without many, many hours of involvement and support from my various international collaborators. this includes (among others): Prof Joga Setiawan (Diponegoro University) and Dr. Hanifa Denny (Diponegoro University) lead coordination for the Indonesia Vida work; Prof Joaquin Salas (Centro de Investigación en Ciencia Aplicada y Tecnología Avanzada, Universidad Querétaro) and Mr. Alejandro Monsivais (Mexican Space Agency) led coordination for the Mexico Vida work; Jose Guiridi (Ministerio de Ciencia, Tecnología, Conocimiento e Innovación) led coordination for the Chile Vida effort; and Zolana Joao, Gilson Santos, Eduina Teodoro, and Joana Caetano (Management Office of the National Space Program) led coordination for the Angola Vida work. In particular, however, I must extend my gratitude and affection towards Mr. Felipe Mandarino of \ac{ipp} in Rio de Janeiro, Brazil. Felipe has remained actively engaged with my work for several years now, amid multiple shifts in research direction, a pandemic, changes in governments (both here in the US and there in Brazil). He gave me a place to work in Rio de Janeiro, showed me around, and introduced me to the various folks that I need to speak to for this research work. I doubt that I will ever be able to repay him sufficiently for this.
%
%Finally, I wish to state the following:
%
%\blockquote{MIT and this author acknowledge Indigenous Peoples as the traditional stewards of the land, and the enduring relationship that exists between them and their traditional territories. The land on which this work was performed and these words were written is the traditional unceded territory of the Wampanoag Nation, Massachusett, and Nipmuc peoples. We acknowledge the painful history of genocide and forced occupation of their territory, and we honor and respect the many diverse indigenous people connected to this land on which we gather from time immemorial.}
%
%The above statement is adapted from the \ac{mit} \ac{ipac} statement in partnership with \ac{mit}'s \ac{aises}, the Native American Students Association (NASA)\footnote{To avoid confusion with the \acf{nasa}, the Native American Students Association will always be written out fully in this work.}, and other Native American MIT students. This statement is particularly important for my work and for Space Enabled, because, while we pursue various forms of equity, justice, and sustainable development, including with other indigenous groups, such work does not allow us to absolve ourselves of our sins and responsibilities, both past and ongoing. As of this writing, we have not worked directly with the Wampanoag or Nipmuc peoples (nor do I know if they wish to work with us). Until genuine actions and not mere words are taken to address these atrocities, more work is required.


%%%%%%%%%%%%%%%%%%%%%%%%%%%%%%%%%%%%%%%%%%%%%%%%%%%%%%%%%%%%%%%%%%%%%%
% -*-latex-*-
