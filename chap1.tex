%% This is an example first chapter.  You should put chapter/appendix that you
%% write into a separate file, and add a line \include{yourfilename} to
%% main.tex, where `yourfilename.tex' is the name of the chapter/appendix file.
%% You can process specific files by typing their names in at the 
%% \files=
%% prompt when you run the file main.tex through LaTeX.
\chapter{Introduction}

A quick note on the user of first person pronouns in this piece. The word 'I' will obviously refer to the author, Jack Reid, and will be commonly used when describing work that I have done, arguments that I am asserting, etc. That said, the \ac{evdt} Framework and its various implementations, including the Vida \ac{dss}, were not solo projects but instead involved multiple contributors, both inside the Space Enabled Research Group and outside of it. Thus when I use 'we' when talking about \ac{evdt} I will be referring to this collection of individuals. Additionally, sometimes I will use 'we' to refer to the Space Enabled Research Group, particularly when discussing our group's set of methodologies and principles. Finally, on occassion, I may use 'we' in the general humanistic sense. I will strive to make in which sense I am using 'we' clear in context.

\section{Research Questions}

The work described in this thesis centers on exploring the efficacy and difficulties of \textit{collaboratively developing} a \textit{systems-architecture-informed}, multidisciplinary \textit{\ac{gis} \ac{dss}} for \textit{sustainable development} applications that makes significant use of \textit{remote observation data}. The meanings and histories of the emphasized terms will be explored further in later chapters of this thesis, along with the methods used in this exploration and the findings that resulted.

\section{Framing}

This piece is fundamentally about modeling, in particular, multidisciplinary modeling, and how modeling can inform actual action. Now individual models are inherently simplifications, intentional or otherwise, aimed at accomplishing a goal. They are metaphors for how the world really works, intended to enhance human faculties and focus our intention. Now the problem with such metaphors is that, as Elizabeth Ostrom puts it, "Relying on metaphors as the foundation for policy advice can lead to results substantially different from those presumed to be likely... One can get trapped in one's own intellectual web. When years have been spent in the development of a theory with considerable power and elegance, analysts obviously will want to apply this tool to as many situations as possible... Confusing a model with the theory of which it is one representation can limit applicability still further" \cite{ostromGoverningCommonsEvolution2015}. 

This is of course only compounded when multiple models from different domains are strung together, as will be described later. We must accordingly be focused on maintaining intellectual humility and avoid catching outself in our own web. Fortunately, such interdisciplinary humility is a key principle of the Space Enabled Research Group of which I am a part. We choose to practice a certain "theoretical pluralism" \cite{turkleEmpathyDiariesMemoir2021} in our methods, learning from those of different fields and not assuming that, merely becaues we have chose a certain approach, it is the only or the best possible approach.

In addition to our theoretical pluralism, we must also practice a humility in application. Much of our sustainable development work takes place in communities or even countries to which we are outsiders. There is a real danger that we rush in and prescribe the wrong solution to a problem that the community faces or misidentify the problem altogether. We could even to identify a problem were none, in fact exists, pathologizing the normal and natural, the Victorian England medical profession did to women \cite{duffinConspicuousConsumptiveWoman2012}. 

As is described further later, we strive to avoid this by allowing actual community members to identify the problem; by speaking with multiple community members to garner different perspectives; and, when possible, spending time in the community outselves. These latter two components are key, because even the member of a community may be afflicted with significant misaprenhensions about aspects of their own community, particularly of those who are seen inferior due to economic class, race, gender, education, or some other marker. Jane Jacob's described such a phenomena vividly in her classic text, \textit{The Death and Life of Great American Cities} \cite{jacobsDeathLifeGreat2016}:

\blockquote{Consider, for example the orthodox planning reaction to a district called the North End in Boston. This is an old, low-rent area merging into the heavy indsutry of the waterfront, and it is officially considered Boston's wost slum and civic shame... When I saw the North End again in 1959, I was amazed at the change. Dozens and dozens of buildings had been rehabilitated... The general street atmosphere of buoyancy, friendliness, and good health was so infectious that I began asking directions of people just for the fun of getting in on some talk. I had seen a lot of Boston in the past couple of days, most of it sorely distressing, and this struck me, with relief, as the healthiest place in the city... I called a Boston planner I know.

"Why in the world are you down in the North End?" he said, "That's a slum!... It has among the lowest delinquency, disease, and infant mortality rates in the city. It has has the lowest ratio of rent to income in the city... the child population is just above average for the city, on the nose. The death rate is low, 8.8 per thousand, against the average city rate of 11.2. The TB death rate is very low, less than 1 per ten thousand, [I] can't understand it, it's lower even than Brookline's. In the old days the North End used to be the city's worst spot for tuberculosis, but all that has changed. Well, they must be strong people. Of course it's a terrible slum."

"You should have more slums like this," I said.} 


\section{Methodology}

\section{Space Enabled Principles}


\section{Structure of Thesis}

Chapter 2 will layout the \ac{evdt} framework used through this work along with its theoretical underpinnings, motivation for its pursuit, and various potential critiques. Chapter 3 will lay out the methodology in more detail. Chapter 4...






%\section{Section sample 1}


%\begin{enumerate}
%  \item Item 1.
%  \item Item 2.
%  \item Item 3.
%\end{enumerate}
%
%
%
%\begin{eqnarray*}
%a_i & = & a_j + a_k \\
%a_i & = & 2a_j + a_k \\
%a_i & = & 4a_j + a_k \\
%a_i & = & 8a_j + a_k \\
%a_i & = & a_j - a_k \\
%a_i & = & a_j \ll m \mbox{shift}
%\end{eqnarray*}
%instead of the multiplication.  For example, you could use:
%\begin{eqnarray*}
%r & = & 4s + s\\
%r & = & r + r
%\end{eqnarray*}
%Or by xx:
%\begin{eqnarray*}
%t & = & 2s + s \\
%r & = & 2t + s \\
%r & = & 8r + t
%\end{eqnarray*}
