\chapter{Methodology} \label{ch:method}



\section{Mapping and Visualization}


"A single map is but one of an indefiniteliy large number of maps that might be produced for the same situation or from the same data." \cite{monmonierHowLieMaps1996}

Data maps have a long history. Tufte dates them to the seventeenth century and cites Edmond Halley's 1686 chart of trade winds as "one of the first data maps" \cite{tufteVisualDisplayQuantitative2001} though arguably Polynesian knot maps are a data map that long predates this [CITE].

Choropleths are one of the more common types of non-imagery geospatial data that \ac{evdt} uses. These are maps that express "quantity in area" (i.e. some statistic tied to a particular geographic area with color, texture, or shading). It should be noted that choropleths have a few well-known limitations, including the ecological fallacy and the modifiable areal unit problem \cite{cramptonRethinkingMapsIdentity2011}. It is for these reasons that \ac{evdt} does not rely entirely on choropleths and why we strive to store data with the finest geospatial resolution available.

Initial versions of \ac{evdt} and Vida featured quite large graphics. Tufte argues that graphics in general should be significantle shrunk and that "many data graphics can be reduced in area to half their current published size with virtually not loss in legibility and information." \cite{tufteVisualDisplayQuantitative2001} Inaccordance with this Shrink Principle, these graphics were greatly reduced in later versions.

Early verions of \ac{evdt} were structured as entirely object-oriented, and later versions remained primarily object-oriented. This has many advantages but also comes at certain costs, the most important of which include (a) difficulty in recording continuous spatial variables and (b) a requirement to pre-identify the different classes (objects) to sort phenomena and relationships into \cite{goodchildModelingEarth2011}. 

It is recognized that this desktop version comes with numerous downsides. \textit{theirwork}, an early collaborative, open source \ac{gis} platform, specifically "decided at an early stage to make the software Web-based to allow for a process of rapid development and iteration and allow a maximum number of potential participants." \cite{williamsonTheirworkDevelopmentSustainable2011} It should be noted, however, that \textit{theirwork} was a UK-based project (an area with high internet connectivity penetration) and started in the mid 2000's, a period with significantly diversity of internet browsing methods, which simplified the task of ensuring accessibility.

the meeting arrangment that EVDT supports, Table \ref{table:meeting_arrangements}

\begin{table}[h]
\caption[Different types of meeting arrangements]{Different types of meeting arrangements. Adapted from \cite{jankowskiGISGroupDecision2001}}
\label{table:meeting_arrangements}
\begin{center}
\begin{tabular}{ L{3cm} L{5.5cm}  L{5.5cm}}  \hline
 & \textit{Same time} &\textit{Different time}  \\ \cline{2-3}
\textbf{\textit{Same place}} & \textbf{Conventional Meeting} \qquad \textit{Advantage:} 
\vspace{-5mm}
\begin{itemize}
    \setlength{\itemsep}{0pt}%
    \setlength{\parskip}{0pt}%
	\item{face-to-face expressions}
	\item{immediate response}
\end{itemize} &
\textbf{Storyboard meeting} \qquad \textit{Advantage:} 
\vspace{-5mm}
\begin{itemize}
    \setlength{\itemsep}{0pt}%
    \setlength{\parskip}{0pt}%
	\item{scheduling is easy}
	\item{respond anytime}
	\item{leave-behind note}
\end{itemize} 
\\
& \textit{Disadvantage:} 
\vspace{-5mm}
\begin{itemize}
    \setlength{\itemsep}{0pt}%
    \setlength{\parskip}{0pt}%
	\item{scheduling is difficult}
\end{itemize} &
\textit{Disadvantage:} 
\vspace{-5mm}
\begin{itemize}
    \setlength{\itemsep}{0pt}%
    \setlength{\parskip}{0pt}%
	\item{meeting takes longer}
	\item{difficult to maintain in the long run}
\end{itemize} 
\\ \hline

\textbf{\textit{Different place}} & \textbf{Conference call meeting} \qquad \textit{Advantage:} 
\vspace{-5mm}
\begin{itemize}
    \setlength{\itemsep}{0pt}%
    \setlength{\parskip}{0pt}%
	\item{no need to travel}
	\item{immediate response}
\end{itemize} &
\textbf{Distributed meeting} \qquad \textit{Advantage:} 
\vspace{-5mm}
\begin{itemize}
    \setlength{\itemsep}{0pt}%
    \setlength{\parskip}{0pt}%
	\item{scheduling is convenient}
	\item{no need to travel}
	\item{submit response anytime}
\end{itemize} 
\\
& \textit{Disadvantage:} 
\vspace{-5mm}
\begin{itemize}
    \setlength{\itemsep}{0pt}%
    \setlength{\parskip}{0pt}%
	\item{limited personal perspective from participants}
	\item{meeting protocols are difficult to interpret}
	\item{difficult to maintain meeting dynamics}
\end{itemize} &
\textit{Disadvantage:} 
\vspace{-5mm}
\begin{itemize}
    \setlength{\itemsep}{0pt}%
    \setlength{\parskip}{0pt}%
	\item{meeting takes longer}
	\item{meeting dynamics are different from normal meeting ("netiquette" instead of face-to-face etiquette)}
\end{itemize} 
\\ \hline
\end{tabular}
\end{center}
\end{table}